\documentclass[a4paper]{article}

\usepackage{mathtools}
\usepackage{amsfonts}
\usepackage{amsmath}
\usepackage{amsthm}
\usepackage[a4paper, total={6in, 8in}]{geometry}
\usepackage[english]{babel}
\usepackage[utf8]{inputenc}
\usepackage{fancyhdr}
\usepackage[english]{babel}
\usepackage[utf8]{inputenc}
\usepackage{graphicx}
\usepackage{physics}
\usepackage[colorinlistoftodos]{todonotes}

\DeclarePairedDelimiter\ceil{\lceil}{\rceil}
\DeclarePairedDelimiter\floor{\lfloor}{\rfloor}

\DeclareMathOperator{\adj}{adj}
\DeclareMathOperator{\im}{im}
\DeclareMathOperator{\nullity}{nullity}
\DeclareMathOperator{\sign}{sign}
\DeclareMathOperator\dom{dom}
\DeclareMathOperator\lcm{lcm}
\DeclareMathOperator{\ran}{ran}
\DeclareMathOperator{\ext}{Ext}
\DeclareMathOperator{\dist}{dist}
\DeclareMathOperator{\diam}{diam}
\DeclareMathOperator{\aut}{Aut}
\DeclareMathOperator{\inn}{Inn}
\DeclareMathOperator{\syl}{Syl}
\DeclareMathOperator{\homo}{Hom}

\newcommand{\C}{\mathbb{C}}
\newcommand{\CP}{\mathbb{CP}}
\newcommand{\GG}{\mathbb{G}}
\newcommand{\F}{\mathbb{F}}
\newcommand{\N}{\mathbb{N}}
\newcommand{\Q}{\mathbb{Q}}
\newcommand{\R}{\mathbb{R}}
\newcommand{\RP}{\mathbb{RP}}
\newcommand{\T}{\mathbb{T}}
\newcommand{\Z}{\mathbb{Z}}
\renewcommand{\H}{\mathbb{H}}

\theoremstyle{definition}
\newtheorem{defn}{Definition}[subsection]
\newtheorem{axm}[defn]{Axiom}
\newtheorem{thm}[defn]{Theorem}
\newtheorem{prp}[defn]{Proposition}
\newtheorem{lmm}[defn]{Lemma}
\newtheorem{crl}[defn]{Corollary}

\raggedright

\pagestyle{fancy}
\fancyhf{}
\rhead{Labix}
\lhead{Kähler Manifolds}
\rfoot{\thepage}

\title{Kähler Manifolds}

\author{Labix}

\date{\today}
\begin{document}
\maketitle
\begin{abstract}
\end{abstract}
\pagebreak
\tableofcontents
\pagebreak

\section{Kähler Manifolds}
\subsection{Kähler Manifolds}
\begin{defn}{Kähler Manifolds}{} A Kähler metric is a Hermitian metric $h$ whose associated $(1,1)$-form $\omega$ is closed. 
\end{defn}

\begin{defn}{Kähler Manifolds}{} A Kähler manifold is a complex manifold $M$ with a Hermitian metric $h$ whose associated $(1,1)$-form $\omega$ is closed. 
\end{defn}

\begin{prp}{}{} Every Kähler manifold $M$ is a Riemannian manifold. \tcbline
\begin{proof}
We have seen that every hermitian metric induces a Riemannian metric. 
\end{proof}
\end{prp}

Let $M$ be a Kähler manifold with associated $(1,1)$-form $\omega$. Recall that we can write $\omega$ in local coordinates in $(U,\phi=(z_1,\dots,z_n))$ as $$\omega=\frac{i}{2}\sum_{i,j=1}^nh_{ij}dz_i\wedge d\overline{z}_j$$ where $h_{ji}=\overline{h}_{ij}$. This is the case even when $M$ is just a Hermitian manifold. With the Kähler structure, we can do more. $$\omega=\frac{i}{2}\sum_{i=1}^n\chi_i\wedge \overline{\chi}_i$$

\begin{prp}{}{} Let $M$ be a Kähler manifold with associated $(1,1)$-form $\omega$. Then $\omega^d/d!$ is the volume element of the Riemannian metric $g$ defined by the Kähler form $\omega$. 
\end{prp}

\begin{prp}{}{} Let $M$ be a Kähler manifold of complex dimension $n$. Then $$\dim(H^{2k}(M,\R))>0$$ for all $k=0,\dots,n$. 
\end{prp}




\end{document}