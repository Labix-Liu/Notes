\documentclass[a4paper]{article}

\input{C:/Users/liula/Desktop/Latex/Headers V1.2.tex}

\pagestyle{fancy}
\fancyhf{}
\rhead{Labix}
\lhead{Algebraic Topology 3}
\rfoot{\thepage}

\title{Algebraic Topology 3}

\author{Labix}

\date{\today}
\begin{document}
\maketitle
\begin{abstract}
Algebraic Topology 3 picks up from Algebraic Topology 2 and defines the final invariant for homotopy equivalence called the homotopy groups. We shall see that such homotopy groups is a complete invariant for CW-complexes up to homotopy equivalence. CW-complexes also benefit from the homotopy groups with the homotopy analogue of excision and a unique new theorem called the suspension theorem that implies stability of the homotopy groups. 
\end{abstract}
~\\~\\
References: 
\begin{itemize}
\item Notes on Algebraic Topology by Oscar Randal-Williams: \\
The first chapter gives a complete treatment of the first three sections of these notes, as well as providing the importance of fibrations on the higher homotopy groups. These notes are highly recommended to understanding the first three sections. 

\item Algebraic Topology by Allen Hatcher: \\
A more or less complete dictionary on all topics of these notes. However it is prone to the same problem in the sense that Hatcher's book is rather terse and definitions and parts of some theorems are scattered throughout the paragraphs rather than having a complete statement for reference. Nevertheless it is still the standard reference of the notes, albeit organized in a slightly different way. 

\item A non-visual proof that higher homotopy groups are abelian by Shintaro Fushida-Hardy: \\
This short piece of article proves that the higher homotopy groups are abelian in a purely algebraic way. Most geometric visualization of such a proof has the same underlying idea as the algebraic method. 
\end{itemize}

\pagebreak
\tableofcontents

\pagebreak

\section{The Higher Homotopy Groups}
The journey of Algebraic Topology began with the fundamental group, where we assigned a group to every space functorially. The notion of fundamental group heavily involves the notion of homotopy and therefore is heavily related to the notion of homotopy. However, one realizes that even with Seifert-van Kampen theorem and the theory of covering spaces, it is not easy to compute the fundamental group of a space. This is party, but not wholly due to the fundamental group is in general not abelian. If we instead work in an abelian setting, one is able to distinguish two non-isomorphic groups simply by analysing the torsion subgroups. Therefore we refine the concept of the fundamental group and procured the notion of homology and cohomology. Both functorial invariants now produce graded abelian groups for each space, one for each dimension $n\in\N$. In the case of cohomology, there is a canonical ring structure on cohomology that interacts with the topology of the underlying space. \\~\\

Now we turn to the final main invariant of topological spaces. The homotopy groups $\pi_n(X,x_0)$ serves as both a generalization of the fundamental group $\pi_1(X,x_0)$ in higher dimensions and a homotopic analogue to homology via the Hurewicz homomorphism $$h:\pi_n(X)\to H_n(X)$$ It is a strong invariant that is closely related to the notion of homotopy, all the while having mostly abelian groups as its output. The trade off is that the homotopy groups are very hard to compute. Such trade off has led to the blossoming of Algebraic Topology in its fullest. For instance, stable homotopy theory stems from a crucial fact called the Freudenthal suspension theorem, which states that such a sequence $$\pi_n(X)\to\pi_{n+1}(\Sigma X)\to\cdots$$ eventually stabilizes for large enough $n$. \\~\\

In this chapter we will closely study the $n$th homotopy groups such as its properties and develop tools to compute them. 

\subsection{The nth Homotopy Groups}
We begin not with the definition of the homotopy groups, but rather a slight generalization of pointed spaces and maps between them. 

\begin{defn}{Pairs of Space}{} Let $X$ be a topological space. A pair of space is a pair $(X,A)$ where $A\subseteq X$ is a subspace of $X$. A map of pairs $f:(X,A)\to(Y,B)$ is a continuous map $f:X\to Y$ such that $f(A)\subseteq B$. 
\end{defn}

\begin{defn}{Homotopy between Maps of Pairs}{} Let $f,g:(X,A)\to (Y,B)$ be maps of pairs. A homotopy between $f$ and $g$ is a homotopy $H:X\times[0,1]\to Y$ such that $H(A\times[0,1])\subseteq B$. 
\end{defn}

\begin{defn}{The nth Homotopy Groups}{} Let $(X,x_0)$ be a pointed space. Define the $n$th homotopy group $\pi_n(X,x_0)$ to be $$\pi_n(X,x_0)=\frac{\left\{\gamma:\left(I^n,\partial I^n\right)\to\left(X,\{x_0\}\right)\;\bigg{|}\;\gamma\text{ is continuous }\right\}}{\simeq}$$ where we say that $f\simeq g$ if there exists a homotopy between $f$ and $g$. 
\end{defn}

Notice that the definition coincides with that of the fundamental group when $n=1$, and hence $\pi_n$ is indeed a generalization. 

\begin{lmm}{}{} For any $n\in\N$, the two spaces $(I^n,\partial I^n)$ and $(S^n,s_0)$ are homotopy equivalent. 
\end{lmm}

Therefore an alternate viewpoint of the homotopy groups is instead the collection of maps from the pointed $n$-sphere to the space $X$ quotient homotopy. Indeed an $n$-dimensional sphere has an $n$-dimensional hole enclosed by the sphere itself. Therefore in order to detect $n$-dimensional holes in a space, we are permitted to try and fit $n$-spheres into the space. \\~\\

Spheres are also advantageous for the definition of $\pi_n$ because spheres only has an $n$-dimensional hole and no other holes in any dimension. Therefore we are capturing the minimal amount of information on $n$-dimensional holes without producing excess data. \\~\\

Now we have defined the set $\pi_n(X,x_0)$ for a pointed space to have the word group in its name. We will also need to procure a canonical group structure on the set $\pi_n(X,x_0)$. This will be similar with that of the fundamental group. 

\begin{defn}{Concatenation}{} Let $n\geq 1$. Let $(X,x_0)$ be a pointed space. Let $f,g:(I^n,\partial I^n)\to(X,x_0)$ be maps. Define the composition of $f$ and $g$ by the formula $$(f\cdot g)(t_1,\dots,t_n)=\begin{cases}
f(2t_1,t_2,\dots,t_n) & \text{ if } 0\leq t_1\leq\frac{1}{2}\\
g(2t_1-1,t_2,\dots,t_n) & \text{ if } \frac{1}{2}\leq t\leq 1
\end{cases}$$ for $f,g\in\pi_n(X,x_0)$. 
\end{defn}

Notice that concatenation is really just the same concatenation between elements of the fundamental group but instead with more coordinates. The group structure on $\pi_n(X,x_0)$ uses concatenation and such a proof also uses the same homotopies as in Algebraic Topology 1, but with more coordinates. 

\begin{thm}{}{} Let $(X,x_0)$ be a pointed space and $n\geq 1$. The operation $\cdot$ on the equivalence classes in $\pi_n(X,x_0)$ is well defined and endows it with the structure of a group. \tcbline
\begin{proof}
We first show that the operation is well defined on $\pi_n(X,x_0)$. Suppose that $f_1\overset{\partial}{\simeq} g_1:(I^n,\partial I^n)\to(X,x_0)$ via the homotopy $H_1$ and $f_2\overset{\partial}{\simeq} g_2:(I^n,\partial I^n)\to(X,x_0)$ via the homotopy $H_2$. Consider the map $H:I^n\times[0,1]\to X$ defined by $$H(x_1,\dots,x_n,t)=\begin{cases}
H_1(2x_1,\dots,x_n,t) & \text{if } 0\leq x_1\leq\frac{1}{2}\\
H_2(2x_1-1,\dots,x_n,t) & \text{if }\frac{1}{2}\leq x_1\leq 1
\end{cases}$$ 
Now when $t=0$, we have that $H(x_1,\dots,x_n,0)=f_1\cdot f_2$. When $t=1$, we have that $H(x_1,\dots,x_n,1)=g_1\cdot g_2$. Now notice that by definition of $H_1$ and $H_2$, if one of $x_1,\dots,x_n$ is equal to $0$ or $1$, then $H_1$ and $H_2$ is constant and maps to $x_0$. This means that $H$ also has such property and hence $H$ is a homotopy $(I,\partial I^n)$ to $(X,x_0)$. \\~\\

We now have an appropriate binary operation on $\pi_n(X,x_0)$. It is clearly associative since the composition of maps are associativity and one can re-parametrize homotopies with different traversal speeds. I claim that the constant map $e_{x_0}:(I,\partial I^n)\to(X,x_0)$ defined by $e_{x_0}(x)=x_0$ is the identity. Let $f:(I^n,\partial I^n)\to(X,x_0)$ be arbitrary. Define the homotopy from $e_{x_0}\cdot f$ to $f$ by $$H(x_1,\dots,x_n,t)=\begin{cases}
e_{x_0}(x_1,\dots,x_n)=x_0 & \text{ if }0\leq x_1\leq\frac{1-t}{2}\\
f\left(\frac{2s+t-1}{t+1}\right) & \text{ if }\frac{1-t}{2}\leq x_1\leq 1
\end{cases}$$ A similar homotopy proves that $f\cdot e_{x_0}\simeq f$. For the inverse, I claim that $\overline{f}:(I^n,\partial I^n)\to(X,x_0)$ defined by $\overline{f}(1-x_1,\dots,x_n)$ is the inverse of $f$. Indeed, define a homotopy from $f\cdot\overline{f}$ to $e_{x_0}$ by $$H(x_1,\dots,x_n,t)=\begin{cases}
e_{x_0}(x_1,\dots,x_n)=x_0 & \text{ if }0\leq x_1\leq\frac{t}{2}\text{ or }\frac{1-t}{2}\leq x_1\leq 1\\
f(2x_1-t,x_2,\dots,x_n) & \text{ if }\frac{t}{2}\leq x_1\leq\frac{1}{2}\\
\overline{f}(2s+t-1) & \text{ if }\frac{1}{2}\leq x_1\leq\frac{1-t}{2}
\end{cases}$$
\end{proof}
\end{thm}

However, what makes each $\pi_n(X,x_0)$ for $n\geq 2$ different from the fundamental group $\pi_1(X,x_0)$ is the abelian group structure on $\pi_n(X,x_0)$. 

\begin{thm}{}{} Let $(X,x_0)$ be a pointed space. Then the $n$th homotopy group $$\pi_n(X,x_0)$$ together with concatenation is abelian. \tcbline
\begin{proof}
Define a new operation $\star:\pi_n(X,x_0)\times\pi_n(X,x_0)\to\pi_n(X,x_0)$ by $$[f]\star[g]=\begin{cases}
f(t_1,2t_2,\dots,t_n) & \text{ if } 0\leq t_1\leq\frac{1}{2}\\
g(t_1,2t_2-1,\dots,t_n) & \text{ if } \frac{1}{2}\leq t\leq 1
\end{cases}$$ Such an operation clearly also defines an abelian group structure on $\pi_n(X,x_0)$ using the same argument. Now I want to prove that $$([f]\ast[g])\star([h]\ast [k])=([f]\star [h])\ast([g]\star [k])$$ This is true because $$([f]\ast[g])\star([h]\ast [k])=\begin{cases}
f(2x_1,2x_2,x_3,\dots,x_n) & \text{ if } 0\leq x_1,x_2\leq\frac{1}{2}\\
g(2x_1,2x_2-1,x_3,\dots,x_n) & \text{ if } 0\leq x_1\leq\frac{1}{2} \text{ and } \frac{1}{2}\leq x_2\leq 1\\
h(2x_1-1,2x_2,x_3,\dots,x_n) & \text{ if } \frac{1}{2}\leq x_1\leq 1 \text{ and } 0\leq x_2\leq\frac{1}{2}\\
k(2x_1,2x_2-1,x_3,\dots,x_n) & \text{ if } \frac{1}{2}\leq x_1,x_2\leq 1
\end{cases}$$ and $$([f]\star [h])\ast([g]\star [k])=\begin{cases}
f(2x_1,2x_2,x_3,\dots,x_n) & \text{ if } 0\leq x_1,x_2\leq\frac{1}{2}\\
h(2x_1-1,2x_2,x_3,\dots,x_n) & \text{ if } \frac{1}{2}\leq x_1\leq 1 \text{ and } 0\leq x_2\leq\frac{1}{2}\\
g(2x_1,2x_2-1,x_3,\dots,x_n) & \text{ if } 0\leq x_1\leq\frac{1}{2} \text{ and } \frac{1}{2}\leq x_2\leq 1\\
k(2x_1,2x_2-1,x_3,\dots,x_n) & \text{ if } \frac{1}{2}\leq x_1,x_2\leq 1
\end{cases}$$ which are entirely the same. Now I claim that $\ast=\star$. It is clear that both binary operations have the same identity element $e_{x_0}$. Now we have that $$f\ast g=(f\star 1)\ast (1\star g)=(f\ast 1)\star(1\ast g)=f\star g$$ Finally, I claim that $\ast$ is commutative. We have that $$f\ast g=(1\star f)\ast(g\star 1)=(1\ast g)\star(f\ast 1)=g\star f=g\ast f$$ Thus we conclude. 
\end{proof}
\end{thm}

The above technique is actually called the Eckmann-Hilton argument. In particular, it shows that concatenation of paths need not be defined via the first coordinate. Any choice of coordinate to perform concatenation will result in the same group structure. \\~\\

Geometrically speaking, 

\subsection{Properties of Homotopy}
The homotopy groups also satisfy functorial properties similar to the fundamental group and the (co)homology groups. 

\begin{thm}{Functoriality}{} Let $(X,x_0)$ and $(Y,y_0)$ be pointed spaces and let $f:(X,x_0)\to(Y,y_0)$ be a pointed map. Then the induced map $$\pi_n(f):\pi_n(X,x_0)\to\pi_n(Y,y_0)$$ defined by $[\gamma]\mapsto[f\circ\gamma]$ is a group homomorphism. Moreover, it satisfies the following functorial properties. 
\begin{itemize}
\item If $g:(Y,y_0)\to(Z,z_0)$ is a pointed map then $$\pi_n(g\circ f)=\pi_n(g)\circ\pi_n(f)$$
\item If $\text{id}_{(X,x_0)}:(X,x_0)\to(X,x_0)$ is the identity map then $$\pi_n(\text{id}_{(X,x_0)})=\text{id}_{\pi_n(X,x_0)}$$
\end{itemize} \tcbline
\begin{proof}
Firstly, let us show that it is a group homomorphism. Let $\gamma_1,\gamma_2\in\pi_n(X,x_0)$. We have that $$\pi_n(f)([\gamma_1]\cdot[\gamma_2])=[f\circ(\gamma_1\cdot\gamma_2)]=[f\circ\gamma_1\cdot f\circ\gamma_2]=\pi_n(f)([\gamma_1])\cdot\pi_n(f)([\gamma_2])$$ where the second equality is true because homotopies are preserved under function composition. It remains to show associativity and unitality. 
\begin{itemize}
\item Associativity: We have that $$\pi_n(g\circ f)([\gamma])=[g\circ f\circ\gamma]=\pi_n(g)([f\circ\gamma])=(\pi_n(g)\circ\pi_n(f))([\gamma])$$
\item Unitality: We have that $$\pi_n(\text{id}_{(X,x_0)})([\gamma])=[\text{id}_{(X,x_0)}\circ\gamma]=[\gamma]=\text{id}_{\pi_n(X,x_0)}([\gamma])$$
\end{itemize}
And so we conclude. 
\end{proof}
\end{thm}

Similar to all other functorial properties we have seen throughout algebraic topology, a homeomorphism of spaces give an isomorphism on homotopy groups. Now that we know about category theory, we see that such a result does not depend on the definition of the homotopy groups or the (co)homology groups, but is in fact due to the functorial properties of each invariant. \\~\\

Similar to (co)homology and the fundamental group, the homotopy groups are defined via a quotient with homotopy. Therefore we expect the homotopy groups to not be able to distinguish between homotopy equivalent spaces but not homeomorphic spaces. 

\begin{thm}{Homotopy Equivalence}{} Let $(X,x_0),(Y,y_0)$ be pointed spaces and $f,g:(X,x_0)\to (Y,y_0)$ be pointed maps. If $f$ and $g$ are homotopic, then the induced maps $$\pi_n(f)=\pi_n(g):\pi_n(X,x_0)\to\pi_n(Y,y_0)$$ are equal. Moreover, if $f$ is a homotopy equivalence, then $\pi_n(f)$ is an isomorphism. \tcbline
\begin{proof}
Let $[\gamma]\in\pi_n(X,x_0)$. Suppose that $f$ and $g$ are homotopic via $F:X\times I\to Y$. now define $$H(x_1,\dots,x_n,t)=F(\gamma(x_1,\dots,x_n),t)$$ Then it is clear that $H(x_1,\dots,x_n,0)=f\circ\gamma$ and $H(x_1,\dots,x_n,1)=g\circ\gamma$. Thus $[f\circ\gamma]=[g\circ\gamma]$ and so we conclude that $\pi_n(f)([\gamma])=\pi_n(g)([\gamma])$. \\~\\

If $f$ is a homotopy equivalence, then there exists $g:(Y,y_0)\to(X,x_0)$ such that $g\circ f\simeq\text{id}_{(X,x_0)}$ and $f\circ g\simeq\text{id}_{(Y,y_0)}$. By funtoriality and homotopy equivalence, we have that $$\pi_n(g)\circ\pi_n(f)=\text{id}_{\pi_n(X,x_0)}\;\;\;\;\text{ and }\;\;\;\;\pi_n(f)\circ\pi_n(g)=\text{id}_{\pi_n(Y,y_0)}$$ and so we conclude. 
\end{proof}
\end{thm}

While the theory of covering spaces provided great insight for the structure of the fundamental group as well the space itself, the theory no longer works for higher homotopy groups due to the following proposition. 

\begin{prp}{}{} Let $(X,x_0)$ be a pointed space and let $p:(\tilde{X},\tilde{x}_0)\to(X,x_0)$ be a covering space. Then $p$ induces isomorphisms $$\pi_n(p):\pi_n(\tilde{X},\tilde{x}_0)\overset{\cong}{\longrightarrow}\pi_n(X,x_0)$$ for all $n\geq 2$. 
\end{prp}

While covering spaces no longer prove to be useful for insights on the homotopy groups, fibrations will be the correct analogue of covering spaces to computing the higher homotopy groups. In fact, covering spaces themselves are also fibrations. We will see fibrations in later sections. \\~\\

Similar to the fundamental group, changing the base point via a path induces isomorphisms on homotopy groups with the same space but different base point. 

\begin{thm}{}{} Let $(X,x_0)$ and $(X,x_1)$ be pointed spaces with the same base space. Let $u:I\to X$ be a path from $x_0$ to $x_1$. Define the induced map $$u_\#:\pi_n(X,x_0)\to\pi_n(X,x_1)$$ as follows. For $[\gamma]\in\pi_n(X,x_0)$ define $u_\#([\gamma])$ by first shrinking the domain of $\gamma$ to a smaller concentric cube in $I^n$. Then inserting the path $\gamma$ on each radical segment of the shell between the smaller cube and $\partial I^n$. \\~\\
The construction of $u_\#$ is a group isomorphism. Moreover, it satisfies the following universal properties. 
\begin{itemize}
\item If $v:I\to X$ is a path from $x_1$ to $x_2$ and $u\cdot v$ is the concatenation of these paths, then $$(u\cdot v)_\#=u_\#\circ v_\#$$
\item If $c_{x_0}$ is the constant path from $x_0$ to $x_0$ then $(c_{x_0})_\#$ is the identity
\end{itemize}
\end{thm}

\begin{prp}{}{} Let $(X,x_0)$ and $(X,x_1)$ be pointed spaces with the same base space. Let $u,v:I\to X$ be paths from $x_0$ to $x_1$. If $u$ and $v$ are homotopic relative to end points then the induced maps $$u_\#=v_\#:\pi_n(X,x_0)\to\pi_n(X,x_1)$$ are equal. 
\end{prp}

This shows that if $X$ is path connected, then $\pi_n(X,x_0)$ no longer depends on the choice of base point. Although there are no canonical isomorphisms between $\pi_n(X,x_0)$ and $\pi_n(X,x_1)$, we still forget about the base point in this case and write the homotopy groups as $\pi_n(X)$. 

\begin{prp}{}{} Let $(X,x_0)$ be a pointed space and $f\in\pi_n(X,x_0)$. Let $u:I\to X$ be a loop on $x_0$. Then $u$ induces a left action of $\pi_1(X,x_0)$ on $\pi_n(X,x_0)$ by the map $$(u,\gamma)\mapsto u_\#(\gamma)$$ In particular, for $n\geq 2$, $\pi_n(X,x_0)$ is a $\Z\pi_1(X,x_0)$-module. 
\end{prp}

\begin{prp}{}{} Let $X_i$ for $i\in I$ be a family of path connected spaces. Then there are isomorphisms $$\pi_n\left(\prod_{i\in I}X_i\right)\cong\prod_{i\in I}\pi_n(X_i)$$
\end{prp}

\subsection{Relative Homotopy Groups}
\begin{defn}{Triplets of Spaces}{} Let $X$ be a topological space. A pointed pair of space is a triple $(X,A_1,A_2)$ where $A_2\subseteq A_1\subseteq X$ are subspaces of $X$. A map between triplets of spaces $f:(X,A_1,A_2)\to(Y,B_1,B_2)$ is a map $f:X\to Y$ such that $f(A_1)\subseteq B_1$ and $f(A_2)\subseteq B_2$. \\~\\
If $A_2=\{x_0\}$ is a single point we say that $(X,A,x_0)$ is a pointed pair of spaces. 
\end{defn}

\begin{defn}{Homotopy between Maps of Triplets}{} Let $f,g:(X,A_1,A_2)\to(Y,B_1,B_2)$ be maps triplets of spaces. A homotopy between $f$ and $g$ is a homotopy between $f:X\to Y$ and $g:X\to Y$, namely $H:X\times[0,1]\to Y$ such that $H(A_1\times[0,1])\subseteq B_1$ and $H(A_2\times[0,1])\subseteq B_2$. 
\end{defn}

\begin{defn}{The nth Relative Homotopy Groups}{} Let $(X,A,x_0)$ be a pointed pair of space. Let $n\geq 2$. Regard $I^{n-1}$ sitting inside $I^n$ by $I^{n-1}=\{(x_1,\dots,x_n)\in I^n\;|\;x_n=0\}$ and let $J^{n-1}=\overline{\partial I^n\setminus I^{n-1}}$. Define the relative homotopy groups of the triple by $$\pi_n(X,A,x_0)=\frac{\left\{\gamma:\left(I^n,\partial I^n,J^{n-1}\right)\to\left(X,A,x_0\right)\;\bigg{|}\;\gamma\text{ is continuous }\right\}}{\simeq}$$ where we say that $f\simeq g$ if there exists a homotopy between $f$ and $g$. 
\end{defn}

It is easy to see that $\pi_n(X,x_0,x_0)=\pi_n(X,x_0)$ so that homotopy groups are a special case of the relative homotopy groups. 

\begin{lmm}{}{} For any $n\in\N$, the two triplets $(I^n,\partial I^n,J^{n-1})$ and $(D^n,S^{n-1},s_0)$ are homotopy equivalent. 
\end{lmm}

\begin{thm}{}{} Let $(X,A,x_0)$ be a pointed pair of space. The composition law on definition 1.1.4 defines a group structure on $\pi_n(X,A,x_0)$ for $n\geq 2$. Moreover, $\pi_n(X,A,x_0)$ is abelian for $n\geq 3$. 
\end{thm}

\subsection{Induced Maps of Relative Homotopy Groups}
\begin{thm}{}{} Let $(X,A,x_0)$ and $(Y,B,y_0)$ be pointed pairs of spaces and $f:(X,A,x_0)\to(Y,B,y_0)$ a map. Then $f$ induces a map on the relative homotopy groups $$f_\ast:\pi_n(X,A,x_0)\to\pi_n(Y,B,y_0)$$ for $n\geq 2$ satisfying the following functorial properties: 
\begin{itemize}
\item $f_\ast$ is a group homomorphism
\item If $g:(Y,B,y_0)\to(Z,C,z_0)$ is a map, then $$(g\circ f)_\ast=g_\ast\circ f_\ast$$
\item If $\text{id}_{(X,A,x_0)}$ is the identity map on $(X,A,x_0)$, then $$(\text{id}_{(X,A,x_0)})_\ast=\text{id}_{\pi_n(X,A,x_0)}$$
\end{itemize}
\end{thm}

\begin{thm}{}{} Let $(X,A,x_0),(Y,B,y_0)$ be pointed pairs of spaces and $f,g:(X,A,x_0)\to (Y,B,y_0)$ be pointed maps. If $f$ and $g$ are homotopic, then the induced maps $$f_\ast=g_\ast:\pi_n(X,A,x_0)\to\pi_n(Y,B,y_0)$$ are equal. Moreover, if $f$ is a homotopy equivalence, then $f_\ast$ is an isomorphism. 
\end{thm}

TBA: change of base point isomorphisms. 

\begin{thm}{The Hurewicz Homomorphism}{} Let $(X,A,x_0)$ be a pointed pair of space. Let $u_n$ be a generator of $H_n(S^n)\cong\Z$. Then the map $$h:\pi_n(X,A,x_0)\to H_n(X,A)$$ defined by $[f]\mapsto f_\ast(u_n)$ is a group homomorphism. 
\end{thm}

\subsection{Long Exact Sequence in Homotopy Groups}
\begin{lmm}{Compression Criterion}{} Let $(X,A,x_0)$ be a pair of spaces with basepoint. Let $f:(D^n,S^{n-1},\ast)\to(X,A,x_0)$ be a map. Then $[f]=[e_{x_0}]\in\pi_n(X,A,x_0)$ if and only if $$(f:D^n\to X)\overset{S^{n-1}}{\simeq}(g:D^n\to X)$$ where $g$ is any map such that $g(X)\subseteq A$. \tcbline
\begin{proof}
Suppose that the second criterion is satisfied. Then it clearly shows that $[f]=[g]\in\pi_n(X,A,x_0)$. Let $r:D^n\times I\to D^n$ be a deformation retract from $D^n$ to $\ast\in S^{n-1}\subset D^n$. Consider the map $g\circ r:D^n\times I\to X$. When $t=0$, this is the map $g$. When $t=1$, $g\circ r$ factors through $\ast$ and so becomes a map $\ast\to X$. In other words, it is the constant map $e_{x_0}$. Moreover, it $g\circ r$ has image in $A$ and so in particular it sends $S^{n-1}$ to $A$. Thus $g\circ r$ is a homotopy between $\text{e}_{x_0}$ and $g$. We conclude that $[f]=[g]=[e_{x_0}]$. \\~\\

Now suppose that $[f]=[e_{x_0}]\in\pi_n(X,A,x_0)$ is given by the homotopy $H:D^n\times I\to X$. This means that $H(D^n\times\{1\})\subseteq\{x_0\}\subset A$ and $H(S^{n-1}\times I)\subset A$. Now $D^n\times I$ deformation retracts to the cup $D^n\times\{1\}\cup S^{n-1}\times I$ by radical projection from the center point of $D^n\times\{0\}$. Thus $H$ can be converted into a map from $D^n\times\{1\}\cup S^{n-1}\times I$ to $X$. Then $H$ is now a homotopy from $f$ to a map $H(-,1):D^n\to X$ which has image in $A$, relative to $S^{n-1}$. Thus we conclude. 
\end{proof}
\end{lmm}

\begin{thm}{}{} Let $X$ be a space and $A,B$ be subspaces of $X$ such that $B\subseteq A\subseteq X$. Let $x_0\in B$. Then there is a long exact sequence in relative homotopy groups: \\~\\
\adjustbox{scale=0.88,center}{\begin{tikzcd}
	\cdots & {\pi_n(A,B,x_0)} & {\pi_n(X,B,x_0)} & {\pi_n(X,A,x_0)} & {\pi_{n-1}(A,B,x_0)} & \cdots & {\pi_1(X,A,x_0)}
	\arrow[from=1-1, to=1-2]
	\arrow["{i_\ast}", from=1-2, to=1-3]
	\arrow["{j_\ast}", from=1-3, to=1-4]
	\arrow["{\partial_n}", from=1-4, to=1-5]
	\arrow[from=1-5, to=1-6]
	\arrow[from=1-6, to=1-7]
\end{tikzcd}}\\~\\
where $i:(A,B,x_0)\to(X,B,x_0)$ and $j:(X,B,x_0)\to(X,A,x_0)$ are the inclusions and $\partial:\pi_n(X,A,x_0)\to\pi_{n-1}(A,B,x_0)$ is given by $[\gamma]\mapsto[\gamma|_{I^{n-1}}]$ \tcbline
\begin{proof}{}{}
\end{proof}
\end{thm}

TBA: Naturality of the sequence. 

\begin{thm}{}{} Let $(X,A,x_0)$ be a pointed pair of spaces. The relative homotopy groups and (absolute) homotopy groups of $(X,A,x_0)$ fit into a long exact sequence \\~\\
\adjustbox{scale=0.75,center}{\begin{tikzcd}
	\cdots & {\pi_{n+1}(X,A,x_0)} & {\pi_n(A,x_0)} & {\pi_n(X,x_0)} & {\pi_n(X,A,x_0)} & {\pi_{n-1}(A,x_0)} & \cdots & {\pi_0(X,x_0)} & 0
	\arrow[from=1-1, to=1-2]
	\arrow["{\partial_{n+1}}", from=1-2, to=1-3]
	\arrow["{i_\ast}", from=1-3, to=1-4]
	\arrow["{j_\ast}", from=1-4, to=1-5]
	\arrow["{\partial_n}", from=1-5, to=1-6]
	\arrow[from=1-6, to=1-7]
	\arrow[from=1-8, to=1-9]
	\arrow[from=1-7, to=1-8]
\end{tikzcd}}\\~\\
where $\partial_n$ is defined by $[f]\mapsto [f|_{I^{n-1}}]$ and $i_\ast$ and $j_\ast$ are induced by inclusions. 
\end{thm}

Note that even though at the end of the sequence group structures are not defined, exactness still makes sense: kernels in this case consists of elements that map to the homotopy class of the constant map. 

\begin{thm}{}{} Let $p:E\to B$ be a fiber bundle. Let $A\subseteq B$. Let $y_0\in E$ and $p(y_0)=x_0$. Then there is an isomorphism $$\pi_n(E,p^{-1}(A),y_0)\cong\pi_n(B,A,x_0)$$ given by the induced map $p_\ast$ for all $n\geq 2$. 
\end{thm}

\subsection{Homotopical Connectivity}
\begin{defn}{n-Connected Space}{} Let $X$ be a space. We say that it is $n$-connected if $$\pi_k(X,x_0)=0$$ for $0\leq k\leq n$ and some $x_0\in X$. 
\end{defn}

Note that $\pi_0(X,x_0)$ implies that $X$ is path connected. Hence the notion of $n$-connectedness does not depend on the base point by the change of base point isomorphism. In particular, $\pi_k(X,x_0)=0$ for $0\leq k\leq n$ and some $x_0\in X$ if and only if $\pi_k(X,x_0)=0$ for $0\leq k\leq n$ for all $x_0\in X$. (Hatcher)

\begin{defn}{n-Connected Pair of Spaces}{} Let $(X,A)$ be a pair of space. We say that it is $n$-connected if the following are true. 
\begin{itemize}
\item $\pi_k(X,A,x_0)=0$ for $0<k\leq n$ and all $x_0\in A$. 
\item $\pi_0(\iota):\pi_0(A)\to\pi_0(X)$ is surjective. 
\end{itemize}
\end{defn}

TBA: conditions in P.346 of Hatcher

\begin{defn}{Weakly Contractible}{} Let $X$ be a space. We say that $X$ is weakly contractible if $$\pi_n(X)=0$$ for all $n\geq 0$. 
\end{defn}

\pagebreak
\section{Weak Equivalences and CW-Complexes}
\subsection{Weak Homotopy Equivalence}
\begin{defn}{Weak Homotopy Equivalence}{} We say that a map $f:X\to Y$ is a weak homotopy equivalence if it induces isomorphisms on all homotopy groups $\pi_n$ on any choice of base point. 
\end{defn}

TBA: compression lemma in Hatcher

\begin{thm}{}{} Let $X,Y$ be spaces and let $f:X\to Y$ be a weak homotopy equivalence. Then $f$ induces isomorphisms $$f_\ast:H_n(X;G)\overset{\cong}{\longrightarrow}H_n(Y;G)\;\;\;\;\text{ and }\;\;\;\;f^\ast:H^n(Y;G)\overset{\cong}{\longrightarrow}H^n(X;G)$$ for any group $G$ and all $n\in\N$. 
\end{thm}

This theorem shows that the higher homotopy groups is not a weaker invariant than homology and cohomology. Indeed, the theorem states that if the all homotopy groups are isomorphic, then all their (co)homology groups will be isomorphic. 

\begin{prp}{}{} Let $X,Y$ be spaces and let $f:X\to Y$ be a weak homotopy equivalence. Then $f$ induces bijections $$[Z,X]\cong[Z,Y]\;\;\;\;\text{ and }\;\;\;\;[Z,X]_\ast\cong[Z,Y]_\ast$$ for all CW-complexes $Z$. 
\end{prp}

\subsection{Whitehead's Theorem}
\begin{thm}{Whitehead's Theorem}{} If $X$ and $Y$ are CW-complexes and $f:X\to Y$ is a weak homotopy equivalence, then $f$ is a homotopy equivalence. 
\end{thm}

TBA: extension lemma in Hatcher. 

\begin{crl}{}{} If $X$ and $Y$ are CW-complexes with $\pi_1(X)=\pi_1(Y)=0$ and $f:X\to Y$ induces isomorphisms on homology groups $H_n$ for all $n$, then $f$ is a homotopy equivalence. 
\end{crl}

\subsection{Cellular Approximations}
\begin{defn}{Cellular Maps}{} Let $X$ and $Y$ be CW-complexes. A map $f:X\to Y$ is called cellular if $f(X_n)\subset Y_n$ for all $n$, where $X_n$ is the $n$-skeleton of $X$. 
\end{defn}

\begin{defn}{Cellular Approximations}{} Let $X$ and $Y$ be CW-complexes. We say that $f:X\to Y$ has a cellular approximations if $f$ is homotopic to a cellular map $f':X\to Y$. 
\end{defn}

To this end we need to revisit the notion of polyhedra. 

\begin{defn}{Convex Polyhedra}{} Let $n\in\N$. A convex polyhedra is a subset $S$ of $\R^n$ of the form $$S=\left\{x=(x_1,\dots,x_n)\in\R^n\;\bigg{|}\;\sum_{k=1}^na_{k,1}x_k\leq b_1,\dots,\sum_{k=1}^n a_{k,s}x_k\leq b_s\right\}$$ for some $a_{1,1},\dots,a_{n,s},b_1,\dots,b_s\in\R$. 
\end{defn}

\begin{lmm}{}{} Let $X$ be a CW complex. Let $Y\subseteq X$ be a CW subcomplex. Let $Z$ be obtained by attaching a cell $e^k$ to $Y$. Let $f:I^n\to Z$ be a map. Then there exists a map $g:I^n\to X$ such that $$f\overset{f^{-1}(Y)}{\simeq} g$$ and $g$ is such that the following is true. There exists a simplex $\Delta^k\subset e^k$ and polyhedra $P_1,\dots,P_d$ such that $g$ is the (possibly empty) union $$g^{-1}(\Delta^k)=\bigcup_{k=1}^n$$ and $g|_{P_t}$ is the restriction of a linear surjection $\R^n\to\R^k$ for all $t$. 
\end{lmm}

\begin{thm}{Cellular Approximation Theorem}{} Any map $f:X\to Y$ between CW-complexes has a cellular approximation $f':X\to Y$. Moreover, if $f$ is already cellular on a subcomplex $A\subseteq X$, then we can take $f'|_A=f|_A$. 
\end{thm}

\begin{thm}{Relative Cellular Approximation}{} Any map $f:(X,A)\to (Y,B)$ between pairs of CW-complexes has a cellular approximation. 
\end{thm}

\begin{crl}{}{} Let $A\subset X$ be CW-complexes and suppose that all cells $X\setminus A$ have dimension larger than $n$. Then $(X,A)$ is $n$-connected. 
\end{crl}

\begin{crl}{}{} Let $X$ be a CW complex and let $X^n$ be its $n$-skeleton. Then $(X,X^n)$ is $n$-connected. Moreover, the inclusion $X^n\hookrightarrow X$ induces an isomorphism $$\pi_k(X^n)\to\pi_k(X)$$ for $0\leq k<n$ and a surjection for $k=n$. 
\end{crl}

\subsection{CW Approximations}
\begin{defn}{CW Approximation}{} Let $X$ be a space. A CW approximation of $X$ is a weak homotopy equivalence $f:Z\to X$ where $Z$ is a CW complex. 
\end{defn}

The goal of this section is that every space has a CW approximation. The given homotopy equivalence makes this notion powerful because this means that for any space $X$, there exists a CW-complex such that $X$ and $Z$ are homotopy equivalent, and moreover, has isomorphic homotopy, homology and cohomology groups. 

\begin{defn}{CW Model}{} Let $(X,A)$ be a non-empty pair of CW-complexes. An $n$-connected CW model of $(X,A)$ is an $n$-connected CW pair $(Z,A)$ together with a map $f:Z\to X$ with $f|_A=\text{id}_A$ such that $$f_\ast:\pi_i(Z)\to\pi_i(X)$$ is an isomorphism for $i>n$ and an injection for $i=n$ for any choice of base point. 
\end{defn}

\begin{thm}{}{} For any non-empty pair $(X,A)$ of CW-complexes, there exists an $n$-connected model $(Z,A)$. Moreover, $Z$ can be built from $A$ by attaching cells of dimension greater than $n$. 
\end{thm}

\begin{thm}{}{} Every pair of spaces $(X,A)$ has a CW approximation. Such a CW approximation is unique up to homotopy equivalence. 
\end{thm}

\pagebreak

\section{Main Results of Homotopy Theory on CW-Complexes}
\subsection{Excision for Homotopy Groups}
\begin{thm}{The Homotopy Excision Theorem (Blaker's Massey Theorem)}{} Let $X$ be a CW-complex and $A,B$ be sub complexes such that $X=A\cup B$ and $A\cap B\neq\emptyset$. If $(A,A\cap B)$ is $m$-connected and $(B,A\cap B)$ is $n$-connected for $m,n\geq 0$, then the map $$\pi_k(\iota):\pi_k(A,A\cap B)\to\pi_k(X,B)$$ induced by the inclusion $\iota:(A,A\cap B)\to(X,B)$ is an isomorphism for $0\leq k<m+n$ and a surjection for $k=m+n$. \tcbline
\begin{proof}
We prove this by considering successively more general cases, starting from the simplest one. \\~\\

Case 1: $A$ is obtained from $A\cap B$ by attaching some $e_\alpha^{m+1}$-cells and $B$ is obtained from $A\cap B$ by attaching one $e^{n+1}$-cell. \\
We want to show that the map $\pi_k(A,A\cap B)\to(X,B)$ is surjective. Let $f:(I^k,\partial I^k,J^{k-1})\to(X,B,x_0)$ represent an equivalence class in $\pi_k(X,B)$. Since $f$ is continuous, $f$ preserves compactness and hence the image of $f$ is compact. By a property of CW complexes, $\im(f)$ meets only finitely many of the cells $e_\alpha^{m+1}$ and $e^{n+1}$. 
\end{proof}
\end{thm}

\begin{prp}{}{} Let $(X,A)$ be a pair of $r$-connected CW complexes and let $A$ be $s$-connected. Then the map $$p_\ast:\pi_k(X,A)\to\pi_k(X/A)$$ induced by the quotient map $p:X\to X/A$ is an isomorphism for $0\leq k\leq r+s$ and a surjection for $k=r+s+1$. 
\end{prp}

\subsection{Hurewicz's Theorem}
\begin{thm}{Hurewicz's Homomorphism}{} Let $X$ be a path connected space. Then for any $n\in\N$, there is a group homomorphism $$h_n:\pi_n(X)\to H_n(X)$$ called the Hurewicz homomorphism, defined as follows. Let $[u_n]\in H_n(S^n)$ be a canonical generator. Then $h_n([f])=f_\ast(u_n)$. 
\end{thm}

\begin{thm}{Hurewicz's Theorem}{} Let $X$ be a space. Then the following are true regarding Hurewicz's homomorphism. 
\begin{itemize}
\item Let $n\geq 2$. If $X$ is $(n-1)$-connected, then $\widetilde{H}_k(X)=0$ for all $0\leq k<n$. Moreover, the Hurewicz homomorphism $$h_n:\pi_n(X)\to H_n(X)$$ is an isomorphism. Moreover, $h_{n+1}$ is a surjection. 
\item Let $n=1$, then Hurewicz's homomorphism induces an isomorphism $$\overline{h_1}:\pi_1(X)^\text{ab}\to H_1(X)$$ 
\end{itemize}
\end{thm}

\begin{thm}{Relative Hurewicz's Homomorphism}{} Let $(X,A)$ be a pair of spaces. Then for any $n\geq 1$, there is a group homomorphism $$h_n:\pi_n(X,A)\to H_n(X,A)$$ called the relative Hurewicz homomorphism, defined as follows. Let $[u_n]\in H_n(S^n,\partial S^n)$ be a canonical generator. Then $h_n([f])=f_\ast(u_n)$. 
\end{thm}

\begin{thm}{Relative Hurewicz's Theorem}{} Let $(X,A)$ be a pair of spaces. Let $n\geq 2$. If $X$ and $A$ are path connected and $(X,A)$ is $(n-1)$-connected, then $H_k(X,A)=0$ for all $0\leq k<n$. Moreover, the Hurewicz homomorphism $$h_n:\pi_n(X,A,x_0)\to H_n(X,A)$$ is an isomorphism. 
\end{thm}

\begin{thm}{Naturality of Hurewicz's Homomorphism}{} Let $(X,x_0)$ and $(Y,y_0)$ be pointed spaces and let $f:(X,x_0)\to(Y,y_0)$ be a map. Then the following diagram is commutative: \\~\\
\adjustbox{scale=1,center}{\begin{tikzcd}
	{\pi_k(X,x_0)} & {\pi_k(Y,y_0)} \\
	{H_k(X)} & {H_k(Y)}
	\arrow["{\pi_k(f)}", from=1-1, to=1-2]
	\arrow["{h_k}"', from=1-1, to=2-1]
	\arrow["{h_k}", from=1-2, to=2-2]
	\arrow["{f_\ast}"', from=2-1, to=2-2]
\end{tikzcd}}\\~\\
where $h$ is the Hurewicz homomorphism. Moreover, a similar diagram is also commutative for the relative Hurewicz homomorphism. 
\end{thm}

The connection between the homotopy groups and the homology groups begs the question of whether there is a relationship between the homotopy groups and cohomology groups that is not implicit by the relation between homology and cohomology. This is answered in Stable Homotopy Theory, when we introduced Brown's representability theorem. 

\subsection{Eilenberg-MacLane Spaces}
\begin{defn}{Eilenberg-MacLane Space}{} Let $G$ be a group and $n\in\N$. We say that a space $X$ is an Eilenberg-MacLane space of type $K(G,n)$ if $$\pi_k(X)=\begin{cases}
K(G,n) & \text{ if } k=n\\
0 & \text{ otherwise }
\end{cases}$$
\end{defn}

We often denote this space $X$ directly by $X=K(G,n)$. 

\begin{prp}{}{} Let $G$ be a group. Then there exists a $K(G,1)$-CW complex. 
\end{prp}

\begin{thm}{}{} Let $G$ be an abelian group and $n\geq 2$. Then there exists a $K(G,n)$-CW complex. Moreover, it is uniquely determined by $G$ and $n$. 
\end{thm}

The Eilenberg-Maclane spaces are a fundamental object of study in algebraic topology because it is a universal object. This is again part of Stable Homotopy Theory and is the same theorem that gives the connection between homotopy groups and cohomology groups. \\~\\

We will not prove this here, but we will give the theorem: If $G$ is an abelian group, then there are natural isomorphisms $$H^n(X;G)\cong[X,K(G,n)]_\ast$$ that is natural in the following sense. If $f:X\to Y$ is a map, then there is a commutative diagram: \\~\\
\adjustbox{scale=1,center}{\begin{tikzcd}
	{H^n(Y;G)} & {H^n(X;G)} \\
	{[Y,K(G,n)]_\ast} & {[X,K(G,n)]_\ast}
	\arrow["{f^\ast}", from=1-1, to=1-2]
	\arrow["\cong"', from=1-1, to=2-1]
	\arrow["\cong", from=1-2, to=2-2]
	\arrow["{f^\ast}"', from=2-1, to=2-2]
\end{tikzcd}}\\~\\

\pagebreak
\section{Relation Between Homotopy Groups and (Co)Homology Groups}
\subsection{Weak Equivalences and (Co)Homological Isomorphisms}
\begin{prp}{}{} Let $X,Y$ be spaces. Let $A$ be an abelian group. Let $f:X\to Y$ be a weak equivalence. Then $$f_\ast:H_n(X;A)\to H_n(Y;A)\;\;\;\;\text{ and }\;\;\;\;f^\ast:H^n(Y;A)\to H^n(X;A)$$ are isomorphisms for all $n$. 
\end{prp}

\subsection{Cohomology Theory with Weak Equivalences}
\begin{defn}{Generalized Cohomology Theory for Spaces}{} A Generalized cohomology theory is a collection of contravariant functors and natural transformations $$h^n:\bold{Top}^2\to\bold{Ab}\;\;\;\;\;\;\;\text{ and }\;\;\;\;\;\;\;\delta^n:h^n\circ F\Rightarrow h^{n+1}(X,A)$$ where $F(X,A)=(A,\emptyset)$, for each $n\in\N$, satisfying the above first four axioms and the following. 
\begin{itemize}
\item Weak Equivalence: If $f:(X,A)\to(Y,B)$ is a weak equivalence, then $$f_\ast:h^n(Y,B)\to h^n(X,A)$$ is an isomorphism. 
\end{itemize}
\end{defn}


\pagebreak
\section{The Stable Phenomena}
\subsection{Freudenthal Suspension Theorem}
\begin{thm}{Freudenthal Suspension Theorem}{} Let $X$ be an $n$-connected CW complex. Then for $0\leq k\leq 2n$, the induced map $$\Sigma_\ast:\pi_k(X)\to\pi_{k+1}(\Sigma X)$$ is an isomorphism. For $k=2n+1$, $\Sigma_\ast$ is a surjection. 
\end{thm}

We can keep on suspending the space and the maps. Indeed if $X$ is $n$-connected then, by Freudenthal suspension theorem $\Sigma X$ is $(n+1)$-connected. We can then apply the suspension theorem again on $\Sigma X$ and we see that $\Sigma^2X$ is $(n+2)$-connected. \\

The following theorem is also said to be the Freudenthal suspension theorem. 

\begin{thm}{}{} Let $Y$ be $(n-1)$-connected. Consider the reduced suspension functor $\Sigma:\bold{hTop}_\ast\to\bold{hTop}_\ast$. Then $\Sigma:[X,Y]\to[\Sigma X,\Sigma Y]$ is bijective if $\dim(X)<2n-1$. Moreover, it is a surjection if $\dim(X)=2n-1$. 
\end{thm}

\begin{crl}{}{} There is an isomorphism $$\pi_{n+k}(S^n)\cong\pi_{n+k+1}(S^{n+1})$$ for all $n\geq k+2$. 
\end{crl}

\begin{prp}{}{} Let $X$ be a space. Let $k\in\N$. Then the the following sequence of suspensions $$\pi_k(X)\to\pi_{k+1}(\Sigma X)\to\pi_{k+2}(\Sigma^2X)\to\cdots$$ are eventually isomorphisms. \tcbline
\begin{proof}
Let $X$ be $n$-connected. There are two cases. \\~\\

Let $k\leq 2n$. By Freudenthal suspension theorem, if $k\leq 2n$ then $\pi_k(X)\cong\pi_{k+1}(\Sigma X)$. Then $\Sigma X$ is $(n+1)$-connected hence $\pi_{k+1}(\Sigma X)\cong\pi_{k+2}(\Sigma^2X)$ is an isomorphism since $k+1\leq 2n+2$. More generally, for $r\in\N$, $\Sigma^rX$ is $(r+n)$-connected hence $$\pi_{k+r}(\Sigma^rX)\cong\pi_{k+r+1}(\Sigma^{r+1}X)$$ is an isomorphism since $k+r\leq 2n+2r$. \\~\\

Now if $k>2n$, then there exists $r\in\N$ such that $k+r\leq 2n+2r$. Such an $r$ is given by say $k-2n$. Then by Freudenthal suspension theorem, $$\pi_{k+r}(\Sigma^rX)\cong\pi_{k+r+1}(\Sigma^{r+1}X)$$ is an isomorphism. More generally, for $m\in\N$, $\Sigma^{r+m}X$ is $(r+m+n)$-connected hence $$\pi_{k+r+m}(\Sigma^{r+m}X)\cong\pi_{k+r+m+1}(\Sigma^{r+m+1}X)$$ is an isomorphism since $k+r+m\leq 2n+2r+2m$. 
\end{proof}
\end{prp}

\subsection{The Stable Homotopy Groups}
In Algebraic Topology, we refer to a property or invariant being stable if applying the suspension functor to the space does not change the property (up to possibly a shift in index). The first such example is one we have already encountered. 

\begin{defn}{Stable Homotopy Groups}{} Let $X$ be a space. Let $n\in\N$. Define the $n$th stable homotopy groups of $X$ to be $$\pi_n^s(X)=\colim_{k\to\infty}\pi_{n+k}(\Sigma^kX)$$
\end{defn}

This is well defined because of the Freudenthal suspension theorem, which states that the groups in the direct limit eventually stabilize. Indeed, the same theorem shows the following. 

\begin{prp}{}{} Let $X$ be a space. Then there is an isomorphism $$\pi_n^s(X)\cong\pi_{n+1}^s(\Sigma X)$$ induced by the suspension functor between stable homotopy groups for any $n\in\N$. \tcbline
\begin{proof}
This can be seen by simply unwinding the definitions. We have that 
\begin{align*}
\pi_{n+1}^s(\Sigma X)&=\colim_{k\to\infty}\pi_{n+k+1}(\Sigma^{k+1}X)\\
&=\colim_{k\to\infty}\pi_{n+k}(\Sigma^kX)\\
&=\pi_n^s(X)
\end{align*}
and so we conclude. 
\end{proof}
\end{prp}

New: Graded ring structure on $\pi_\ast^s$. \\

Recall that for $n\geq 2$, the set $[\Sigma^2X,Z]$ for any spaces $X$ and $Z$ are abelian. Moreover, the suspension map $\Sigma:[\Sigma^2X,Z]\to[\Sigma^3X,\Sigma Z]$ is a group homomorphism. In particular, we can choose $Z=\Sigma^2Y$. Now since the category $\bold{Ab}$ of abelian groups is cocomplete, the following inverse system \\~\\
\adjustbox{scale=1,center}{\begin{tikzcd}
	{[\Sigma^2 X,\Sigma^2Y]} & {[\Sigma^3X,\Sigma^3Y]} & {[\Sigma^4X,\Sigma^4Y]} & \cdots
	\arrow["\Sigma", from=1-1, to=1-2]
	\arrow["\Sigma", from=1-2, to=1-3]
	\arrow[from=1-3, to=1-4]
\end{tikzcd}}\\~\\
has an inverse limit. This leads to the following definition. 

\begin{defn}{Set of Stable Homotopy Classes of Maps}{} Let $X$ and $Y$ be space. The set of stable homotopy classes of maps from $X$ to $Y$ is defined to be the abelian group $$[X,Y]^s=\colim_{n\in\N\setminus\{0,1\}}[\Sigma^nX,\Sigma^nY]$$
\end{defn}

The following observation is crucial. When $X$ and $Y$ are pointed CW complexes, the abelian groups $[\Sigma^n X,\Sigma^n Y]$ also become stable under suspensions. 

\begin{thm}{}{} Let $X$ and $Y$ be pointed CW complexes. Then the sequence of abelian groups, \\~\\
\adjustbox{scale=1,center}{\begin{tikzcd}
	{[\Sigma^nX,\Sigma^nY]} & {[\Sigma^{n+1}X,\Sigma^{n+1}Y]} & \cdots
	\arrow["\Sigma", from=1-1, to=1-2]
	\arrow[from=1-2, to=1-3]
\end{tikzcd}}\\~\\
are isomorphic under the suspension functor for $n>\dim(X)$ and $n\geq 2$. 
\end{thm}

New: When $X$ is compact, $[X,QY]_\ast=[X,Y]_\ast^s$. 

\begin{thm}{}{} The stable homotopy groups define a collection of functors $\pi_n^s:\bold{CW}_\ast\to\bold{Ab}$ as follows. 
\begin{itemize}
\item For $X$ a pointed CW complex, $\pi_n^s(X)$ is the $n$th stable homotopy group of $X$
\item For $f:X\to Y$ a map, $$\pi_n^s(f):\pi_n^s(X)\to\pi_n^s(Y)$$ is the image of $f$ under the canonical map $[X,Y]\to[X,Y]^s$
\end{itemize}
\end{thm}

\begin{thm}{}{} The stable homotopy functors $\pi_n^s:\bold{CW}_\ast\to\bold{Ab}$ for each $n\in\N$ defines a reduced homology theory. 
\end{thm}

This is untrue for the unstable homotopy groups. In particular, the fundamental group is not necessarily abelian. 

\pagebreak
\section{Bonus?}
\subsection{Postnikov Towers}
\begin{defn}{Postnikov Towers}{} Let $X$ be a path connected space. A Postnikov tower is the following commutative diagram \\~\\
\adjustbox{scale=1.0,center}{\begin{tikzcd}
	& X \\
	\\
	\cdots & {X_n} & {X_{n-1}} & \cdots & {X_2} & {X_1} & \ast
	\arrow[from=1-2, to=3-2]
	\arrow[from=1-2, to=3-3]
	\arrow[from=1-2, to=3-5]
	\arrow[from=1-2, to=3-6]
	\arrow[from=3-1, to=3-2]
	\arrow["{p_n}"', from=3-2, to=3-3]
	\arrow[from=3-3, to=3-4]
	\arrow[from=3-4, to=3-5]
	\arrow["{p_2}"', from=3-5, to=3-6]
	\arrow["{p_1}"', from=3-6, to=3-7]
\end{tikzcd}}\\~\\
such that the following are true. 
\begin{itemize}
\item The maps $X\to X_n$ for each $n\in\N$ induces isomorphisms $\pi_i(X)\cong\pi_i(X_n)$ for $i\leq n$. 
\item $\pi_i(X_n)$ for $i>n$. 
\item Each $p_n:X_n\to X_{n-1}$ for $n\in\N$ is a fibration with fiber $K(\pi_n(X),n)$. 
\end{itemize}
\end{defn}

\begin{thm}{}{} Suppose that there is an inverse system of spaces \\~\\
\adjustbox{scale=1.0,center}{\begin{tikzcd}
	& \lim_{n\to\infty}X_n \\
	\\
	\cdots & {X_n} & {X_{n-1}} & \cdots & {X_2} & {X_1} & \ast
	\arrow[from=1-2, to=3-2]
	\arrow[from=1-2, to=3-3]
	\arrow[from=1-2, to=3-5]
	\arrow[from=1-2, to=3-6]
	\arrow[from=3-1, to=3-2]
	\arrow["{p_n}"', from=3-2, to=3-3]
	\arrow[from=3-3, to=3-4]
	\arrow[from=3-4, to=3-5]
	\arrow["{p_2}"', from=3-5, to=3-6]
	\arrow["{p_1}"', from=3-6, to=3-7]
\end{tikzcd}}\\~\\
The functor $\pi_i$ for $i\in\N$ induces a cone in $\bold{Grp}$. By definition of $\lim_{\leftarrow}\pi_i(X_n)$, there is a unique map $$\lambda:\pi_i\left(\lim_{\leftarrow}X_n\right)\to\lim_{\leftarrow}\pi_i(X_n)$$ Then the following are true regarding $\lambda$. 
\begin{itemize}
\item $\lambda$ is surjective
\item $\lambda$ is injective if the maps $\pi_{i+1}(X_n)\to\pi_{i+1}(X_{n-1})$ are surjective for sufficient large $n$. 
\end{itemize}
\end{thm}

\begin{prp}{}{} Let $X$ be a connected CW complex. Then there exists a Postnikov tower for $X$. 
\end{prp}

\begin{prp}{}{} Let $X$ be a connected CW complex. Choose a Postnikov tower of $X$. Then there is a weak homotopy equivalence $$X\simeq\lim_{\leftarrow}X_n$$ so that $X$ is a CW approximation of $\lim_{\leftarrow}X_n$. 
\end{prp}








\end{document}
