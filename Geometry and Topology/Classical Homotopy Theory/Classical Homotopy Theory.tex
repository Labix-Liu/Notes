\documentclass[a4paper]{article}

\input{C:/Users/liula/Desktop/Latex/Headers V1.2.tex}

\pagestyle{fancy}
\fancyhf{}
\rhead{Labix}
\lhead{Classical Homotopy Theory}
\rfoot{\thepage}

\title{Classical Homotopy Theory}

\author{Labix}

\date{\today}
\begin{document}
\maketitle
\begin{abstract}
\begin{itemize}
\item Notes on Algebraic Topology by Oscar Randal-Williams
\end{itemize}
\end{abstract}
\pagebreak
\tableofcontents

\pagebreak
\section{A Convenient Category of Spaces}
Reason: 
\begin{itemize}
\item Want $-\wedge-$ associative and unital and commutative (so that the category is symmetric monoidal)
\item Want adjunction $-\times X$ and $\Hom_{\mC}(X,-)$ (non pointed)
\item Want adjunction $-\wedge X$ and $\text{Map}_\ast(X,-)$ (pointed) (Intuitively, $X\wedge Y$ represents maps from $X\times Y$ that are base point preserving separately in each variable)
\end{itemize}


\subsection{Compactly Generated Spaces}
\begin{defn}{Compactly Generated Spaces}{} Let $X$ be a space. We say that $X$ is compactly generated ($k$-space) if for every set $A\subseteq X$, $A$ is open if and only if $A\cap K$ is open in $K$ for every compact subspace $K\subseteq X$. 
\end{defn}

\begin{defn}{Category of Compactly Generated Spaces}{} Define the category of compactly generated spaces $\bold{CG}$ to be the full subcategory of $\bold{Top}$ consisting of spaces that are compactly generated. In other words, $\bold{CG}$ consists of the following data: 
\begin{itemize}
\item $\text{Obj}(\bold{CG})$ consists of all spaces that are compactly generated. 
\item For $X,Y\in\text{Obj}(\bold{CG})$, the morphisms are $$\Hom_{\bold{CG}}(X,Y)=\Hom_{\bold{Top}}(X,Y)$$
\item Association is given by composition of functions. 
\end{itemize}
Define similarly the category of pointed compactly generated spaces $\bold{CG}_\ast$. 
\end{defn}

\begin{defn}{New $k$-space from Old}{} Let $X$ be a space. Define $k(X)$ to be the set $X$ together with the topology defined as follows: $A\subseteq X$ is open if and only if $A\cap K$ is open in $K$ for every compact subspace $K\subseteq X$. 
\end{defn}

\begin{lmm}{}{} Let $X$ be a space. Then $k(X)$ is a compactly generated space. 
\end{lmm}

Unfortunately $X\times Y$ may not be compactly generated even when $X$ and $Y$ are. But as it turns out, products do exists in $\bold{CG}$ and are given by $X\times_{\bold{CG}}Y=k(X\times_{\bold{Top}} Y)$. 

\begin{prp}{}{} Let $X,Y$ be compactly generated spaces. Then the categorical product of $X$ and $Y$ in the category of compactly generated spaces is given by $$X\times_{\bold{CG}}Y=k(X\times_{\bold{Top}} Y)$$
\end{prp}

\begin{prp}{}{} Every CW complex is compactly generated. 
\end{prp}

\begin{defn}{Category of Compactly Generated and Weakly Hausdorff Spaces}{} Define the category of compactly generated and weakly Hausdorff spaces $\bold{CGWH}$ to be the full subcategory of $\bold{Top}$ consisting of spaces that are compactly generated and weakly Hausdorff. In other words, $\bold{CGWH}$ consists of the following data: 
\begin{itemize}
\item $\text{Obj}(\bold{CGWH})$ consists of all spaces that are compactly generated and weakly Hausdorff. 
\item For $X,Y\in\text{Obj}(\bold{CGWH})$, the morphisms are $$\Hom_{\bold{CGWH}}(X,Y)=\Hom_{\bold{Top}}(X,Y)$$
\item Association is given by composition of functions. 
\end{itemize}
Define similarly the category of pointed compactly generated spaces $\bold{CGWH}_\ast$. 
\end{defn}

\begin{prp}{}{} A compactly generated space $X$ is weakly Hausdorff if and only if the diagonal subspace $\Delta=\{(x,x)\;|\;x\in X\}$ is closed in $X\times X$. 
\end{prp}

\begin{prp}{}{} Product of CGWH is CGWH
\end{prp}

CGWH is complete and cocomplete

\subsection{The Cartesian Product and the Mapping Space}
\begin{defn}{The Mapping Space}{} Let $X,Y\in\bold{CG}$. Define the mapping space of $X$ and $Y$ by $$\text{Map}(X,Y)=k(\Hom_{\bold{CG}}(X,Y))$$ where $\Hom_{\bold{CG}}(X,Y)$ is equipped with the compact open topology. If $(X,x_0)$ and $(Y,y_0)$ are pointed spaces, define the mapping space to be $$\text{Map}_\ast((X,x_0),(Y,y_0))=k(\Hom_{\bold{CG}}((X,x_0),(Y,y_0)))$$
\end{defn}

By restricting to also weakly Hausdorff spaces, we obtain an adjunction. 

\begin{thm}{}{} Let $X,Y,Z\in\bold{CGWH}$. Then the functors $-\times_{\bold{CGWH}}Y:\bold{CGWH}\to\bold{CGWH}$ and $\text{Map}(Y,-):\bold{CGWH}\to\bold{CGWH}$ are adjoint functors with the adjunction formula $$\Hom_{\bold{CGWH}}(X\times_{\bold{CGWH}}Y,Z)\cong\Hom_{\bold{CGWH}}(X,\text{Map}(Y,Z))$$ Moreover, by giving the Hom set the compact open topology and applying $k$, we obtain an isomorphism $$\text{Map}(X\times_{\bold{CGWH}}Y,Z)\cong\text{Map}(X,\text{Map}(Y,Z))$$
\end{thm}

\subsection{The Smash Product and the Pointed Mapping Space}
Aside from the adjunction between the product space and the mapping space, another major reason one considers compactly generated spaces is that the smash product gives another adjunction. 

\begin{defn}{The Smash Product}{} Let $(X,x_0)$ and $(Y,y_0)$ be pointed topological spaces. Define the smash product of the two pointed spaces to be the pointed space $$X\wedge Y=\frac{X\times Y}{X\vee Y}$$ together with the point $(x_0,y_0)$. 
\end{defn}

\begin{prp}{}{} Let $X,Y,Z$ be compactly generated spaces with a chosen base point. Then the following are true. 
\begin{itemize}
\item $(X\wedge Y)\wedge Z\cong X\wedge(Y\wedge Z)$
\item $X\wedge Y\cong Y\wedge X$
\end{itemize}
\end{prp}

\begin{thm}{}{} The category $\bold{CG}$ of compactly generated spaces is a symmetric monoidal category with operator the smash product $\wedge:\bold{CG}\times\bold{CG}\to\bold{CG}$ and the unit $S^0$. 
\end{thm}

Note that this is not true if we do not restrict the spaces to the category of compactly generated spaces. 

\begin{lmm}{}{} Let $X$ be a pointed space. Then the reduced suspension and the smash product with the circle $$\Sigma X\cong X\wedge S^1$$ are homeomorphic spaces. 
\end{lmm}

\begin{thm}{}{} Let $X,Y,Z$ be compactly generated with a chosen basepoint. Then the functors $-\wedge Y:\mK_\ast\to\mK_\ast$ and $\text{Map}_\ast(Y,-):\mK_\ast\to\mK_\ast$ are adjoint functors with the adjunction formula $$\Hom_{\mK_\ast}(X\wedge Y,Z)\cong\Hom_{\mK_\ast}(X,\text{Map}_\ast(Y,Z))$$ Moreover, by giving the Hom set the compact open topology and applying $k$, we obtain an isomorphism $$\text{Map}_\ast(X\wedge Y,Z)\cong\text{Map}_\ast(X,\text{Map}_\ast(Y,Z))$$
\end{thm}

By choosing $Y=I$ in the adjunction, we recover the usual suspension-loopspace adjunction in $\bold{Top}_\ast$. 

\begin{crl}{}{} Let $X$ be a compactly generated space with a chosen basepoint. Then there is a natural homeomorphism $$\text{Map}_\ast(\Sigma X,Y)\cong\text{Map}_\ast(X,\Omega Y)$$ given by adjunction of the functors $-\wedge S^1:\mK_\ast\to\mK_\ast$ and $\text{Map}_\ast(S^1,-):\mK_\ast\to\mK_\ast$. 
\end{crl}

\subsection{The Mapping Cylinder and the Mapping Path Space}
Equipped with the Cartesian closed structure in $\bold{CG}$ together with a canonical topology on the mapping space $Y^X$, we can now talk about the duality between the mapping cylinder and the mapping path space. 

\begin{defn}{Mapping Cylinder}{} Let $X,Y$ be spaces and let $f:X\to Y$ a map. Define the mapping cylinder of $f$ to be $$M_f=\frac{(X\times I)\amalg Y}{(x,0)\sim f(x)}=(X\times I)\amalg_fY$$ for $f:X\times\{1\}\cong X\to Y$ together with the quotient topology. It is the push forward of $f$ and the inclusion map $i_0:X\cong X\times\{0\}\hookrightarrow X\times I$. 
\end{defn}

\begin{lmm}{}{} Let $X,Y$ be spaces and let $f:X\to Y$ be a map. Then $Y$ is a deformation retract of $M_f$. 
\end{lmm}

\begin{defn}{The Mapping Path Space}{} Let $X,Y$ be spaces and let $f:X\to Y$ be a map. Define the mapping path space of $f$ to be $$P_f=\{(x,\gamma)\in X\times\text{Map}(I,Y)\;|\;\gamma(0)=f(x)\}$$ It is the pull back of $f$ and $\pi_0:\text{Map}(I,Y)\to Y$ given by $\pi_0(\gamma)=\gamma(0)$ in $\bold{CG}$. 
\end{defn}

\subsection{Base Point Independency}
\begin{defn}{Well-Pointed Space}{} Let $(X,x_0)$ be a pointed space. We say that $(X,x_0)$ is well-pointed if $X\times\{0\}\cup\{x_0\}\times I$ is a retract of $X\times I$. In this case we call the base point non-degenerate. 
\end{defn}

\begin{prp}{}{} Let $(X,x_0)$ be a pointed space. Then $(X,x_0)$ is well-pointed if and only if the inclusion $x_0\to X$ is a cofibration. 
\end{prp}

\begin{prp}{}{} Let $(X,x_0)$ be a pointed space. If $(X,x_0)$ is well-pointed, then the reduced and unreduced cone are homotopy equivalent. 
\end{prp}

\begin{prp}{}{} Let $(X,x_0)$ be a pointed space. If $(X,x_0)$ is well-pointed, then the reduced and unreduced suspension are homotopy equivalent. 
\end{prp}

\subsection{Homotopy Commutative and Homotopy Coherence Squares}
\begin{defn}{Homotopy Commutative Squares}{} Let $W,X,Y,Z$ be spaces. Consider the following diagram: \\~\\
\adjustbox{scale=1.0,center}{\begin{tikzcd}
	W & Y \\
	X & Z
	\arrow["f", from=1-1, to=1-2]
	\arrow["g"', from=1-1, to=2-1]
	\arrow["h", from=1-2, to=2-2]
	\arrow["k"', from=2-1, to=2-2]
\end{tikzcd}}\\~\\
of spaces. We say that the square is homotopy commutative if there exists a homotopy $H$ from $k\circ f$ to $h\circ g$. 
\end{defn}

Commutative squares are homotopy commutative. 

\pagebreak
\section{Fibers and Cofibers, Fibrations and Cofibrations}
\subsection{Fibers and Cofibers}
\begin{defn}{Fibers of a Map}{} Let $X,Y$ be spaces. Let $f:X\to Y$ be a map. Define the fiber of $f$ at $y\in Y$ to be $$\text{Fib}_y(f)=f^{-1}(y)$$
\end{defn}
\begin{defn}{Fiber Sequences}{} Let $X,Y,Z$ be pointed spaces. Let $f:X\to Y$ and $g:Y\to Z$ be maps. We say that \\~\\
\adjustbox{scale=1,center}{\begin{tikzcd}
	X & Y & Z
	\arrow["f", from=1-1, to=1-2]
	\arrow["g", from=1-2, to=1-3]
\end{tikzcd}}\\~\\
is a fiber sequence if $f$ induces a homotopy equivalence from $f$ to $g^{-1}(z_0)$ where $z_0$ is the base point of $Z$. 
\end{defn}

\begin{prp}{}{} Let $X,Y$ be pointed spaces. Let $f:X\to Y$ be maps. Then \\~\\
\adjustbox{scale=1,center}{\begin{tikzcd}
	{\text{fib}_y(f)} & X & Y
	\arrow[hook, from=1-1, to=1-2]
	\arrow["f", from=1-2, to=1-3]
\end{tikzcd}}\\~\\
is a homotopy equivalence. 
\end{prp}

\begin{prp}{}{} Let $X,Y,Z$ be pointed spaces. Let $f:X\to Y$ and $g:Y\to Z$ be maps. Let $z_0\in Z$. Suppose that \\~\\
\adjustbox{scale=1,center}{\begin{tikzcd}
	X & Y & Z
	\arrow["f", from=1-1, to=1-2]
	\arrow["g", from=1-2, to=1-3]
\end{tikzcd}}\\~\\
is a fiber sequence. Then \\~\\
\adjustbox{scale=1,center}{\begin{tikzcd}
	{\Omega X} & {\Omega Y} & {\Omega Z}
	\arrow["{\Omega f}", from=1-1, to=1-2]
	\arrow["{\Omega g}", from=1-2, to=1-3]
\end{tikzcd}}\\~\\
is also a fiber sequence. 
\end{prp}

\begin{prp}{}{} Let $X,Y,B$ be spaces. Let $f:X\to B$ and $g:Y\to B$ be maps. If $f$ and $g$ are homotopy equivalent, then for any $b\in B$, the fibers $$\text{Fib}_b(f)\simeq\text{Fib}_b(g)$$ are homotopy equivalent. \tcbline
\begin{proof}
Suppose that $f$ and $g$ are homotopy equivalent via two maps $h:X\to Y$ and $k:Y\to X$. This means that there exists a homotopy $H:X\times I\to X$ such that $H(-,0)=h\circ k$ and $H(-,1)=\text{id}_X$. Similarly, there exists a homotopy $K:Y\times I\to Y$ such that $K(-,0)=k\circ h$ and $K(-,1)=\text{id}_Y$. Consider the map $h|_{f^{-1}(b)}:f^{-1}(b)\to g^{-1}(b)$ and similarly for $k|_{g^{-1}(b)}$. Define two maps $\overline{H}:H|_{f^{-1}(b)\times I}$ and $\overline{K}=K|_{g^{-1}(b)\times I}$. To show that they are homotopies, we just need to show that $\overline{H}\subseteq f^{-1}(b)$ and similarly for $\overline{K}$. Now by definition, each $H(-,t):X\to Y$ is such that $g(H(-,t))=f$. Choose $x\in f^{-1}(b)$. Then $$g(H(x,t))=f(x)=b$$ Hence the entire homotopy $\overline{H}$ stays in the fiber $f^{-1}(b)$. Hence $\overline{H}$ is a well defined homotopy on $f^{-1}(b)$. Similarly for $\overline{K}$. Hence the two fibers are homotopy equivalent. 
\end{proof}
\end{prp}

Unfortunately for most maps $f:X\to Y$, the fibers themselves are not homeomorphic, and not even homotopy equivalent. 

\begin{eg}{}{} The fibers of the projection map $S^1\to\R$ to the $x$-axis are not homotopy equivalent. \tcbline
\begin{proof}
It is clear that the fiber of $S^1\to\R$ is either empty, consist of one point, or of two points. Neither two of the three are homotopy equivalent. 
\end{proof}
\end{eg}

There are two ways to proceed from here. We first try to find a set of maps in which all fibers are homotopy equivalent. This is the content of this section. Otherwise, we try and define a new notion of fiber so that we obtain homotopy equivalence. This is the content of the next section. 

\begin{defn}{Cofibers of a Map}{} Let $X,Y$ be spaces. Let $f:X\to Y$ be a map. Define the cofiber of $f$ to be $$\text{Cofib}(f)=\frac{Y}{f(X)}$$
\end{defn}

\subsection{More on Fibrations and Cofibrations}
\begin{prp}{}{} Let $X,Y$ be spaces. Let $f:X\to Y$ be a map. Let $x_0\in X$. Then the evaluation map $$\text{ev}_{x_0}:\text{Map}(X,Y)\to Y$$ defined by $f\mapsto f(x_0)$ is a fibration. 
\end{prp}

\begin{prp}{}{} Let $X,Y,Z\in\bold{CGWH}$. Let $f:X\to Y$ be a map. Then the following are true. 
\begin{itemize}
\item If $f$ is a fibration, then the induced map $$f_\ast:\text{Map}(Z,X)\to\text{Map}(Z,Y)$$ is a fibration. 
\item If $f$ is a cofibration, then the map $$f\times\text{id}_Z:X\times Z\to Y\times Z$$ is a cofibration. 
\end{itemize}
\end{prp}

\begin{prp}{}{} Let $X,Y,Z\in\bold{CGWH}$. Let $p:X\to Y$ be a map. 
\begin{itemize}
\item If $p$ is a fibration and $f:Z\to Y$ is a map, then the pullback $X\times_YZ\to Z$ of $p$ and $f$ is a fibration \\~\\
\adjustbox{scale=1,center}{\begin{tikzcd}
	{X\times_YZ} & X \\
	Z & Y
	\arrow[from=1-1, to=1-2]
	\arrow["{\text{fibration}}"', dashed, from=1-1, to=2-1]
	\arrow["p", from=1-2, to=2-2]
	\arrow["f"', from=2-1, to=2-2]
\end{tikzcd}}\\~\\
\item If $p$ is a cofibration and $g:X\to Z$ is a map, then the push forward $Z\to Z\coprod_XY$ of $p$ and $g$ is a cofibration \\~\\
\adjustbox{scale=1,center}{\begin{tikzcd}
	X & Z \\
	Y & {Z\coprod_XY}
	\arrow["g", from=1-1, to=1-2]
	\arrow["p"', from=1-1, to=2-1]
	\arrow["{\text{cofibration}}", dashed, from=1-2, to=2-2]
	\arrow[from=2-1, to=2-2]
\end{tikzcd}}\\~\\
\end{itemize}
\end{prp}

\begin{prp}{}{} Let $X,Y,Z\in\bold{CGWH}$. Let $p:X\to Y$ be a map. 
\begin{itemize}
\item If $p$ is a fibration and $f:Z\to Y$ is a (homotopy) weak equivalence, then the pullback $X\times_YZ\to X$ of $p$ and $f$ is a (homotopy) weak equivalence \\~\\
\adjustbox{scale=1,center}{\begin{tikzcd}
	{X\times_YZ} & X \\
	Z & Y
	\arrow["\simeq", dashed, from=1-1, to=1-2]
	\arrow[from=1-1, to=2-1]
	\arrow["p", from=1-2, to=2-2]
	\arrow["{f,\simeq}"', from=2-1, to=2-2]
\end{tikzcd}}\\~\\
\item If $p$ is a cofibration and $g:X\to Z$ is a (homotopy) weak equivalence, then the push forward $Y\to Z\coprod_XY$ of $p$ and $g$ is a (homotopy) weak equivalence \\~\\
\adjustbox{scale=1,center}{\begin{tikzcd}
	X & Z \\
	Y & {Z\coprod_XY}
	\arrow["{g, \simeq}", from=1-1, to=1-2]
	\arrow["p"', from=1-1, to=2-1]
	\arrow[from=1-2, to=2-2]
	\arrow["\simeq"', dashed, from=2-1, to=2-2]
\end{tikzcd}}\\~\\
\end{itemize}
\end{prp}

\begin{prp}{}{} Let $E_1,E_2,B_1,B_2$ be spaces. Let $p_1:E_1\to B_1$ and $p_2:E_2\to B_2$ be fibrations. Let $(f:E_1\to E_2,g:B_1\to B_2)$ be a map from $p_1$ to $p_2$. If $f$ and $g$ are homotopy equivalences, then for any $b_1\in B_1$, the fibers $$\text{Fib}_{b_1}(p_1)\simeq\text{Fib}_{g(b_1)}(p_2)$$ are homotopy equivalent. This is displayed in the following diagram: \\~\\
\adjustbox{scale=1,center}{\begin{tikzcd}
	{\text{Fib}_{b_1}(p_1)} & {\text{Fib}_{g(b_1)}(p_2)} \\
	{E_1} & {E_2} \\
	{B_1} & {B_2}
	\arrow["\simeq", from=1-1, to=1-2]
	\arrow[hook, from=1-1, to=2-1]
	\arrow[hook, from=1-2, to=2-2]
	\arrow["{f, \simeq}", from=2-1, to=2-2]
	\arrow["{p_1}"', from=2-1, to=3-1]
	\arrow["{p_2}", from=2-2, to=3-2]
	\arrow["{g, \simeq}"', from=3-1, to=3-2]
\end{tikzcd}}\\~\\
\end{prp}

\begin{prp}{}{} Let $X_1,X_2,Y_1,Y_2$ be spaces. Let $p_1:X_1\to Y_1$ and $p_2:X_2\to Y_2$ be cofibrations. Let $(f:X_1\to X_2,g:Y_1\to Y_2)$ be a map from $p_1$ to $p_2$. If $f$ and $g$ are homotopy equivalences, then there are homotopy equivalences $$\text{cofib}(p_1)\simeq\text{cofib}(p_2)$$ induced by $g$. 
\end{prp}

\subsection{Fibration and Cofibration Replacements}
\begin{prp}{}{} Let $X,Y\in\bold{CGWH}$ be spaces. Let $f:X\to Y$ be a map. Then the map $$q:P_f\to Y$$ given by $q(x,\gamma)=\gamma(1)$ is a fibration. \tcbline
\begin{proof}
Suppose that we are given a homotopy lifting problem: \\~\\
\adjustbox{scale=1.0,center}{\begin{tikzcd}
	{A\times\{0\}} && P_f \\
	\\
	{A\times I} && Y
	\arrow["H"', from=3-1, to=3-3]
	\arrow["{\widetilde{H}}"{description}, dashed, from=3-1, to=1-3]
	\arrow["q", from=1-3, to=3-3]
	\arrow["\iota"', hook, from=1-1, to=3-1]
	\arrow["g", from=1-1, to=1-3]
\end{tikzcd}}\\~\\
We write $g(a)=(g_1(a),g_2(a))$ for the components of $g$. Now recall that the definition of the mapping path space implies that $f(g_1(a))=g_2(a)(0)$. By commutativity of the diagram and definition of $q$ we also have $g_2(a)(1)=H(a,0)$. Define $\tilde{H}:A\times I\to P_f$ by the fomula $$\tilde{H}(a,t)=(g_1(a),h_2(a,t))$$ where $$h_2(a,t)(s)=\begin{cases}
g_2(a)(1+t)(s) & \text{ if }0\leq s\leq\frac{1}{1+t}\\
H(a,(1+t)s-1) & \text{ if }\frac{1}{1+t}\leq s\leq 1
\end{cases}$$ The definition of $h_2$ makes sense because $g_2(a)(1)=H(a,0)$. By the gluing lemma $h_2(a,t)$ is continuous. $h_2$ is also continuous in $a$ and $t$ because $g_2(a)$ is a path and $g_2$ is continuous and $H$ is continuous in both variables and the composite of continuous functions are continuous. Hence $\tilde{H}$ is continuous. Now $\tilde{H}(-,0)=(g_1(-),g_2(-))=g(-)$. Thus $\tilde{H}(-,0)$ is a lift of $g$. It remains to show that $\tilde{H}$ is a lift of $H$. We have that 
\begin{align*}
q(\tilde{H}(a,t))&=q(g_1(a),h_2(a,t))\\
&=h_2(a,t)(1)\\
&=H(a,t)
\end{align*}
and so we conclude. 
\end{proof}
\end{prp}

We can factorize any continuous map into a fibration and a homotopy equivalence through the mapping path space. Because we are working with the mapping path space here, we need to restrict our attention to compactly generated space. 

\begin{thm}{}{} Let $X,Y\in\bold{CGWH}$. Let $f:X\to Y$ be a map. Then there exists a homotopy equivalence $h:X\to P_f$ such that the following diagram commutes: \\~\\
\adjustbox{scale=1,center}{\begin{tikzcd}
	X && Y \\
	& {P_f}
	\arrow["f", from=1-1, to=1-3]
	\arrow["{\exists h}"', dashed, from=1-1, to=2-2]
	\arrow["q"', from=2-2, to=1-3]
\end{tikzcd}} \\~\\
where $q:P_f\to Y$ is the above defined map. \tcbline
\begin{proof}
Define the map $h:X\to P_f$ by $h(x)=(x,e_{f(x)})$. It is easy to see that $q\circ h=f$. I claim that the projection map $p_X:P_f\to X$ gives the homotopy inverse of $h$. Define a map $H:P_f\times I\to P_f$ by $$H(x,\gamma,t)=(x,\gamma_t)$$ where $\gamma_s$ is the path $s\mapsto\gamma(st)$. It is continuous since the composition of continuous functions are continuous and each component of $H$ is continuous. Also, we have that $h(p_X(x,\gamma))=h(x)=(x,e_{f(x)})$ and $H(x,\gamma,0)=(x,\gamma_0)=(x,e_{f(x)})$ so that $H(-,0)=h\circ p_X$. When $t=1$ we also have $$H(-,1)=(x,\gamma,1)=(x,\gamma_1)=(x,\gamma)=\text{id}_{P_f}(x,\gamma)$$ so that $H$ is a homotopy. 
\end{proof}
\end{thm}

\begin{prp}{}{} Let $X,Y\in\bold{CGWH}$ be spaces. Let $f:X\to Y$ be a map. Let $h:X\to P_f$ be the map that gives a factorization $q\circ h=f$. If $f$ is a fibration, then $h$ is a fiber homotopy equivalence. 
\end{prp}

Cofibrations and fibrations are dual in the following sense. Recall from section 1 that if $X$ and $Y$ are in $\bold{CGWH}$, then there is a bijection $$\Hom_{\bold{CGWH}}(X\times I,Y)\cong\Hom_{\bold{CGWH}}(X,\text{Map}(I,Y))$$ Now under this bijection, we can rewrite the diagram in the homotopy lifting property: \\~\\
\adjustbox{scale=1.0,center}{\begin{tikzcd}
	X & {E^I} \\
	X & {B^I}
	\arrow[from=1-1, to=1-2]
	\arrow["{\text{id}_X}"', from=1-1, to=2-1]
	\arrow["{p_\ast}", from=1-2, to=2-2]
	\arrow["{\exists\tilde{H}}", dashed, from=2-1, to=1-2]
	\arrow["H"', from=2-1, to=2-2]
\end{tikzcd}}\\~\\

\begin{prp}{}{} Let $A,X\in\bold{CGWH}$ be spaces. Let $f:A\to X$ be a map. Then the map $$q:A\to M_f$$ given by $q(a)=[a,0]$ is a cofibration. \tcbline
\begin{proof}
Suppose that we are given a homotopy lifting problem: \\~\\
\adjustbox{scale=1.0,center}{\begin{tikzcd}
	{A\cong A\times\{0\}} & {A\times I} \\
	{X\cong X\times\{0\}} & {X\times I} \\
	&& Y
	\arrow["{\iota_0}", from=1-1, to=1-2]
	\arrow["i"', from=1-1, to=2-1]
	\arrow["{i\times\text{id}_I}", from=1-2, to=2-2]
	\arrow["H", bend left = 30, from=1-2, to=3-3]
	\arrow["{\iota_0}"', from=2-1, to=2-2]
	\arrow["f", bend right = 20, from=2-1, to=3-3]
	\arrow["{\tilde{H}}", dashed, from=2-2, to=3-3]
\end{tikzcd}}\\~\\
\end{proof}
\end{prp}

Dual to the factorization through the mapping path space, we can factorize a map into a homotopy equivalence and a cofibration through the mapping cylinder $$M_f=\frac{(X\times I)\amalg Y}{(x,0)\sim f(x)}=(X\times I)\amalg_fY$$

\begin{thm}{}{} Let $f:A\to X$ be a map. Then the inclusion map $i:A\to M_f$ defined by $i(a)=[a,0]$ is a cofibration. Moreover, there exists a homotopy equivalence $h:M_f\to X$ such that the following diagram commutes: \\~\\
\adjustbox{scale=1,center}{\begin{tikzcd}
	A && X \\
	& {M_f}
	\arrow["f", from=1-1, to=1-3]
	\arrow["i"', from=1-1, to=2-2]
	\arrow["{\exists h}"', dashed, from=2-2, to=1-3]
\end{tikzcd}}
\end{thm}

\subsection{(Co)Fibers of a (Co)Fibration are Homotopic}
The following definition is a supporting notion for our proof that fibers of a fibration are homotopy equivalent. 

\begin{defn}{Induced Map of Fibers}{} Let $p:E\to B$. Let $\gamma:I\to B$ be a path from $b_1$ to $b_2$. Define the induced map of fibers of $\gamma$ as follows: The map $H:E_{b_1}\times I\to B$ defined by $H(x,t)=\gamma(t)$ is a homotopy. Using the HLP of $p$, we obtain a lift: \\~\\
\adjustbox{scale=1,center}{\begin{tikzcd}
	{E_{b_1}\times\{0\}} & E \\
	{E_{b_1}\times I} & B
	\arrow["{\widetilde{H(-,0)}}", hook, from=1-1, to=1-2]
	\arrow[hook, from=1-1, to=2-1]
	\arrow["p", from=1-2, to=2-2]
	\arrow["{\widetilde{H}}"{description}, dashed, from=2-1, to=1-2]
	\arrow["H"', from=2-1, to=2-2]
\end{tikzcd}} \\~\\
Since $p\circ\widetilde{H}(x,t)=\gamma(t)$, we have that $\widetilde{H}(x,1)\in E_{b_2}$. The induced map of fibers is then the map $$L_\gamma:E_{b_1}\to E_{b_2}$$ defined by $L_\gamma=\widetilde{H(-,1)}$
\end{defn}

\begin{lmm}{}{} Let $p:E\to B$ be a fibration. Let $\gamma:I\to B$ be a path from $b_1$ to $b_2$. Then the following are true regarding $L_\gamma$. 
\begin{itemize}
\item If $\gamma\simeq\gamma'$ relative to boundary, then $L_\gamma\simeq L_{\gamma'}$.
\item If $\gamma:I\to B$ and $\gamma':I\to B$ are two composable paths, there is a homotopy equivalence $L_{\gamma\cdot\gamma'}\simeq L_{\gamma'}\circ L_\gamma$
\end{itemize} \tcbline
\begin{proof}
\begin{itemize}
\item Let $F:I\times I\to B$ be a homotopy equivalence from $\gamma$ to $\gamma'$. Now consider the map $G:E_{b_1}\times I\times I\to B$ defined by $G(x,s,t)=F(s,t)$. Notice that $G(x,s,0)=F(s,0)=\gamma(s)$ and $G(x,s,1)=F(s,1)=\gamma'(s)$. Thus, we proceed as above by lifting $G(x,s,0)$ and $G(x,s,1)$ to obtain respectively $\widetilde{G(x,s,0)}$ and $\widetilde{G(x,s,1)}$ for which $\widetilde{G(x,1,0)}=L_\gamma$ and $\widetilde{G(x,1,1)}=L_{\gamma'}$. Now define $K:E_{b_1}\times I\times\partial I\to E$ by $$K(x,s,t)=\begin{cases}
\widetilde{G(x,s,1)} & \text{ if } t=0\\
G(x,s,1) & \text{ if } t=1
\end{cases}$$ We now obtain a homotopy called $\widetilde{G}:E_{b_1}\times I\times I\to E$ by the homotopy lifting property: \\~\\
\adjustbox{scale=1,center}{\begin{tikzcd}
	{X\times I\times\partial I} & E \\
	{X\times I\times I} & B
	\arrow["K", from=1-1, to=1-2]
	\arrow[hook, from=1-1, to=2-1]
	\arrow["p", from=1-2, to=2-2]
	\arrow["\widetilde{G}"{description}, dashed, from=2-1, to=1-2]
	\arrow["G"', from=2-1, to=2-2]
\end{tikzcd}} \\~\\
Now $\tilde{G}(-,1,-):E_b\times I\to E$ is then a homotopy equivalence from $\widetilde{G}(x,1,0)=L_\gamma$ to $\widetilde{G}(x,1,1)=L_{\gamma'}$. 
\item We can repeat the above construction for $\gamma$ and $\gamma'$ to obtain homotopies $G:E_{b_1}\times I\to E$ and $G':E_{b_1}\times I\to E$ such that when $t=1$ we recover $\tilde{\gamma}$, $\tilde{\gamma'}$ and $\tilde{\gamma\cdot\gamma'}$ respectively. Now the composition of $G$ and $G'$ by traversing along $t\in I$ with twice the speed gives precisely a lift of $\gamma\cdot\gamma'$ (one can check the boundary conditions). Thus $L_{\gamma\cdot\gamma'}$ obtained in this manner coincides up to homotopy equivalence to $L_{\gamma'}\circ L_\gamma$ by invoking part a). 
\end{itemize}
\end{proof}
\end{lmm}

\begin{thm}{}{} Let $p:E\to B$ be a fibration. Let $b_1$ and $b_2$ lie in the same path component of $B$. Then there is a homotopy equivalence $$E_{b_1}\simeq E_{b_2}$$ given by the lift of any path $\gamma:I\to B$ from $b_1$ to $b_2$. \tcbline
\begin{proof}
Let $\gamma:I\to B$ be a path from $b_1$ to $b_2$. From the above, it follows that $L_{\overline{\gamma}}\circ L_\gamma\simeq\text{id}_{E_b}$ for any loop $\gamma:I\to B$ with basepoint $b$. We conclude that $L_\gamma$ is a homotopy equivalence and so the fibers of $p:E\to B$ are homotopy equivalent. 
\end{proof}
\end{thm}

\subsection{Algebra of Fibrations and Cofibrations}
\begin{prp}{}{} Let $X,Y$ be spaces. Let $f:X\to Y$ be a cofibration. Then for all spaces $Z$, the induced map $$f^\ast:\text{Map}(Y,Z)\to\text{Map}(X,Z)$$ is a fibration. 
\end{prp}

\pagebreak
\section{Homotopy Fibers and Homotopy Cofibers}
\subsection{Basic Definitions}
\begin{defn}{Homotopy Fibers and Cofibers}{} Let $f:X\to Y$ be a map. Define the homotopy fiber of $f$ at $y\in Y$ to be $$\text{hofiber}_y(f)=\{(x,\phi)\in X\times\text{Map}(I,Y)\;|\;f(x)=\phi(0), \phi(1)=y\}=\text{Fib}_y(P_f\to Y)$$
\end{defn}

TBA: hofiber = pullback $P_f\to Y\leftarrow\ast$ (time $t=1$ and $\ast\mapsto y$). \\~\\

Since the map $P_f\to Y$ is a fibration, the fibers of $P_f\to Y$, and hence the homotopy fibers of $f$ are all homotopy equivalent. 

\begin{prp}{}{} Let $p:E\to B$ be a fibration. Then the there is a homotopy equivalence $$\text{Fib}_b(f)\simeq\text{Hofib}_b(f)$$ for each $b\in B$, given by the inclusion map $x\mapsto(x,e_x)$. \tcbline
\begin{proof}
Consider the following diagram \\~\\
\adjustbox{scale=1,center}{\begin{tikzcd}
	{\text{Fib}_b(f)} & {\text{Hofib}_b(f)} \\
	X & {P_f} \\
	Y & Y
	\arrow["\simeq", from=1-1, to=1-2]
	\arrow[hook, from=1-1, to=2-1]
	\arrow[hook, from=1-2, to=2-2]
	\arrow["{h, \simeq}", from=2-1, to=2-2]
	\arrow["f"', from=2-1, to=3-1]
	\arrow["q", from=2-2, to=3-2]
	\arrow["{\text{id}_Y}"', from=3-1, to=3-2]
\end{tikzcd}} \\~\\
and apply 2.4.6 to conclude. 
\end{proof}
\end{prp}

\begin{prp}{}{} Let $X,Y$ be spaces. Let $f,g:X\to Y$ be maps. If $f,g$ are homotopic, then for all $y\in Y$, there is a homotopy equivalence $$\text{hofib}_y(f)\simeq\text{hofib}_y(g)$$ induced by the homotopy from $f$ to $g$. 
\end{prp}

\begin{eg}{}{} Let $X,Y$ be spaces. Let $f:X\to Y$ be a map. 
\begin{itemize}
\item The homotopy fiber of $\{y\}\hookrightarrow Y$ is given by $\text{hofib}_y(\{y\}\hookrightarrow Y)\cong\Omega Y$
\item If $f$ is null-homotopic, then the homotopy fiber of $f$ is given by $\text{hofib}_y(f)\cong X\times\Omega Y$
\end{itemize}
\end{eg}

\begin{defn}{Homotopy Fibers and Cofibers}{} Let $f:X\to Y$ be a map. Define the homotopy cofiber of $f$ to be $$\text{hocofiber}(f)=\frac{(X\times I)\amalg Y}{(x,1)\sim f(x),(x,0)\sim(x',0)}=\text{Cofib}(X\to M_f)=C_f$$
\end{defn}

\begin{prp}{}{} Let $f:X\to Y$ be a cofibration. Then there is a homotopy equivalence $$\text{hocofib}(f)\simeq\text{cofib}(f)$$ given by $(x,t)\mapsto(x,1)$. 
\end{prp}

\begin{prp}{}{} Let $X,Y$ be spaces. Let $f,g:X\to Y$ be maps. If $f,g$ are homotopic, then for all $y\in Y$, there is a homotopy equivalence $$\text{hocofib}(f)\simeq\text{hocofib}(g)$$ induced by the homotopy from $f$ to $g$. 
\end{prp}

\subsection{Fiber and Cofiber Sequences}
\begin{defn}{Exact Sequence of Sets}{} Let $A,B,C$ be pointed sets. Let $f:A\to B$ and $g:B\to C$. We say that the diagram \\~\\
\adjustbox{scale=1.0,center}{\begin{tikzcd}
	A & B & C
	\arrow["f", from=1-1, to=1-2]
	\arrow["g", from=1-2, to=1-3]
\end{tikzcd}}\\~\\
is exact if $\im(f)=g^{-1}(c_0)$ where $c_0$ is the base point of $C$. 
\end{defn}

We now write a fibration as a sequence $F\to E\to B$ for $F$ the fiber of the fibration $p:E\to B$. This compact notation allows the following theorem to be formulated nicely. 

\begin{thm}{}{} Let $X,Y$ be pointed spaces. Let $f:X\to Y$ be a map. Let $F_f=\text{hofiber}_y(f)$ denote the homotopy fiber. Define the following maps. 
\begin{itemize}
\item The map $-\Omega f:\Omega X\to\Omega Y$ given by $\gamma\mapsto\overline{f\circ\gamma}$ (reversing the traversal)
\item The map $k:\Omega Y\hookrightarrow F_f$ given by inclusion. 
\end{itemize}
Consider the following diagram of spaces: \\~\\
\adjustbox{scale=1.0,center}{\begin{tikzcd}
	\cdots & {\Omega^2 X} & {\Omega^2Y} & {\Omega F_f} & {\Omega X} & {\Omega Y} & {F_f} & X & Y
	\arrow[from=1-1, to=1-2]
	\arrow["{\Omega^2 f}", from=1-2, to=1-3]
	\arrow["{-\Omega k}", from=1-3, to=1-4]
	\arrow["{-\Omega\text{incl}}", from=1-4, to=1-5]
	\arrow["{-\Omega f}", from=1-5, to=1-6]
	\arrow["k", from=1-6, to=1-7]
	\arrow["\text{incl}", from=1-7, to=1-8]
	\arrow["f", from=1-8, to=1-9]
\end{tikzcd}}\\~\\
Then up to homotopy equivalence, every two consecutive maps form a fiber sequence. 
\end{thm}

\begin{thm}{}{} Let $X,Y$ be pointed spaces. Let $f:X\to Y$ be a map. Let $F_f=\text{hofiber}_y(f)$ denote the homotopy fiber. Then for any space $Z$, \\~\\
\adjustbox{scale=1.0,center}{\begin{tikzcd}
	\cdots & {[Z,\Omega^2X]} & {[Z,\Omega^2Y]} & {[Z,\Omega F_f]} & {[Z,\Omega X]} & {[Z,\Omega Y]}
	\arrow[from=1-1, to=1-2]
	\arrow["{\Omega^2f\circ -}", from=1-2, to=1-3]
	\arrow["{-\Omega k\circ -}", from=1-3, to=1-4]
	\arrow["{-\Omega\text{incl}\circ -}", from=1-4, to=1-5]
	\arrow["{-\Omega f\circ -}", from=1-5, to=1-6]
\end{tikzcd}}\\~\\
is an exact sequence of abelian groups. Moreover, the diagram of sets \\~\\
\adjustbox{scale=1.0,center}{\begin{tikzcd}
	{[Z,\Omega Y]} & {[Z,F_f]} & {[Z,X]} & {[Z,Y]}
	\arrow["{k\circ -}", from=1-1, to=1-2]
	\arrow["{\text{incl}\circ -}", from=1-2, to=1-3]
	\arrow["{f\circ -}", from=1-3, to=1-4]
\end{tikzcd}}\\~\\
is an exact sequence of sets. 
\end{thm}

There is then the dual notion of loop spaces and the corresponding sequence. Write a cofibration $f:A\to X$ with homotopy cofiber $B$ as $B\to A\to X$. 

\begin{thm}{}{} Let $X,Y$ be pointed spaces. Let $f:X\to Y$ be a map. Define the following maps. 
\begin{itemize}
\item The projection map $\pi:C_f\to\frac{C_f}{Y}\cong\Sigma X$. 
\item The map $-\Sigma f:\Sigma X\to\Sigma Y$ given by $x\wedge t\mapsto f(x)\wedge(1-t)$. 
\end{itemize}
Consider the following diagram of spaces: \\~\\
\adjustbox{scale=1.0,center}{\begin{tikzcd}
	X & Y & {C_f} & {\Sigma X} & {\Sigma Y} & {\Sigma C_f} & {\Sigma^2X} & {\Sigma^2Y} & \cdots
	\arrow["f", from=1-1, to=1-2]
	\arrow["\text{incl}", from=1-2, to=1-3]
	\arrow["\pi", from=1-3, to=1-4]
	\arrow["{-\Sigma f}", from=1-4, to=1-5]
	\arrow["{-\Sigma\text{incl}}", from=1-5, to=1-6]
	\arrow["{-\Sigma\pi}", from=1-6, to=1-7]
	\arrow["{\Sigma^2 f}", from=1-7, to=1-8]
	\arrow[from=1-8, to=1-9]
\end{tikzcd}}\\~\\
Then up to homotopy equivalence, every two consecutive maps form a cofiber sequence. 
\end{thm}

\begin{thm}{}{} Let $X,Y$ be pointed spaces. Let $f:X\to Y$ be a map. Then for any space $Z$, \\~\\
\adjustbox{scale=1.0,center}{\begin{tikzcd}
	\cdots & {[\Sigma^2Y,Z]} & {[\Sigma^2X,Z]} & {[\Sigma C_f,Z]} & {[\Sigma Y,Z]} & {[\Sigma X,Z]}
	\arrow[from=1-1, to=1-2]
	\arrow["{-\circ\Sigma^2 f}", from=1-2, to=1-3]
	\arrow["{-\circ(-\Sigma\pi)}", from=1-3, to=1-4]
	\arrow["{-\circ(-\Sigma\text{incl})}", from=1-4, to=1-5]
	\arrow["{-\circ(-\Sigma f)}", from=1-5, to=1-6]
\end{tikzcd}}\\~\\
is an exact sequence of abelian groups. Moreover, the diagram of sets \\~\\
\adjustbox{scale=1.0,center}{\begin{tikzcd}
	{[\Sigma X,Z]} & {[C_f,Z]} & {[Y,Z]} & {[X,Z]}
	\arrow["{-\circ\pi}", from=1-1, to=1-2]
	\arrow["{-\circ\text{incl}}", from=1-2, to=1-3]
	\arrow["{-\circ f}", from=1-3, to=1-4]
\end{tikzcd}}\\~\\
is an exact sequence of sets. 
\end{thm}

\begin{thm}{Homotopy Long Exact Sequence in Fibration}{} Let $p:E\to B$ be a fibration over a path connected space $B$ with fiber $F$. Let $\iota:F\hookrightarrow E$ be the inclusion of the fiber. Then there is a long exact sequence in homotopy groups: \\~\\
\adjustbox{scale=0.75,center}{\begin{tikzcd}
	\cdots & {\pi_{n+1}(B,b_0)} & {\pi_n(F,e_0)} & {\pi_n(E,e_0)} & {\pi_n(B,b_0)} & {\pi_{n-1}(F,e_0)} & \cdots & {\pi_1(E,e_0)} & {\pi_1(B,b_0)}
	\arrow[from=1-1, to=1-2]
	\arrow["\partial", from=1-2, to=1-3]
	\arrow["{\iota_\ast}", from=1-3, to=1-4]
	\arrow["{p_\ast}", from=1-4, to=1-5]
	\arrow["\partial", from=1-5, to=1-6]
	\arrow[from=1-6, to=1-7]
	\arrow[from=1-7, to=1-8]
	\arrow["{p_\ast}", from=1-8, to=1-9]
\end{tikzcd}}\\~\\
for $e_0\in E$ and $b_0=p(e_0)$. Moreover, $p_\ast$ is an isomorphism. 
\end{thm}

\begin{thm}{Homology Long Exact Sequence in Cofibration}{} Let $p:X\to Y$ be a cofibration with cofiber $C=\frac{Y}{p(X)}$. Let $\text{proj}:Y\to C$ be the projection map. Then there is a long exact sequence in homology groups: \\~\\
\adjustbox{scale=0.85,center}{\begin{tikzcd}
	\cdots & {\widetilde{H}_{n+1}(C)} & {\widetilde{H}_n(X)} & {\widetilde{H}_n(Y)} & {\widetilde{H}_n(C)} & {\widetilde{H}_{n-1}(X} & \cdots & {\widetilde{H}_0(Y)} & {\widetilde{H}_0(B,b_0)}
	\arrow[from=1-1, to=1-2]
	\arrow["\partial", from=1-2, to=1-3]
	\arrow["{f_\ast}", from=1-3, to=1-4]
	\arrow["{\text{proj}_\ast}", from=1-4, to=1-5]
	\arrow["\partial", from=1-5, to=1-6]
	\arrow[from=1-6, to=1-7]
	\arrow[from=1-7, to=1-8]
	\arrow["{\text{proj}_\ast}", from=1-8, to=1-9]
\end{tikzcd}}\\~\\
\end{thm}

\subsection{n-Connected Maps}
\begin{defn}{n-Connected Maps}{} Let $X,Y$ be pointed spaces. Let $f:X\to Y$ be a map. We say that $f$ is $n$-connected if the induced map $$\pi_k(f):\pi_k(X)\to\pi_k(Y)$$ is an isomorphism for $0\leq k<n$ and a surjection for $k=n$. 
\end{defn}

We can rephrase some of the corner stone theorems of homotopy theory using $n$-connected maps. 
\begin{itemize}
\item The homotopy excision theorem can be rephrased into the following. For $X$ a CW-complex and $A,B$ sub complexes of $X$ such that $X=A\cup B$ and $A\cap B\neq\emptyset$. If $(A,A\cap B)$ is $m$-connected and $(B,A\cap B)$ is $n$-connected for $m,n\geq 0$, then the inclusion $$\iota:(A,A\cap B)\to(X,B)$$ is $(m+n)$-connected. 
\item The Freudenthal suspension theorem says that if $X$ is an $n$-connected CW complex, then the map $$\Omega\Sigma:X\to\Omega(\Sigma(X))$$ is a $(2n+1)$-connected map. 
\end{itemize}

\begin{prp}{}{} Let $X,Y$ be pointed spaces. Let $f:X\to Y$ be a map. Then the following are equivalent. 
\begin{itemize}
\item $f$ is $k$-connected. 
\item $\text{Hofib}_y(f)$ is $(k-1)$-connected for all $y\in Y$. 
\item $(M_f,X)$ is a $k$-connected space. 
\end{itemize}
\end{prp}

\begin{lmm}{}{} Let $(X,x_0)$ be a pointed space. Then $(X,x_0)$ is $n$-connected if and only if the inclusion map $\iota:\{x_0\}\hookrightarrow X$ is $n$-connected. 
\end{lmm}

\begin{prp}{}{} Let $X$ be a space. Let $n\in\N$. Then the following are equivalent. 
\begin{itemize}
\item $X$ is $n$-connected. 
\item $\pi_k(X)=0$ for all $0\leq k\leq n$. 
\item For all $-1\leq k\leq n$, every map $S^k\to X$ extends to a map $D^{k+1}\to X$. 
\item $(CX,X)$ is $(n+1)$-connected. 
\end{itemize}
\end{prp}

\begin{prp}{}{} Let $(X,A)$ be a space. Let $n\in\N$. Then the following are equivalent. 
\begin{itemize}
\item $(X,A)$ is $n$-connected
\item For all $0<k\leq n$, $\pi_k(X,A)=0$ and $\pi_0(A)\to\pi_0(X)$ is surjective. 
\item $\iota:A\hookrightarrow X$ is $n$-connected. 
\end{itemize}
\end{prp}

\begin{prp}{}{} Let $X,Y$ be spaces. Let $f:X\to Y$ be a map. Let $n\in\N$. If $X$ is not empty, then the following are equivalent. 
\begin{itemize}
\item $f$ is $n$-connected
\item $\text{hofiber}_y(f)$ is $n$-connected for all $y\in Y$. 
\item For all $0<k\leq n$, $\pi_k(f)$ is an isomorphism and $\pi_n(f)$ is surjective. 
\end{itemize}
\end{prp}

\begin{prp}{}{} Let $X,Y$ be spaces. Let $f:X\to Y$ be a map. If $f$ is $k$-connected, then $\text{hocofib}(f)$ is $k$-connected. 
\end{prp}

\begin{prp}{}{} Let $X,Y$ be spaces. Let $f:X\to Y$ be a map. Let $X$ be simply connected. Then $\text{hocofib}(f)$ is $k$-connected if and only if $f$ is $k$-connected. 
\end{prp}

\pagebreak
\section{Homotopy Pullbacks and Pushouts}
Homotopy pullbacks and pushouts are a special case of homotopy limits and colimits. It would be fruitful for us to first consider this case also because of how it is related to maps of spaces and (co)fibrations. 

\subsection{The Standard Homotopy Pullback and Pushout}
Consider the following diagram: \\~\\
\adjustbox{scale=1.0,center}{\begin{tikzcd}
	X & Z & Y \\
	{X'} & {Z'} & {Y'}
	\arrow["f", from=1-1, to=1-2]
	\arrow["{e_X}"', from=1-1, to=2-1]
	\arrow["{e_Z}"', from=1-2, to=2-2]
	\arrow["g"', from=1-3, to=1-2]
	\arrow["{e_Y}", from=1-3, to=2-3]
	\arrow["{f'}"', from=2-1, to=2-2]
	\arrow["{g'}", from=2-3, to=2-2]
\end{tikzcd}}\\~\\
We can form pullbacks of the upper and lower horizontals and obtain an induced map. However, when $e_X,e_Y,e_Z$ are homotopy equivalences, the induced map is not a homotopy equivalence. We remedy this by introducing a homotopic notion of pullbacks. 

\begin{defn}{The Standard Homotopy Pullback}{} Let $X,Y,Z\in\bold{CGWH}$ be spaces. Let $f:X\to Z$ and $g:Y\to Z$ be maps. Define the standard homotopy pullback of $f$ and $g$ to be the subspace $$\text{holim}(X\overset{f}{\rightarrow}Z\overset{g}{\leftarrow}Y)=\{(x,\alpha,y)\in X\times\text{Map}(I,Z)\times Y\;|\;\alpha(0)=f(x),\alpha(1)=g(y)\}$$
\end{defn}

The idea is that normally in pullbacks, we require that under $f$ and $g$ the elements of the pullback must arrive at the same point in $Z$. But here we relax the requirement by simply allowing elements of the homotopy pullback to arrive at the same path component of $Z$ (so up to the existence of an homotopy of the two points in $Z$). 

\begin{defn}{The Canonical Map of Homotopy Pullbacks}{} Let $X,Y,Z\in\bold{CGWH}$ be spaces. Let $f:X\to Y$ and $g:Y\to Z$ be maps. Define the canonical map from the pullback to the homotopy pullback $$c:\lim(X\overset{f}{\rightarrow}Z\overset{g}{\leftarrow}Y)\to\text{holim}(X\overset{f}{\rightarrow}Z\overset{g}{\leftarrow}Y)$$ to be given by $(x,y)\mapsto(x,e_{f(x)=g(y)},y)$ where $e$ refers to the constant loop at $f(x)=g(y)$. 
\end{defn}

Recall that we motivated the definition of a homotopy pullback from the fact that pullbacks does not work well with homotopy. We can now show that homotopy pullbacks remedy the issue. 

\begin{thm}{The Matching Lemma}{} Suppose that we have a commutative diagram of spaces \\~\\
\adjustbox{scale=1.0,center}{\begin{tikzcd}
	X & Z & Y \\
	{X'} & {Z'} & {Y'}
	\arrow["f", from=1-1, to=1-2]
	\arrow["{e_X}"', from=1-1, to=2-1]
	\arrow["{e_Z}"', from=1-2, to=2-2]
	\arrow["g"', from=1-3, to=1-2]
	\arrow["{e_Y}", from=1-3, to=2-3]
	\arrow["{f'}"', from=2-1, to=2-2]
	\arrow["{g'}", from=2-3, to=2-2]
\end{tikzcd}}\\~\\
in $\bold{CGWH}$. Define the map $$\phi_{X,Z,Y}^{X',Z',Y'}:\text{holim}(X\overset{f}{\rightarrow}Z\overset{g}{\leftarrow}Y)\to\text{holim}(X'\overset{f'}{\rightarrow}Z'\overset{g'}{\leftarrow}Y')$$ by the formula $(x,\gamma,y)\mapsto(e_X(x),e_Z\circ\gamma,e_Y(y))$. Then the following are true. 
\begin{itemize}
\item If each $e_X,e_Y,e_Z$ are homotopy equivalences, then $\phi$ is a homotopy equivalence. 
\item If each $e_X,e_Y,e_Z$ are weak equivalences, then $\phi$ is a weak equivalence. 
\end{itemize} \tcbline
\begin{proof}
We first prove the case for homotopy equivalence. Consider the following commutative diagram: \\~\\
\adjustbox{scale=1.0,center}{\begin{tikzcd}
	X & Z & Y \\
	X & {Z'} & Y \\
	{X'} & {Z'} & {Y'}
	\arrow["f", from=1-1, to=1-2]
	\arrow["{\text{id}_X}"', from=1-1, to=2-1]
	\arrow["{e_Z}", from=1-2, to=2-2]
	\arrow["g"', from=1-3, to=1-2]
	\arrow["{\text{id}_Y}", from=1-3, to=2-3]
	\arrow["{e_Z\circ f}", from=2-1, to=2-2]
	\arrow["{e_X}"', from=2-1, to=3-1]
	\arrow["{\text{id}_{Z'}}", from=2-2, to=3-2]
	\arrow["{e_Z\circ g}"', from=2-3, to=2-2]
	\arrow["{e_Y}", from=2-3, to=3-3]
	\arrow["{f'}"', from=3-1, to=3-2]
	\arrow["{g'}", from=3-3, to=3-2]
\end{tikzcd}}\\~\\
We prove that the homotopy pullback of the first row is homotopy equivalent to that of the second, and we prove that the homotopy pullback of the second row is homotopy equivalent to that of the third. \\~\\

Since $e_Z$ is a homotopy equivalence, we can find a homotopy inverse $k$ for $e_Z$ and a homotopy $H:Z\times I\to Z$ such that $H(-,0)=\text{id}_Z$ and $H(-,1)=k\circ e_Z$. Define a map $$\rho:\text{holim}(X\overset{f}{\rightarrow}Z'\overset{g}{\leftarrow}Y)\to\text{holim}(X\overset{e_Z\circ f}{\rightarrow}Z\overset{e_Z\circ g}{\leftarrow}Y)$$ by the formula $$(x,\gamma',y)\mapsto(x,H(f(x),-)\ast k(\gamma'(-))\ast\overline{H(g(y),-)}:I\to Z,y)$$ where $\ast$ denotes concatenation of paths. The path concatenation is well defined because we have that $H(f(x),1)=(k\circ e_Z\circ f)(x)=(k\circ\gamma')(0)$ and $k(\gamma'(1))=k(e_Z(g(y)))=H(g(y),1)$. This is well defined on the homotopy pullback because we have that 
\begin{itemize}
\item $H(f(x),-)\ast k(\gamma'(-))\ast\overline{H(g(y),-)}(0)=H(f(x),0)=\text{id}_Z(f(x))=f(x)$
\item $H(f(x),-)\ast k(\gamma'(-))\ast\overline{H(g(y),-)}(1)=H(g(y),0)=\text{id}_Z(g(y))=g(y)$
\end{itemize}
I claim that this map is the homotopy inverse to the map $\phi=\phi_{X,Y,Z}^{X,Y,Z'}$. We have that 
\begin{align*}
\rho(\phi(x,\gamma,y))&=\rho(x,e_Z\circ\gamma,y)\\
&=(x,H(f(x),-)\ast k(e_Z(\gamma(-))\ast\overline{H(g(y),-)},y)
\end{align*}
Now I claim that the middle path is homotopic to $\gamma$. For the first component of the concatenation, the path $H(f(x),t):I\to Z$ can be contracted to $H(f(x),0)=f(x)=\gamma(0)$ so you can homotope the traversal along $H(f(x),-)$ to the single point $f(x)=\gamma(0)$. For the third component of the concatenation, this is similar so we can homotope the traversal of $\overline{H(g(y),-)}$ to the single point $g(y)=\gamma(1)$. The middle part of the path is homotopic to $\gamma$ because $k\circ e_Z$ is homotopic to $\text{id}_Z$. Thus we conclude. 
\end{proof}
\end{thm}

Dually, we define the notion of standard homotopy pushouts. 

\begin{defn}{The Standard Homotopy Pushout}{} Let $X,Y,Z\in\bold{CGWH}$ be spaces. Let $f:Z\to X$ and $g:Z\to Y$ be maps. Define the standard homotopy pushout of the diagram to be the quotient space $$\hocolim(X\overset{f}{\leftarrow}Z\overset{g}{\rightarrow}Y)=\frac{X\amalg(Z\times I)\amalg Y}{\sim}$$ where $\sim$ is the equivalence relation generated by $f(z)\sim (z,0)$ and $g(z)\sim(z,1)$ for $z\in Z$. 
\end{defn}

TBA: Mimic the univ property up to homotopy

\begin{defn}{The Canonical Map of Homotopy Pushouts}{} Let $X,Y,Z\in\bold{CGWH}$ be spaces. Let $f:Z\to X$ and $g:Z\to Y$ be maps. Define the canonical map of the homotopy pushout of the diagram to be the map $$s:\hocolim(X\overset{f}{\leftarrow}Z\overset{g}{\rightarrow}Y)\to\colim(X\overset{f}{\leftarrow}Z\overset{g}{\rightarrow}Y)$$ given by the formula $$u\mapsto\begin{cases}
u & \text{ if }u\in X\\
f(z)=g(z) & \text{ if }u=(z,t)\in Z\times I\\
u & \text{ if }u\in Y
\end{cases}$$
\end{defn}

\begin{thm}{The Gluing Lemma}{} Suppose that we have a commutative diagram of spaces \\~\\
\adjustbox{scale=1.0,center}{\begin{tikzcd}
	X & Z & Y \\
	{X'} & {Z'} & {Y'}
	\arrow["f"', from=1-2, to=1-1]
	\arrow["{e_X}"', from=1-1, to=2-1]
	\arrow["{e_Z}"', from=1-2, to=2-2]
	\arrow["g", from=1-2, to=1-3]
	\arrow["{e_Y}", from=1-3, to=2-3]
	\arrow["{f'}", from=2-2, to=2-1]
	\arrow["{g'}"', from=2-2, to=2-3]
\end{tikzcd}}\\~\\
in $\bold{CGWH}$. If each $e_X,e_Y,e_Z$ are (homotopy) weak equivalences, then the induced map $$\hocolim(X\overset{f}{\leftarrow}Z\overset{g}{\rightarrow}Y)\to\hocolim(X'\overset{f'}{\leftarrow}Z'\overset{g'}{\rightarrow}Y')$$ defined by the formula $$u\mapsto\begin{cases}
e_X(u) & \text{ if }u\in X\\
(e_Z(v),t) & \text{ if }u=(v,t)\in Z\times I\\
e_Y(u) & \text{ if }u\in Y
\end{cases}$$ is a (homotopy) weak equivalence. 
\end{thm}

\subsection{Homotopy Pullbacks and Pushout Squares}
\begin{defn}{Homotopy Commutative Squares}{} Let $W,X,Y,Z\in\bold{CGWH}$ be spaces such that there is a (not necessarily commutative) diagram \\~\\
\adjustbox{scale=1.0,center}{\begin{tikzcd}
	W & Y \\
	X & Z
	\arrow[from=1-1, to=1-2]
	\arrow[from=1-1, to=2-1]
	\arrow[from=1-2, to=2-2]
	\arrow[from=2-1, to=2-2]
\end{tikzcd}}\\~\\
We say that the diagram is homotopy commutative if there exists a homotopy $H$ from $W\to X\to Z$ to $W\to Y\to Z$. 
\end{defn}

Notice that hidden from the square is the data of a homotopy. This is different from the commutative diagrams that we are used to. 

\begin{defn}{Homotopy Pullback Squares}{} Let $W,X,Y,Z\in\bold{CGWH}$ be spaces such that there is a (not necessarily commutative) diagram \\~\\
\adjustbox{scale=1.0,center}{\begin{tikzcd}
	W & Y \\
	X & Z
	\arrow[from=1-1, to=1-2]
	\arrow[from=1-1, to=2-1]
	\arrow[from=1-2, to=2-2]
	\arrow[from=2-1, to=2-2]
\end{tikzcd}}\\~\\
\begin{itemize}
\item We say that the diagram is a homotopy pullback if the map $$\alpha:W\to\lim(X\overset{f}{\rightarrow}Z\overset{g}{\leftarrow}Y)\overset{c}{\longrightarrow}\text{holim}(X\overset{f}{\rightarrow}Z\overset{g}{\leftarrow}Y)$$ is a weak equivalence. 
\item We say that the diagram is $k$-cartesian if $\alpha$ is $k$-connected. 
\end{itemize}
\end{defn}

\begin{defn}{Homotopy Pushout Squares}{} Let $W,X,Y,Z\in\bold{CGWH}$ be spaces such that there is a (not necessarily commutative) diagram \\~\\
\adjustbox{scale=1.0,center}{\begin{tikzcd}
	W & Y \\
	X & Z
	\arrow[from=1-1, to=1-2]
	\arrow[from=1-1, to=2-1]
	\arrow[from=1-2, to=2-2]
	\arrow[from=2-1, to=2-2]
\end{tikzcd}}\\~\\
\begin{itemize}
\item We say that the square is a homotopy pushout square if the map $$\beta:\hocolim(X\overset{f}{\leftarrow}W\overset{g}{\rightarrow}Y)\overset{s}{\longrightarrow}\colim(X\overset{f}{\leftarrow}W\overset{g}{\rightarrow}Y)\to Z$$ is a weak equivalence. 
\item We say that the diagram is $k$-cocartesian if $\beta$ is $k$-connected. 
\end{itemize}
\end{defn}

\begin{prp}{}{} Consider the following (not necessarily commutative) square \\~\\
\adjustbox{scale=1.0,center}{\begin{tikzcd}
	{X_1} & {X_2} & {X_3} \\
	{Y_1} & {Y_2} & {Y_3}
	\arrow[from=1-1, to=1-2]
	\arrow[from=1-1, to=2-1]
	\arrow[from=1-2, to=1-3]
	\arrow[from=1-2, to=2-2]
	\arrow[from=1-3, to=2-3]
	\arrow[from=2-1, to=2-2]
	\arrow[from=2-2, to=2-3]
\end{tikzcd}}\\~\\
in $\bold{CGWH}$. Let the right square be a homotopy pullback square. Then the left square is a homotopy pullback if and only if the rectangle is a homotopy pullback square. (6.4.4 Arkhowitz)
\end{prp}

\begin{prp}{}{} Consider the following (not necessarily commutative) square \\~\\
\adjustbox{scale=1.0,center}{\begin{tikzcd}
	{X_1} & {X_2} & {X_3} \\
	{Y_1} & {Y_2} & {Y_3}
	\arrow[from=1-1, to=1-2]
	\arrow[from=1-1, to=2-1]
	\arrow[from=1-2, to=1-3]
	\arrow[from=1-2, to=2-2]
	\arrow[from=1-3, to=2-3]
	\arrow[from=2-1, to=2-2]
	\arrow[from=2-2, to=2-3]
\end{tikzcd}}\\~\\
in $\bold{CGWH}$. Let the right square be a homotopy pullback square. Then the left square is a homotopy pullback if and only if the rectangle is a homotopy pullback square. (6.4.4 Arkhowitz)
\end{prp}

\begin{prp}{}{} Let $W,X,Y,Z\in\bold{CGWH}$ be spaces such that there is a (not necessarily commutative) diagram \\~\\
\adjustbox{scale=1.0,center}{\begin{tikzcd}
	W & Y \\
	X & Z
	\arrow[from=1-1, to=1-2]
	\arrow[from=1-1, to=2-1]
	\arrow[from=1-2, to=2-2]
	\arrow[from=2-1, to=2-2]
\end{tikzcd}}\\~\\
Then the following are true. 
\begin{itemize}
\item If the square is a homotopy pullback, and $Y\to Z$ is a weak equivalence, then $W\to X$ is a weak equivalence. 
\item If $Y\to Z$ and $W\to X$ are weak equivalence, then the square is a homotopy pullback. 
\end{itemize}
\end{prp}

\subsection{Relation to (Co)Fibrations}
\begin{prp}{}{} Let $X,Y,Z\in\bold{CGWH}$ be spaces. Let $f:X\to Z$ and $g:Y\to Z$ be maps. Then the following spaces are homeomorphic to $$\text{holim}(X\overset{f}{\rightarrow}Z\overset{g}{\leftarrow}Y)$$ via the canonical map. 
\begin{itemize}
\item $\lim(P_f\rightarrow Z\overset{g}{\leftarrow}Y)$
\item $\lim(X\overset{f}{\rightarrow}Z\leftarrow P_g)$
\item $\lim(P_f\rightarrow Z\leftarrow P_g)$
\end{itemize}
\end{prp}

In model theoretic terms: the homotopy pullback can be computed by the standard pullback of fibrant replacements. \\

\begin{prp}{}{} Let $X,Y,Z\in\bold{CGWH}$ be spaces. Let $f:Z\to X$ and $g:Z\to Y$ be maps. Then the following spaces are homeomorphic to $$\hocolim(X\overset{f}{\leftarrow}Z\overset{g}{\rightarrow}Y)$$ via the canonical map. 
\begin{itemize}
\item $\colim(M_f\leftarrow Z\rightarrow Y)$
\item $\colim(X\leftarrow Z\rightarrow M_g)$
\item $\colim(M_f\leftarrow Z\rightarrow M_g)$
\end{itemize}
\end{prp}

When one of the maps $f$ or $g$ is a fibration, then the notion of a pullback coincides with that of homotopy pullback. 

\begin{prp}{}{} Let $X,Y,Z\in\bold{CGWH}$ be spaces. Let $f:X\to Z$ and $g:Y\to Z$ be maps. If $f$ or $g$ is a fibration, then there is a homotopy equivalence $$\lim(X\overset{f}{\rightarrow}Z\overset{g}{\leftarrow}Y)\simeq\text{holim}(X\overset{f}{\rightarrow}Z\overset{g}{\leftarrow}Y)$$ induced by the canonical map. 
\end{prp}

\begin{prp}{}{} Let $X,Y,Z\in\bold{CGWH}$ be spaces. Let $f:Z\to X$ and $g:Z\to Y$ be maps. If $f$ or $g$ is a cofibration, then there is a homotopy equivalence $$\colim(X\overset{f}{\leftarrow}Z\overset{g}{\rightarrow}Y)\simeq\text{hocolim}(X\overset{f}{\leftarrow}Z\overset{g}{\rightarrow}Y)$$ induced by the canonical map. 
\end{prp}

Recall that the mapping path space $P_f$ of a map $f:X\to Y$ is defined to be $$P_f=f^\ast(\text{Map}(I,Y))=\{(x,\phi)\subseteq X\times\text{Map}(I,Y)\;|\;f(x)=\pi_0(\phi)=\phi(0)\}$$ we can now prove that $P_f$ is a homotopy invariance. 

\begin{prp}{}{} Let $X,Y\in\bold{CGWH}$ be spaces. Let $f,g:X\to Y$ be maps. If $f$ and $g$ are homotopic, then the following are true. 
\begin{itemize}
\item $P_f\simeq P_g$
\item $M_f\simeq M_g$
\end{itemize}
\end{prp}

\subsection{Relation to Homotopy (Co)Fibers}
\begin{prp}{}{} Let $X,Y\in\bold{CGWH}$ be spaces. Let $f,g:X\to Y$ be maps. If $f$ and $g$ are homotopic, then the following are true. 
\begin{itemize}
\item $\text{hofiber}_y(f)\simeq\text{hofiber}_y(g)$ for any $y\in Y$. 
\item $\text{hocofiber}(f)\simeq\text{hocofiber}(g)$
\end{itemize}
\end{prp}

Recall that the fiber of a map $f:X\to Y$ behaves poorly because the fibers are not homeomorphic and not even homotopy equivalent. However, we can now prove that the homotopy fibers are the correct notion of a fiber to study because they are homotopy equivalent. 

\begin{crl}{}{} Let $X,Y\in\bold{CGWH}$ be space. Let $f:X\to Y$ be a map. If $y_1,y_2\in Y$ are path connected, then there is a homotopy equivalence $$\text{hofiber}_{y_1}(f)=\text{hofiber}_{y_2}(f)$$
\end{crl}

We can we interpret homotopy pullbacks and pushouts using homotopy (co)fibers. 

\begin{prp}{}{} Let $W,X,Y,Z\in\bold{CGWH}$ be spaces such that following is a (not necessarily commutative) square \\~\\
\adjustbox{scale=1.0,center}{\begin{tikzcd}
	W & Y \\
	X & Z
	\arrow[from=1-1, to=1-2]
	\arrow[from=1-1, to=2-1]
	\arrow[from=1-2, to=2-2]
	\arrow[from=2-1, to=2-2]
\end{tikzcd}}\\~\\
Then the following are true. 
\begin{itemize}
\item The square is a homotopy pullback if and only if for all $x\in X$, the map $$\text{hofiber}_x(W\to X)\to\text{hofiber}_{f(x)}(Y\to Z)$$ is a weak equivalence. 
\item The square is $k$-cartesian if and only if for all $x\in X$, the map $$\text{hofiber}_x(W\to X)\to\text{hofiber}_{f(x)}(Y\to Z)$$ is $k$-connected. 
\end{itemize} \tcbline
\begin{proof}
Begin with the homotopy pullback square \\~\\
\adjustbox{scale=1.0,center}{\begin{tikzcd}
	{\text{holim}(X\rightarrow Z\leftarrow Y)} & Y \\
	X & Z
	\arrow[from=1-1, to=1-2]
	\arrow[from=1-1, to=2-1]
	\arrow[from=1-2, to=2-2]
	\arrow[from=2-1, to=2-2]
\end{tikzcd}}\\~\\
We know that there is a homotopy pullback square given by \\~\\
\adjustbox{scale=1.0,center}{\begin{tikzcd}
	{\text{holim}_y(g)} && Y \\
	\ast & X & Z
	\arrow[from=1-1, to=1-3]
	\arrow[from=1-1, to=2-1]
	\arrow["g", from=1-3, to=2-3]
	\arrow[from=2-1, to=2-2]
	\arrow[from=2-2, to=2-3]
\end{tikzcd}}\\~\\
We can view this as a homotopy pullback square where we consider the composition $\text{holim}_y(g)\to\ast\to X$ as one single map. The gives a comparison of a homotopy pullback square with the standard homotopy pullback. Hence we obtain \\~\\
\adjustbox{scale=1.0,center}{\begin{tikzcd}
	{\text{holim}_y(g)} & {\text{holim}(X\rightarrow Z\leftarrow Y)} & Y \\
	\ast & X & Z
	\arrow[from=1-1, to=1-2]
	\arrow[from=1-1, to=2-1]
	\arrow[from=1-2, to=1-3]
	\arrow[from=1-2, to=2-2]
	\arrow["g", from=1-3, to=2-3]
	\arrow[from=2-1, to=2-2]
	\arrow[from=2-2, to=2-3]
\end{tikzcd}}\\~\\
By the above prp, we conclude that the square on the left is a homotopy pullback. By the same method, we can glue two more squares to obtain the diagram: \\~\\
\adjustbox{scale=1.0,center}{\begin{tikzcd}
	{\text{holim}_y(\alpha)} & {\text{holim}_y(f)} & W \\
	\ast & {\text{holim}_y(g)} & {\text{holim}(X\rightarrow Z\leftarrow Y)} & Y \\
	& \ast & X & Z
	\arrow[from=1-1, to=1-2]
	\arrow[from=1-1, to=2-1]
	\arrow[from=1-2, to=1-3]
	\arrow["u", from=1-2, to=2-2]
	\arrow["\alpha", from=1-3, to=2-3]
	\arrow[from=2-1, to=2-2]
	\arrow[from=2-2, to=2-3]
	\arrow[from=2-2, to=3-2]
	\arrow[from=2-3, to=2-4]
	\arrow[from=2-3, to=3-3]
	\arrow["g", from=2-4, to=3-4]
	\arrow[from=3-2, to=3-3]
	\arrow[from=3-3, to=3-4]
\end{tikzcd}}\\~\\
If $\alpha$ is a weak equivalence, then since the top wide rectangle is a homotopy pullback we have $\text{holim}_y(\alpha)$ is weakly equivalent to $\ast$. But the top left square is a homotopy pullback hence $\text{holim}_y(f)$ is weakly equivalent to $\text{holim}_y(g)$. Conversely, suppose that $u$ is a weak equivalence. Since the top left square is a homotopy pullback, this implies that $\text{holim}_y(\alpha)$ is weakly contractible for all $y$. In articular it is $n$-connected for all $n$. Then this implies that $\alpha$ is $(n+1)$-connected for all $n$. Hence $\alpha$ is a weak equivalence. 
\end{proof}
\end{prp}

\subsection{Connectedness of Homotopy Squares}

\pagebreak
\section{Blakers-Massey Theorem}
\subsection{The Blakers-Massey Theorem for Squares}
The Blakers-Massey theorem is a direct generalization of the homotopy excision theorem. Its proof takes a similar form to the homotopy excision theorem. Let us recall some definitions used. 

\begin{defn}{(Degenerative) Cubes}{} Let $a=(a_1,\dots,a_n)\in\R^n$. Let $\delta>0$. Let $L\subseteq\{1,\dots,n\}$. A cube in $\R^n$ is a set of the form $$W=W(a,\delta,L)=\{x\in\R^n\;|\;a_i\leq x\leq a_i+\delta\text{ for }i\in L\text{ and }x_i=a_i\text{ for }i\notin L\}$$
\end{defn}

The notation is making the object more complicated than what it should look like. $a\in\R^n$ refers to the bottom left coordinate of the cube. $\delta$ is the length of the cube and $L$ refers to the number of non-degenerate faces of the cube. In particular, any cube in $\R^n$ is homeomorphic to the standard cube $I^k$ for some $k\leq n$. 
\begin{itemize}
\item When $n=3$, $W(0,1,\{1,2\})$ is the unit square on the $xy$-plane. 
\item When $n=3$, $W(0,1,\{1,2,3\})$ is the unit cube. 
\item When $n=4$, $W(0,1,\{1,2,3\})$ is the unit cube with nonzero first three coordinates and zero otherwise. 
\end{itemize}

\begin{defn}{Special Sub-cube of a Cube}{} Let $W=W(a,\delta,L)$ be a cube in $\R^n$. Let $j=1$ or $2$. Suppose that $1\leq p\leq\abs{L}$. Define $$K_p^j(W)=\left\{(x_1,\dots,x_n)\in W\;\bigg{|}\;\frac{\delta(j-1)}{2}+a_i<x_i<\frac{\delta j}{2}+a_i\text{ for at least }p\text{ values of }i\in L\right\}$$
\end{defn}

Again the notation is making the object more complicated. Taking $W=W(0,1,\{1,2,3\})=I^3$ in $\R^3$, we have 
\begin{itemize}
\item $K_3^1(W)=W(0,1/2,\{1,2,3\})$ is one eighth of the cube $I^3$ with bottom left corner at the origin. 
\item $K_3^2(W)=W((1/2,1/2,1/2),1/2,\{1,2,3\})$ is one eighth of the cube $I^3$ with bottom left corner at $(1/2,1/2,1/2)$
\item $K_2^1(W)$ allows for one coordinate to go beyond the bottom left one eighth of the cube, and is the union four of the 1/8-sub-cubes that are adjacent to the $xy$-plane, the $yz$-plane and the $xz$-plane. 
\item $K_1^1(W)$ allows for two coordinate to go beyond the bottom left one eighth of the cube, and is equal to $W\setminus K_3^2(W)$. 
\item $K_0^1(W)$ allows for all coordinate to go beyond the bottom left one eighth of the cube, so the condition becomes vacuous and is equal to $W$. 
\end{itemize}

Summarizing, we think of $K_p^j(W)$ as follows. Subdivide the $\abs{L}$-dimensional cube into $2^\abs{L}$ sub-cubes of equal volume. $K_p^1$ is the union of a number of sub-cubes closest to the bottom left sub-cube. $K_p^2$ is the union of a number of sub-cubes closed to the upper right sub-cube. 

\begin{lmm}{}{} Let $Y$ be a space. Let $B\subseteq Y$ be a subspace of $Y$. Let $W=W(a,\partial, L)$ be a cube in $\R^n$. Let $f:W\to Y$ be a map. Let $j=1$ or $2$. Suppose that there exists some $p\leq\abs{L}$ such that $$f^{-1}(B)\cap C\subset K_p^j(C)$$ for all cubes $C\subset\partial W$. Then there exists a map $g:W\to Y$ such that $g\overset{\partial W}{\simeq} f$ and $$g^{-1}(B)\subset K_p^j(W)$$ \tcbline
\begin{proof}
(Proof by Munson in Cubical Homotopy Theory)Firstly, notice that any cube $W$ is homeomorphic to $I^n$ for some $n$, so we can just prove the statement for when $W=I^n$. In this case, our parameters of the cube is given by $I^n=W(a=0,\delta=1,L=\{1,\dots,n\})$ and our $K_p^j(W)$ is given by $$K_p^j(W)=\left\{(x_1,\dots,x_n)\in C\;\bigg{|}\;\frac{j-1}{2}<x_i<\frac{j}{2}\text{ for at least }p\text{ values of }i\in\{1,\dots,n\}\right\}$$~\\

Let $p_j$ be the center of the sub-cube $\left[\frac{j-1}{2},\frac{ji}{2}\right]^n$ inside $I^n$ for $j=1,2$. Let $R$ be a ray with starting point $p_j$. Let $P(R,p_j)$ be the intersection of $R$ and $\partial\left[\frac{j-1}{2},\frac{ji}{2}\right]^n$. Let $Q(R,p_j)$ be the intersection of $R$ and $\partial I^n$. By construction, the points $p_j$, $P(R,p_j)$ and $Q(R,p_j)$ are collinear with $P(R,p_j)$ always being the mid point and $P(R,p_j)$ is possibly equal to $Q(R,p_j)$. Being a line, we can define a linear homotopy from the line $[p_j,P(R,p_j)]$ to the line $[p_j,Q(R,p_j)]$ that fixes the point $p_j$ and sends $P(R,p_j)$ to $Q(R,p_j)$. Denote the homotopy by $h(y,t)$ for $y\in[p_j,P(R,p_j)]$ and $t$ the time variable. \\~\\

Now we can define a homotopy $H_j:I^n\times I\to I^n$ as follows: For each $y\in I^n$, there exists a unique ray $R$ starting at $p_j$ and passing through $y$. Then we obtain a homotopy $h$ from $[p_j,P(R,p_j)]$ to $[p_j,Q(R,p_j)]$ as above. Define $H(y,t)=h(y,t)$. It is clear that $H_j(q,1)=q$ for all $q\in\partial I^n$ so that $H_j$ is a homotopy from the identity, relative to the boundary $\partial I^n$. \\~\\

Let $g=f\circ H_j(-,1)$. From the properties of the homotopy $H_j$, we notice that $f\circ H_j:I^n\times I\to Y$ is a homotopy from $f\circ H(-,0)=f\circ\text{id}=f$ to $f\circ H(-,1)=g$, relative to the boundary $\partial I^n$. Thus we now have a homotopy from $f$ to a map $g$ relative to the boundary. It remains to show that $g^{-1}(B)\subset K_p^j(C)$. \\~\\

Let $z=(z_1,\dots,z_n)\in g^{-1}(B)$. If $z\in\left[\frac{j-1}{2},\frac{j}{2}\right]^n$ then clearly $z\in K_p^j(C)$ is true. So suppose instead that $z=(z_1,\dots,z_n)\in g^{-1}(B)$ satisfies the fact that either $z_a\geq\frac{j}{2}$ or $z_b\leq\frac{j-1}{2}$ for $1\leq a,b\leq n$. Let $R$ be the ray from $p_j$ passing through $z$. Then the condition on $z$ means that $z\in[P(R,p_j),Q(R,p_j)]$. Hence under the homotopy $H$, $z$ is mapped to $\partial I^n$. But $\partial I^n$ is a union $n-1$ dimensional faces of $I^n$ which are cubes. So $H(z,1)$ lies in some cube $C\subseteq\partial I^n$. By construction of $g$, $g(z)=f(H(z,1))$ and $g(z)\in B$ implies that $H(z,1)\in f^{-1}(B)$. Then $H(z,1)\in f^{-1}(B)$ and $H(z,1)\in C$ implies that $$H(z,1)\in f^{-1}(B)\cap C\subseteq K_p^j(C)$$ by the assumption on $f$. Write $H(z,1)=(w_1,\dots,w_n)\in\partial I^n$. This means that $\frac{j-1}{2}<w_i<\frac{j}{2}$ for at least $p$ of the coordinates of $H(z,1)$. \\~\\

Now the ray starting at $p_j$ and passing through $z$ is parametrized by the line $p_j+t(z-p_j)$ for $t\geq 0$. Since $H(z,1)$ lies behind the two points $z$ and $p_j$, we can write $H(z,1)=p_j-t_0(z-p_j)$ for some $t_0\geq 1$. By definition, $p_j$ is the point given in coordinates by $\left(\frac{2j-1}{4},\dots,\frac{2j-1}{4}\right)$. Hence the $i$th coordinate of $H(z,1)$ can be written as $$w_i=\frac{2j-1}{4}+t_0\left(z_i-\frac{2j-1}{4}\right)$$ Recall that in the previous paragraph we found that $\frac{j-1}{2}<w_i<\frac{j}{2}$ for at least $p$ of the coordinates of $H(z,1)$. Substituting $w_i$ into the inequality and simplifying gives $$-\frac{1}{4t_0}+\frac{2j-1}{4}<z_i<\frac{1}{4t_0}+\frac{2j-1}{4}$$ Since $t_0\geq 1$, we get $$-\frac{1}{4}+\frac{2j-1}{4}<-\frac{1}{4t_0}+\frac{2j-1}{4}<z_i<\frac{1}{4t_0}+\frac{2j-1}{4}<\frac{1}{4}+\frac{2j-1}{4}$$ The leftmost and rightmost terms bound $z_i$ between $\frac{j-1}{2}$ and $\frac{j}{2}$ for at least $p$ amount of coordinates $z_i$ of $z$. Hence $z\in K_p^j(C)$. This completes the proof. 
\end{proof}
\end{lmm}

\begin{lmm}{}{} Let $X$ be a space. Let $X_0,X_1,X_2\subseteq X$ be subspaces of $X$ such that $$X=X_1\cup X_2$$ and $X_0=X_1\cap X_2$ is non-empty. Assume that for each $i=1,2$, $(X_i,X_0)$ is $k_i$-connected with $k_i\geq 0$. Let $f:I^n\to X$ be a map. Let $$I^n=\bigcup_k W_k$$ be the decomposition of $I^n$ into cubes $W_k$ such that $f(W_k)\subseteq X_i$ for one of $i=0,1,2$ by the Lebesgue covering lemma. Then there exists a homotopy $$H:I^n\times I\to X$$ such that the following are true. 
\begin{itemize}
\item $f(-)=H(-,0)$
\item If $f(W)\subset X_i$, then $H(W,t)\subset X_i$ for all $t\in I$. 
\item If $f(W)\subset X_0$, then $H(W,t)=f(W)$ for all $t\in I$. 
\item If $f(W)\subset X_i$, then $\left((H(-,1))^{-1}(X_i\setminus X_0)\right)\cap W\subset K_{k_i+1}^i(W)$. 
\end{itemize} \tcbline
\begin{proof}
Let $C^d$ be the union of all cubes of dimension $\leq d$. We induct on $d$, the existence of such a homotopy $H:C^d\times I\to X$ that holds the required conditions true for all cubes $W$ with dimension $\leq d$. \\~\\

We first construct the homotopy for all cubes of dimension $0$. When $\dim(W)=0$, there are two cases: 
\begin{itemize}
\item If $f(W)\subset X_0$, define $H|_{W\times I}$ by $H(w,t)=f(w)$
\item If $f(W)\subset X_j$ and $f(W)\not\subset X_i$ for $1\leq i\neq j\leq 2$, $(X_j,X_0)$ is $(k_j\geq 0)$-connected implies that there exists a path $\gamma:I\to X$ from $f(W)$ to a point in $X_0$. Define $H|_{W\times I}$ by $H(w,t)=\gamma(t)$ (again $W=\{w\}$ is a one point set). 
\end{itemize}
Thus we now have a well defined map $H:C^0\times I\to X$. We need to show that this map satisfies the required conditions. 
\begin{itemize}
\item For each $z\in C^0$, either $H(z,0)=f(z)$ from the first case or $H(z,0)=\gamma(0)=f(z)$. 
\item If $f(W)\subset X_i$, then by construction $H(W,t)\subset X_i$ from the second case. 
\item If $f(W)\subset X_0$, then $H(W,t)=f(W)$ by the first case. 
\item $K_{k_i+1}^i(W)=\{w\}$ is a one point set and $(H(-,1))^{-1}(X_i\setminus X_0)\cap W\subseteq W$ means that this condition is satisfied. 
\end{itemize}~\\

Now $H$ is built on three pieces: the union of cubes landing in $X_i$ for $i=0,1,2$. The second and third conditions guarantee that each of the three pieces define a homotopy on each piece respectively. Since $\partial W\hookrightarrow W$ is a cofibration, we can extend these pieces of homotopy from $0$-dimensional cubes to $1$-dimensional. Recursively we are able to define a homotopy for all cubes of all dimensions inside $I^n$ that satisfy the first three conditions. \\~\\

Therefore now we can invoke the inductive hypothesis, so that there exists a homotopy from $f$ so that the new function satisfy all our required conditions for all cubes of dimension $<d$. With abuse of notation, call the restriction of our newly acquired function to $C^{d-1}$ also by the name $f$. Let $W$ be a cube of dimension $d$. 
\begin{itemize}
\item If $f(W)\subseteq X_0$, define $H|_{W\times I}$ by $H(w,t)=f(w)$
\item If $f(W)\subset X_1$ and $f(W)\not\subset X_2$ and $\dim(W)=d\leq k_1$, $(X_j,X_0)$ is $(k_j\geq 0)$-connected implies there exists a homotopy $K:W\times I\to X$ from $f$ relative to $\partial W$ such that $K(W,1)\subseteq X_0$. Define $H|_{W\times I}$ by $H=K$. 
\item If $f(W)\subset X_1$ and $f(W)\not\subset X_2$ and $\dim(W)=d>k_1$, then by induction we have $$f^{-1}(X_1\setminus X_0)\cap W'\subset K_d^1(W')\subset K_{k_1+1}^1(W')$$ for all $W'\subset\partial W$ (induction is applicable since $\dim(W')<\dim(W)$). By the above lemma, there exists a map $g:W\to X$ such that $g$ and $f$ are homotopic relative to $\partial W$ such that $g^{-1}(X_1\setminus X_0)\subset K_{k_1+1}^1(W)$. Call this homotopy from $f$ to $g$ by $R:W\times I\to X$. Then we define $H|_{W\times I}$ by $H=R$. 
\end{itemize}
(WLOG the cases where $X_1$ is swapped with $X_2$ and $k_1$ is swapped with $k_2$ and $K_p^1(W)$ is swapped with $K_p^2(W)$ in the last two sub-cases has a symmetrical argument). Finally we show that our required conditions are satisfied. 
\begin{itemize}
\item In all cases, $H(-,0)=f$ as one can immediately see. 
\item The second condition holds for all cubes of dimension $<d$ by inductive hypothesis. It also holds for our first and second case since $X_0\subset X_1,X_2$. For the third case, $H$ is a homotopy relative to $\partial W$. Since by induction hypothesis $f(\partial W)\subseteq X_1$, we also have $g(\partial W)\subseteq X_1$. Since $g$ is continuous then $H(W,1)=g(W)\subseteq X_1$. 
\item The third condition holds for all cubes of dimension $<d$ by inductive hypothesis, and holds true for all cubes of dimension $d$ by the first case. 
\item The fourth condition holds true by our argument in the third case, and is vacuously true in the second case since $H(W,1)\subseteq X_0$ implies that $(H(-,1))^{-1}(X_1\setminus X_0)=\emptyset$. 
\end{itemize}
Thus the proof is complete. 
\end{proof}
\end{lmm}

We can now prove a weaker version of Blakers-Massey theorem. 

\begin{prp}{}{} Let $X$ be a space. Let $e^{d_i}$ be a cell of dimension $d_i$ for $i=1,2$. Then the following diagram \\~\\
\adjustbox{scale=1.0,center}{\begin{tikzcd}
	X & X\cup e^{d_1} \\
	X\cup e^{d_2} & X\cup e^{d_1}\cup e^{d_2}
	\arrow[from=1-1, to=1-2]
	\arrow[from=1-1, to=2-1]
	\arrow[from=1-2, to=2-2]
	\arrow[from=2-1, to=2-2]
\end{tikzcd}}\\~\\
given by inclusion maps is $(d_1+d_2-3)$-cartesian. \tcbline
\begin{proof}
(Proof is by Munson in Cubical Homotopy Theory) Let $p_1\in e^{d_1}$ and $p_2\in e^{d_2}$ be interior points. Since $X\cup e^{d_2}$ is weakly equivalent to $X\cup e^{d_1}\cup e^{d_2}\setminus\{p_1\}$ by inclusion (and similarly for $X\cup e^{d_1}$), the above square admits a weak equivalence to the following square: \\~\\
\adjustbox{scale=1.0,center}{\begin{tikzcd}
	{X\cup e^{d_1}\cup e^{d_2}\setminus\{p_1,p_2\}} & {X\cup e^{d_1}\cup e^{d_2}\setminus\{p_2\}} \\
	{X\cup e^{d_1}\cup e^{d_2}\setminus\{p_1\}} & {X\cup e^{d_1}\cup e^{d_2}}
	\arrow[hook, from=1-1, to=1-2]
	\arrow[hook, from=1-1, to=2-1]
	\arrow[hook, from=1-2, to=2-2]
	\arrow[hook, from=2-1, to=2-2]
\end{tikzcd}}\\~\\
Let $Y=X\cup e^{d_1}\cup e^{d_2}$. By thm???? to show that the square is $(d_1+d_2-3)$-cartesian is the same as showing the map $$\text{hofiber}_y(Y\setminus\{p_1,p_2\}\to Y\setminus\{p_1\})\to\text{hofiber}_y(Y\setminus\{p_2\}\to Y)$$ is $(d_1+d_2-3)$-cartesian. Let $$C=\text{hofiber}_y(Y\setminus\{p_2\}\to Y\setminus\{p_2\})\simeq\ast$$ 

Now the intersections below can be calculated to give $$C\cap\text{hofiber}_y(Y\setminus\{p_1,p_2\}\to Y\setminus\{p_1\})=\text{hofiber}_y(Y\setminus\{p_1,p_2\}\to Y\setminus\{p_1,p_2\})$$ and $$C\cap\text{hofiber}_y(Y\setminus\{p_2\}\to Y)=\text{hofiber}_y(Y\setminus\{p_2\}\to Y\setminus\{p_2\})\simeq\ast$$ (because $C$ and the homotopy fibers are all subspaces of $Y\times\text{Map}(I,Y)$). By thm??? we conclude that the square \\~\\
\adjustbox{scale=1.0,center}{\begin{tikzcd}
	{C\cap\text{hofiber}_y(Y\setminus\{p_1,p_2\}\to Y\setminus\{p_1\})} & {\text{hofiber}_y(Y\setminus\{p_1,p_2\}\to Y\setminus\{p_1\})} \\
	C & {C\cup\text{hofiber}_y(Y\setminus\{p_1,p_2\}\to Y\setminus\{p_1\})}
	\arrow[hook, from=1-1, to=1-2]
	\arrow[hook, from=1-1, to=2-1]
	\arrow[hook, from=1-2, to=2-2]
	\arrow[hook, from=2-1, to=2-2]
\end{tikzcd}}\\~\\
is a homotopy pullback. Since the left two terms are contractible, it is a weak equivalence. By thm??? we conclude that the map on the right is a weak equivalence. Similarly, we can consider the diagram \\~\\
\adjustbox{scale=1.0,center}{\begin{tikzcd}
	{C\cap\text{hofiber}_y(Y\setminus\{p_2\}\to Y)} & {\text{hofiber}_y(Y\setminus\{p_2\}\to Y)} \\
	C & {C\cup\text{hofiber}_y(Y\setminus\{p_2\}\to Y)}
	\arrow[hook, from=1-1, to=1-2]
	\arrow[hook, from=1-1, to=2-1]
	\arrow[hook, from=1-2, to=2-2]
	\arrow[hook, from=2-1, to=2-2]
\end{tikzcd}}\\~\\
which is a homotopy pullback. And the fact that the terms on the left are contractible imply that the map on the right (which is in fact equal) $$\text{hofiber}_y(Y\setminus\{p_2\}\to Y)\overset{=}{\hookrightarrow}\text{hofiber}_y(Y\setminus\{p_2\}\to Y)$$ is a weak equivalence by the same theorem. We now have a chain of maps \\~\\
\adjustbox{scale=1.0,center}{\begin{tikzcd}
	{\text{hofiber}_y(Y\setminus\{p_1,p_2\}\to Y\setminus\{p_1\})} & {C\cup\text{hofiber}_y(Y\setminus\{p_1,p_2\}\to Y\setminus\{p_1\})} \\
	\\
	{C\cup \text{hofiber}_y(Y\setminus\{p_2\}\to Y)} & {\text{hofiber}_y(Y\setminus\{p_2\}\to Y)}
	\arrow["{\text{weak eq.}}", hook, from=1-1, to=1-2]
	\arrow[hook, from=1-2, to=3-1]
	\arrow["{\text{weak eq.}}", from=3-1, to=3-2]
\end{tikzcd}}\\~\\
where the middle map is inclusion. It remains to show that the middle map is $(d_1+d_2-3)$-connected. This is the same as showing that the pair $$(C\cup \text{hofiber}_y(Y\setminus\{p_2\}\to Y),C\cup\text{hofiber}_y(Y\setminus\{p_1,p_2\}\to Y\setminus\{p_1\}))$$ is $(d_1+d_2-3)$-connected. \\~\\

To simplify notations let us write the pair as $(A,B)$. Let $\phi:(I^n,\partial I^n)\to(A,B)$ be a map. Recall that $A=\{(x,\phi)\in Y\times\text{Map}(I,Y)\;|\;\phi(0)=y,\phi(1)=x\}$. The first variable is determined by the end point of $\phi$ so giving a map $I^n\to A$ is the same as giving a map $I^n\to\text{Map}(I,Y)$ for which all paths in the image has starting point $y$ and ending point in $Y\setminus\{p_2\}$. By the hom-product adjunction, this is equivalent to giving a map $\psi:I^n\times I\to Y$ such that $\psi(z,0)=y$ is the base point and $\psi(z,1)\in Y\setminus\{p_2\}$. Similarly, we can consider the map $\phi:\partial I^n\to B$ and deduce that $\phi(z)$ is a path lying entirely in the codomain of the map of the homotopy fiber $C=\text{hofib}_y(Y\setminus\{p_2\}\to Y\setminus\{p_2\})$ or it is a path lying entirely in the codomain of the map of the homotopy fiber $\text{hofiber}_y(Y\setminus\{p_1,p_2\}\to Y\setminus\{p_1\}))$. By the same adjunction we conclude that our $\psi$ above must also satisfy for any fixed $z\in\partial I^n$, $\psi(z,t)$ lies entirely in either $Y\setminus\{p_1\}$ or $Y\setminus\{p_2\}$ (and conversely these information give a map $(I^n,\partial I^n)\to(A,B))$ by the adjunction). \\~\\

To summarize: we have a map $$\psi:I^n\times I\to Y$$ such that 
\begin{itemize}
\item $\psi(z,0)=y$ is the base point for all $z\in I^n$. 
\item $\psi(z,1)\in Y\setminus\{p_2\}$ for all $z\in I^n$. 
\item For any fixed $z\in\partial I^n$, $\psi(z,t)$ lies entirely in $Y\setminus\{p_1\}$ or $Y\setminus\{p_2\}$ for varying $t$. 
\end{itemize}
The goal is to make a homotopy from $\psi$ to a map whose third condition holds for any $z\in I^n$ when $n\leq d_1+d_2-3$. Then passing through the adjunction again we see that our original map $(I^n,\partial I^n)\to(A,B)$ is homotopic to the constant map as required. Apply 5.1.4 to obtain a homotopy $H:I^n\times I\times I\to Y$ from $\psi$ to a new map $\eta:I^n\times I\to Y$, such that we have a decomposition of $I^n\times I$ into cubes $W$ and the following are true. 
\begin{enumerate}
\item $\psi(W)\subset Y\setminus\{p_2\}$ implies $H(W,r)\subset Y\setminus\{p_2\}$ for all $r\in I$. 
\item $\psi(W)\subset Y\setminus\{p_1\}$ implies $H(W,r)\subset Y\setminus\{p_1\}$ for all $r\in I$. 
\item $\psi(W)\subset Y\setminus\{p_1,p_2\}$ implies $H(W,r)=\psi(W)$ for all $r\in I$. 
\item $\psi(W)\subset Y\setminus\{p_2\}$ then $(H(-,1)^{-1}(\{p_1\}))\cap W\subset K_{d_1}^1(W)$
\item $\psi(W)\subset Y\setminus\{p_1\}$ then $(H(-,1)^{-1}(\{p_2\}))\cap W\subset K_{d_2}^2(W)$
\end{enumerate}
We claim that $H(z,t,r)$ satisfies the three bullet points for all fixed $r$. 

Firstly, we already know that $\psi(z,0)=y$ is the base point for all $z\in I^n$. So for all cubes $W\subseteq I^n\times\{0\}$, we have $\psi(W)=\{y\}\subset Y\setminus\{p_1,p_2\}$. By 3., we conclude that $H(W,r)=\psi(W)=\{y\}$. Secondly, we know that $\psi(z,1)\in Y\setminus\{p_2\}$ for all $z\in I^n$. So for all cubes $W\subseteq I^n\times\{1\}$, we have $\psi(W)\subset Y\setminus\{p_2\}$. By 1., we conclude that $H(W,r)\subset Y\setminus\{p_2\}$ for all $r\in I$. Finally, according to the third bullet point, $\psi(z,I)$ lies entirely in $Y\setminus\{p_1\}$ or $Y\setminus\{p_2\}$ for $z\in\partial I^n$ WLOG lets say it lies entirely in $Y\setminus\{p_i\}$. Choose cubes $W_1,\dots,W_k$ in the decomposition of $I^n\times I$ so that it forms a minimal cover for $\{z\}\times I\subset W_1\cup\cdots\cup W_k$. By definition of the decomposition, these cubes firstly contain at least one point in $\{z\}\times I$, and $\psi(\{z\}times I)\subset Y\setminus\{p_j\}$ implies that $\psi(W_1),\dots,\psi(W_k)\subset Y\setminus\{p_j\}$. By 1., we conclude that $H(W_1,r),\dots,H(W_k,r)\subset Y\setminus\{p_j\}$ so that $H(W_1\cup\cdots\cup W_k,r)\subset Y\setminus\{p_j\}$. \\~\\

It remains to show that $\eta(-,-)=H(-,-,1)$ satisfies the stronger condition of the third bullet point as desired. Let $n\leq d_1+d_2-3$. We want to show that $\eta(z,I)\subset Y\setminus\{p_j\}$ for some $j$. I claim that this is equivalent to saying $$\text{proj}\left(\eta^{-1}(\{p_1\})\right)\cap\text{proj}\left(\eta^{-1}(\{p_2\})\right)=\emptyset$$ where $\text{proj}$ is the projection to the first coordinate. Indeed if $\eta(z,I)$ always lie inside one of $Y\setminus\{p_j\}$, $j=1,2$, then $\text{proj}\left(\eta^{-1}(\{p_1\})\right)=\{z\in I^n\;|\;\eta(z,I)\subset Y\setminus\{p_1\}\}$ and similarly for the other projection so that their intersection is empty. Conversely if one of $\eta(z,I)$ does not entirely in $Y\setminus\{p_2\}$ then the intersection is non-empty. \\~\\

So it suffices to prove that the intersection given above is empty. So suppose it is non-empty with an element $z_0$. Then there exists $t_1,t_2\in I$ such that $\eta(z_0,t_1)\in Y\setminus\{p_2\}$ and $\eta(z_0,t_2)\in Y\setminus\{p_1\}$. Choose cubes $W_1=W(a_1,\delta_1,L_1),W_2=W(a_2,\delta_2,L_2)$ in the given decomposition of $I^n\times I$ so that $(z_0,t_1)\in W_1$ and $(z_0,t_2)\in W_2$. By 4. and 5. we have $(z_0,t_1)\in\eta^{-1}(\{p_1\})\cap W_1\subset  K_{d_1}^1(W_1)$ and similarly for $(z_0,t_2)$. Then $(z_0,t_j)$ has at least $d_j$ coordinates satisfying the inequalities to lie in $K_{d_j}^1$. Hence $z_0=\text{proj}(z_0,t)$ has at least $d_j-1$ coordinates satisfying those inequalities for each $j=1,2$. For each $j$, $\text{proj}(W_j)$ is a cube containing $z_0$. Subdivide the cubes $W_1$ and $W_2$ further so that $\text{proj}(W_1)=\text{proj}(W_2)$. Since $\text{proj}(z_0,t)=z_0$, this means that $z_0$ has at least $d_j-1$ coordinates satisfying the inequalities of $K_{d_1}^1(W_1)$ and $K_{d_2}^2(W_2)$. But notice that the inequalities of $K_{d_1}^1(W_1)$ and $K_{d_2}^2(W_2)$ are disjoint (one concerns whether the points are at the front of the cube, the other at the back). So these conditions are disjoint and $z_0$ must have at least $d_1+d_2-2$ conditions on its coordinates. This is impossible if $z_0$ has less than $d_1+d_2-3$ coordinates. Hence we are done. \\~\\

Notice that the proof required $d_1,d_2\geq 1$. Assume WLOG that $d_2=0$, then right from the beginning we are considering the map of homotopy fibers $$\text{hofiber}_y(X\to X\cup e^{d_2})\to\text{hofiber}_y(X\cup e{d_1}\to X\cup e^{d_1}\cup e^{d_2})$$ where $X\cup e^{d_1}$ is now the disjoint union of $X$ with a base point. Then $\text{hofiber}_y(X\to X\amalg\ast)$ consists of pairs $(x,\phi)\in X\times\text{Map}(I,X\amalg\ast)$ such that $\phi(0)\in X$ and $\phi(1)=\ast$. But $\ast$ is disjoint from $X$ means that no pairs satisfy this conditions and the homotopy fiber is the empty set. Similarly for the target homotopy fiber. Hence the map of homotopy fibers is the identity and it is trivially true. The proof is similar for $d_1=0$. 
\end{proof}
\end{prp}

\begin{thm}{Blakers-Massey Theorem for Squares}{} Let $X_0,X_1,X_2,X_{12}\in\bold{CGWH}$ be spaces such that the square \\~\\
\adjustbox{scale=1.0,center}{\begin{tikzcd}
	X_0 & X_1 \\
	X_2 & X_{12}
	\arrow[from=1-1, to=1-2]
	\arrow[from=1-1, to=2-1]
	\arrow[from=1-2, to=2-2]
	\arrow[from=2-1, to=2-2]
\end{tikzcd}}\\~\\
is a homotopy pushout. Suppose the map $X_0\to X_i$ is $k_i$-connected for $i=1,2$. Then the diagram is $(k_1+k_2-1)$-cartesian. Explicitly, this means that $$\alpha:X_0\to\text{holim}(X_1\rightarrow X_{12}\leftarrow X_2)$$ is $(k_1+k_2-1)$-connected. \tcbline
\begin{proof}

\end{proof}
\end{thm}

This theorem directly generalizes the homotopy excision theorem in the following way. For $X$ a CW complex and $A,B$ two subcomplexes with non-empty intersection and $X=A\cup B$, consider the following square of inclusions: \\~\\
\adjustbox{scale=1.0,center}{\begin{tikzcd}
	A\cap B & A \\
	B & X
	\arrow[from=1-1, to=1-2]
	\arrow[from=1-1, to=2-1]
	\arrow[from=1-2, to=2-2]
	\arrow[from=2-1, to=2-2]
\end{tikzcd}}\\~\\
We have seen that such a square diagram is a homotopy pushout diagram. Now any inclusion map $W\hookrightarrow Z$ is $k$-connected if and only if $(Z,W)$ is $k$-connected. So $(A,A\cap B)$ is $k_1$-connected and $(X,B)$ is $k_2$-connected. Blaker's-Massey theorem implies that $$\text{hofiber}(A\cap B\to A)\to\text{hofiber}(B,X)$$ is $(k_1+k_2-1)$-connected. But by definition we have an isomorphism $\pi_k(\text{hofiber}(U\to V)\cong\pi_{k+1}(V,U)$. So we are really just saying that $\pi_k(A,A\cap B)\to\pi_k(X,B)$ given by the inclusion is $(k_1+k_2)$-connected. 

\subsection{The Dual Blakers-Massey Theorem for Squares}
\begin{thm}{Dual Blakers-Massey Theorem for Squares}{} Let $X_0,X_1,X_2,X_{12}\in\bold{CGWH}$ be spaces such that the square \\~\\
\adjustbox{scale=1.0,center}{\begin{tikzcd}
	X_0 & X_1 \\
	X_2 & X_{12}
	\arrow[from=1-1, to=1-2]
	\arrow[from=1-1, to=2-1]
	\arrow[from=1-2, to=2-2]
	\arrow[from=2-1, to=2-2]
\end{tikzcd}}\\~\\
is a homotopy pullback. Suppose the map $X_i\to X_{12}$ is $k_i$-connected for $i=1,2$. Then the diagram is $(k_1+k_2-1)$-cocartesian. Explicitly, this means that $$\beta:\hocolim(X_1\leftarrow X_0\rightarrow X_2)\to X_{12}$$ is $(k_1+k_2-1)$-connected.
\end{thm}

\pagebreak
\section{Homotopy n-Cubes}
In algebraic topology, we have learnt about spaces, maps of spaces and maps of maps of spaces. We can say this in a more compact way. Namely, if we think of maps of maps of space as a square (2-cube), we can think of spaces as 0-cubes and maps of spaces as 1-cube. We have studied 2-cubes extensively under the guise of homotopy pullbacks and pushouts. We can now take this further and consider general n-cubes. 

\subsection{Maps between Homotopy Squares}

\subsection{n-Cubes of Spaces}
\begin{defn}{n-Cubes of Spaces}{} Let $n\in\N$. Let $P(n)$ denote the category of posets of the set $\{1,\dots,n\}$. An $n$-cube of spaces is a functor $$X:P(n)\to\bold{CGWH}$$ An $n$-cube of based spaces is a functor $X:P(n)\to\bold{CGWH}_\ast$. 
\end{defn}

Explicitly, an $n$-cube of spaces $X:P(n)\to\bold{CGWH}$ consists of the following data. 
\begin{itemize}
\item For each $S\subseteq\{1,\dots,n\}$ a space $X_S$
\item For each $S\subseteq T$, a map $f_{S\subseteq T}:X_S\to X_T$ such that $f_{S\subseteq S}=1_{X_S}$ and for all $R\subseteq S\subseteq T$, we have a commutative diagram \\~\\
\adjustbox{scale=1.0,center}{\begin{tikzcd}
	{X_R} & {X_S} \\
	& {X_T}
	\arrow["{f_{R\subseteq S}}", from=1-1, to=1-2]
	\arrow["{f_{R\subseteq T}}"', from=1-1, to=2-2]
	\arrow["{f_{S\subseteq T}}", from=1-2, to=2-2]
\end{tikzcd}}\\~\\
\end{itemize}

Omit drawing composite arrows and omit drawing identities. \\
Also: punctured cubes def

\begin{defn}{Cube of Cubes}{} An $n$-cube of $m$-cubes is a functor $$X:P(n)\times P(m)\to\bold{CGWH}$$
\end{defn}

\begin{lmm}{}{} An $n$-cube of $m$-cubes $X$ is precisely an $(n+m)$-cube. 
\end{lmm}

\begin{defn}{Map of $n$-Cubes}{} Let $X,Y:P(n)\to\bold{CGWH}$ be $n$-cubes. A map of $n$-cubes is a natural transformation $F:X\to Y$ such that the assignment $Z:P(n+1)\to\bold{CGWH}$ given by $$Z(S)=\begin{cases}
X(S) & \text{ if } S\subseteq\{1,\dots,n\}\\
Y(S\setminus\{1,\dots, n+1\}) & \text{ if }\{1,\dots,n+1\}\subseteq S
\end{cases}$$
defines an $(n+1)$-cube. 
\end{defn}

objectwise (co)fibration, homotopy (weak) equivalence. homeomorphism

\begin{defn}{Strongly Homotopy Cartesian}{} Let $X$ be an $n$-cube of spaces. We say that $X$ is strongly homotopy cartesian if each of its faces of dimension $n\geq 2$ is homotopy cartesian. 
\end{defn}

\pagebreak
\section{Homotopy Limits and Colimits}

Let $X:\mJ\to\bold{Top}$ be a diagram of spaces. Denote the constant functor of the one point space by $\Delta\ast:\bold{X}\to\bold{Top}$. The data of the constant functor is given as follows. 
\begin{itemize}
\item For each $I\in\mJ$, $\Delta\ast(I)=\ast$
\item For each morphism $f:I\to J$ in $X$, define $\Delta\ast(f)=\text{id}_\ast$. 
\end{itemize}
Consider the set of all natural transformations $\Delta\ast\Rightarrow X$ denoted by $\text{Nat}(\Delta\ast,X)$. Now this set can inherit a subspace topology via the isomorphism (of sets) $$\text{Nat}(\Delta\ast,X)\cong\prod_{I\in\mJ}\Hom_\bold{Top}(\ast,X_I)\subset\prod_{I\in\mJ}X_I$$ There is in fact a canonical homeomorphism between the set of natural transformations and the limit of $X$. 

\begin{thm}{}{} Let $X:\mJ\to\bold{Top}$ be a diagram of spaces. Then there is a canonical homeomorphism $$\lim_\mJ X\cong\text{Nat}(\Delta\ast,X)$$
\end{thm}
Ref: Cubical diagrams

\begin{thm}{}{} Let $X:\mJ\to\bold{Top}$ be a diagram of spaces. Then there is a canonical homeomorphism $$\colim_\mJ X\cong\frac{\coprod_{I\in\mJ}X_I}{\sim}$$ where $x\in X_I\sim y\in X_J$ if and only if there exists $f:I\to J$ such that $X(f)(x)=y$. 
\end{thm}

\subsection{Homotopy Limits and Colimits}
Let $\mJ$ be a small diagram. Let $I\in\mJ$ and denote $\mJ/I$ to be the overcategory with the distinguished object $I$. We can turn it into a topological space by constructing its classifying space $$\mB(\mJ/I)=\abs{N(\mJ/I)}$$ which is the geometric realization of the nerve of $\mJ/I$. We aim to use the overcategory $\mJ/I$ to record homotopy information. Recall that the limit is canonically isomorphic to the equalizer of $$f,g:\prod_{J\in\Obj\mJ}X_J\to\prod_{(\alpha:J\to I)\in\mJ}X_I$$ as follows. 
\begin{itemize}
\item Define $f$ to be the unique map such that $\pi_I\circ f=\pi_{\text{cod}(\alpha)}$ where both $\pi$ are projections. 
\item Define $g$ to be the unique map such that $\pi_{\text{cod}(\alpha)}\circ g=F(\alpha)\circ\pi_{\text{dom}(\alpha)}$
\end{itemize}
We aim to replace each space $X_I$ by the space $\text{Map}(\mB(\mJ/I),X_I)$ (which is why we work with $\bold{CGWH}$). Indeed the hom space and $X_I$ are homotopy equivalent since $\mB(\mJ/I)$ is contractible. Notice that this is no longer a functor in $I$, but rather a bifunctor. This all works with a simplicial model category. 

\begin{defn}{Homotopy Limits}{} Let $\mC$ be a simplicial model category. Let $X:\mJ\to\mC$ be a diagram. Define two maps $$f,g:\prod_{J\in\Obj\mJ}\text{Map}(\mB(\mJ/J),X_J)\to\prod_{(\alpha:J\to I)\in\mJ}\text{Map}(\mB(\mJ/J),X_I)$$ as follows. 
\begin{itemize}
\item Define $f$ to be the unique map such that $$\pi_I\circ f=\text{Map}(\mB(\mJ/J),X(\alpha):X_J\to X_I)$$ for any $\alpha:J\to I$ a morphism in $\mJ$. 
\item Define $g$ to be the unique map such that $$\pi_I\circ g=\text{Map}\left(\mB\left(\mJ/X:\mJ/I\to\mJ/J\right),X_I\right)$$ for any $\alpha:J\to I$ a morphism in $\mJ$. 
\end{itemize}
Define the homotopy limit $\text{holim}X$ of $X$ to be the equalizer of the maps $$\text{holim}X=\text{Eq}(f,g)$$
\end{defn}

Let $\mJ$ be the small diagram consisting of two objects $A,B$ and two non-trivial morphisms $f,g:A\to B$. Let us illustrate the homotopy limit of $\mJ$. The slice category $\mJ/A$ consists only of a single object corresponding to the identity map $\text{id}_A:A\to A$. The slice category $\mJ/B$ consists of $3$ objects corresponding to $\text{id}_B$, $f$ and $g$. Non-trivial morphisms are given as follows: \\~\\
\adjustbox{scale=1.0,center}{\begin{tikzcd}
	f \\
	& {\text{id}_B} \\
	g
	\arrow[from=1-1, to=2-2]
	\arrow[from=3-1, to=2-2]
\end{tikzcd}}\\~\\
Passing to the classifying space, $\mB(\mJ/A)$ is just the one point space and $\mB(\mJ/B)$ becomes the interval $I$. 

Expanding things out show that $$f,g:\text{Map}(,A)\times\text{Map}(,B)$$ 

\begin{prp}{}{} Let $X:\mJ\to\bold{CGWH}$ be a diagram. Then there is a natural transformation $$\text{holim}_\mJ X\cong\text{Nat}(\mB(\mJ/-):\mJ\to\bold{CGWH},X)$$
\end{prp}

\begin{defn}{Homotopy Colimits}{} Let $\mC$ be a simplicial model category. Let $X:\mJ\to\mC$ be a diagram. Define two maps $$f,g:\coprod_{(\alpha:J\to I)\in\mJ}X_J\times\mB\left((I/\mJ)^\text{op}\right)\to\coprod_{J\in\Obj\mJ}X_J\times\mB\left((J/\mJ)^\text{op}\right)$$ as follows. 
\begin{itemize}
\item On each summand of the domain of $f$, define $f$ to be the map $$X(\alpha)\times\text{id}_{\mB\left((I/\mJ)^\text{op}\right)}:X_J\times\mB\left((I/\mJ)^\text{op}\right)\longrightarrow X_I\times\mB\left((I/\mJ)^\text{op}\right)$$ and then injecting the into the coproduct. 
\item On each summand of the domain of $g$, define $g$ to be the map $$\text{id}_{X_J}\times\mB(\mJ/X)^\text{op}:X_J\times\mB\left((I/\mJ)^\text{op}\right)\longrightarrow X_J\times\mB\left((J/\mJ)^\text{op}\right)$$ and then injecting into the coproduct. 
\end{itemize}
Define the homotopy colimit $\hocolim X$ of $X$ to be the coequalizer of the maps $$\hocolim X=\text{Coeq}(f,g)$$
\end{defn}

\pagebreak
\section{???}

\subsection{The Relative Point of View}
Up until this point, in algebraic topology we have asked questions relating to two spaces and tried to answer them. For instance, we can ask whether two spaces are homeomorphic, homotopy equivalent or weakly equivalent. We can also ask these questions in a relative setting, this involves considering maps of spaces as objects themselves, instead of just the spaces. 

\begin{defn}{Maps of Maps}{} Let $X,Y,A,B$ be spaces. Let $f:X\to Y$ and $g:A\to B$ be maps. A map from $f$ to $g$ is a pair of maps $(\alpha:X\to A,\beta:Y\to B)$ such that the following diagram commutes: \\~\\
\adjustbox{scale=1.0,center}{\begin{tikzcd}
	X & Y \\
	A & B
	\arrow["f", from=1-1, to=1-2]
	\arrow["\alpha"', from=1-1, to=2-1]
	\arrow["\beta", from=1-2, to=2-2]
	\arrow["g"', from=2-1, to=2-2]
\end{tikzcd}}\\~\\
\end{defn}

\begin{defn}{Homotopy from Maps to Maps}{} Let $X,Y,A,B$ be spaces. Let $f:X\to Y$ and $g:A\to B$ be maps. Let $(a,b)$ and $(c,d)$ be two maps from $f$ to $g$. We say that $(a,b)$ and $(c,d)$ are homotopic if there exists maps $H:X\times I\to Y$ and $K:A\times I\to B$ such that the following diagram \\~\\
\adjustbox{scale=1.0,center}{\begin{tikzcd}
	{X\times I} & {Y\times I} \\
	A & B
	\arrow["{f\times\text{id}_I}", from=1-1, to=1-2]
	\arrow["H"', from=1-1, to=2-1]
	\arrow["K", from=1-2, to=2-2]
	\arrow["g"', from=2-1, to=2-2]
\end{tikzcd}}\\~\\
commutes and the following are true. 
\begin{itemize}
\item $H$ is a homotopy from $a$ to $c$
\item $K$ is a homotopy from $b$ to $d$
\end{itemize}
\end{defn}

We now restrict to the point of view where maps are over a fixed space $Z$ (In other words we are considering objects of $\bold{Top}_Z$). A map of maps becomes the following data: Let $f:X\to Z$ and $g:Y\to Z$ be maps. A map from $f$ to $g$ is a map $h:X\to Y$ such that $g\circ h=f$. In other words the following diagram commutes: \\~\\
\adjustbox{scale=1.0,center}{\begin{tikzcd}
	X && Y \\
	& Z
	\arrow["h", from=1-1, to=1-3]
	\arrow["f"', from=1-1, to=2-2]
	\arrow["g", from=1-3, to=2-2]
\end{tikzcd}}\\~\\

A homotopy then becomes the following data: Let $f:X\to Z$ and $g:Y\to Z$ be two maps. Let $h,k:X\to Y$ be two maps from $f$ to $g$. We say that $h$ and $k$ are homotopic if there exists a homotopy $H:X\times I\to Y$ from $h$ to $k$ such that $H(-,t)$ for each $t$ is a map over $Z$. This means that the following diagram commutes: \\~\\
\adjustbox{scale=1.0,center}{\begin{tikzcd}
	{X\times I} && Y \\
	& Z
	\arrow["H", from=1-1, to=1-3]
	\arrow["{f\circ p_X}"', from=1-1, to=2-2]
	\arrow["g", from=1-3, to=2-2]
\end{tikzcd}}\\~\\

We can now discuss homotopy equivalences on these maps. 

\begin{defn}{Fiber Homotopy Equivalence}{} Let $X,Y,Z$ be spaces. Let $f:X\to Z$ and $g:Y\to Z$ be maps. We say that $f$ and $g$ are homotopy equivalent if there exists two maps $h:X\to Y$ and $k:Y\to X$ such that $k\circ h$ and $h\circ k$ are both homotopic to the identity over $Z$. 
\end{defn}

The reason that it is called a fiber homotopy equivalence is because it gives homotopy equivalences on fibers. 


\end{document}
