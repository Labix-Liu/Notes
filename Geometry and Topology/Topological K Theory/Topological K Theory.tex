\documentclass[a4paper]{article}

\input{C:/Users/liula/Desktop/Latex/Headers V1.2.tex}

\pagestyle{fancy}
\fancyhf{}
\rhead{Labix}
\lhead{Topological K Theory}
\rfoot{\thepage}

\title{Topological K Theory}

\author{Labix}

\date{\today}
\begin{document}
\maketitle
\begin{abstract}
\end{abstract}
\pagebreak
\tableofcontents

\pagebreak
\section{The K Group of a Space}
\subsection{Grothendieck Completions}
\begin{defn}{Grothendieck Completion}{} Let $A$ be an Abelian monoid. We say that a group $\mG(A)$ together with a monoid homomorphism $i:A\to\mG(A)$ is a Grothendieck completion of $A$ if the following universal property is satisfied. If $j:A\to H$ is another monoid homomorphism where $H$ is an abelian group, then there exists a unique group homomorphism $k:\mG(A)\to H$ such that the following diagram \\~\\
\adjustbox{scale=1.0,center}{\begin{tikzcd}
	A & {\mG(A)} \\
	& H
	\arrow["i", from=1-1, to=1-2]
	\arrow["j"', from=1-1, to=2-2]
	\arrow["{\exists!k}", dashed, from=1-2, to=2-2]
\end{tikzcd}}\\~\\
is commutative. 
\end{defn}

\begin{prp}{}{} Let $X$ be a space. Then $\text{Vect}^\R(X)$ and $\text{Vect}^\C(X)$ are both abelian monoids with the Whitney sum operator. They are moreover a commutative semiring under then tensor product operator. 
\end{prp}

\subsection{The K-Group of a Space}
\begin{thm}{}{} Let $X$ be a compact Hausdorff space. Then for any vector bundle $E\to X$ over $F=\R$ or $\C$, there exists a vector bundle $\widetilde{E}\to X$ over $F$ such that there is an isomorphism $$E\oplus\widetilde{E}\cong X\times F^n$$ to the trivial bundle. 
\end{thm}

\begin{defn}{The K-Group of a Space}{} Let $X$ be a space. Define the real and complex $K$ group of $X$ respectively to be the Grothendieck completion $$KO(X)=\mG(\text{Vect}^\R(X))\;\;\;\;\text{ and }\;\;\;\;KU(X)=\mG(\text{Vect}^\C(X))$$
\end{defn}

\begin{lmm}{}{} Let $X$ be a space. Then $KO(X)$ and $KU(X)$ are both commutative rings with identity. 
\end{lmm}

\begin{defn}{The K Functor}{} Define the $K$ functors $$KO,KU:\bold{Top}\to\bold{CRing}$$ as follows. 
\begin{itemize}
\item For each space $X$, $KO(X)$ is the real $K$-group of $X$ and $KU(X)$ is the complex $K$-group of $X$
\item For each map $f:X\to Y$, $KO(f):KO(Y)\to KO(X)$ is the ring homomorphism that sends each isomorphism class of vector bundle $[E]\in KO(Y)$ to the pullback bundle $[f^\ast(E)]\in KO(X)$. This is similar for $KU(f):KU(Y)\to KU(X)$. 
\end{itemize}
\end{defn}

We use $K:\bold{Top}\to\bold{CRing}$ to mean either the real $K$ groups or the complex $K$ groups when no distinction is needed. 

\subsection{Reduced K-Theory}
\begin{defn}{Reduced K-Theory}{} Let $X$ be a space. Define the reduced $K$-theory of $X$ to be the kernel $$\widetilde{KO}(X)=\ker(KO(X)\to KO(\ast))\;\;\;\;\text{ and }\;\;\;\;\widetilde{KU}(X)=\ker(KU(X)\to KU(\ast))$$
\end{defn}

Similarly, we use $\widetilde{K}:\bold{Top}\to\bold{Ab}$ to mean either the reduced real $K$ groups or the reduced complex $K$ groups when no distinction is needed. \\~\\

The universal property of the kernel turns reduced $K$-theory into a functorial construction. 

\begin{defn}{Reduced K Functor}{} Define the $K$ functors $$\widetilde{KO},\widetilde{KU}:\bold{Top}\to\bold{Ab}$$ as follows. 
\begin{itemize}
\item For each space $X$, $\widetilde{KO}(X)$ is the reduced real $K$-group of $X$ and $\widetilde{KU}(X)$ is the reduced complex $K$-group of $X$
\item For each map $f:X\to Y$, $\widetilde{KO}(f):\widetilde{KO}(Y)\to\widetilde{KO}(X)$ is the ring homomorphism that sends each isomorphism class of vector bundle $[E]\in \widetilde{KO}(Y)$ to the pullback bundle $[f^\ast(E)]\in\widetilde{KO}(X)$. This is similar for $\widetilde{KU}(f):\widetilde{KU}(Y)\to\widetilde{KU}(X)$. 
\end{itemize}
\end{defn}

\begin{thm}{}{} Let $X$ be a compact Hausdorff space. Then the natural homomorphism $K(X)\to\widetilde{K}(X)$ is surjective with kernel $\Z$. In particular, this gives an isomorphism $$K(X)\cong\widetilde{K}(X)\oplus\Z$$ A similar 
\end{thm}

\begin{defn}{Reduced Equivalence}{} Let $X$ be a space. Let $V\to X$ and $W\to X$ be two vector bundles over $X$. We say that $V\sim_\text{red}W$ if $V\oplus T^m\cong W\oplus T^n$ for some $m,n\in\N$ and $T$ the trivial bundle. 
\end{defn}

\begin{prp}{}{} Let $X$ be a space. Then $\text{Vect}^\R/\sim_\text{red}$ and $\text{Vect}^\C/\sim_\text{red}$ form abelian groups respectively with the Whitney sum operation. 
\end{prp}

\begin{thm}{}{} Let $X$ be a space. Then there is are natural isomorphisms $$KO(X)\cong\frac{\text{Vect}^\R}{\sim_\text{red}}\;\;\;\;\text{ and }\;\;\;\;KU(X)\cong\frac{\text{Vect}^\C}{\sim_\text{red}}$$
\end{thm}

\begin{prp}{}{} Let $X$ be a compact Hausdorff space and let $A\subseteq X$ be a closed subspace. Then the inclusion map $i:A\to X$ and the projection map $q:X\to X/A$ induces an exact sequence \\~\\
\adjustbox{scale=1.0,center}{\begin{tikzcd}
	{\widetilde{K}(X/A)} & {\widetilde{K}(X)} & {\widetilde{K}(A)}
	\arrow["{q^\ast}", from=1-1, to=1-2]
	\arrow["{i^\ast}", from=1-2, to=1-3]
\end{tikzcd}}\\~\\
\end{prp}

\subsection{Relative K-Theory}
\begin{defn}{}{} Let $X$ be a space and let $A\subseteq X$ be a closed subspace. Define the relative $K$ theory of the pair $(X,A)$ by $$K(X,A)=\widetilde{K}(X/A)$$
\end{defn}

\pagebreak
\section{Functorial Properties of the K Groups}
\subsection{Representability of the Reduced K Functor}
They key point of being compact Hausdorff is displayed as follows. 

\begin{thm}{Stabilization Theorem}{} Let $X$ be a compact space. Then there is a natural isomorphism $$\lim_{\N}\text{Vect}_n^\R(X)\cong\widetilde{KO}(X)$$ induced by the direct limit of the maps $\text{Vect}_n^\R(X)$. Similarly, there is a natural isomorphism $$\lim_{\N}\text{Vect}_n^\C(X)\cong\widetilde{KU}(X)$$
\end{thm}

Now $\text{Vect}_n^\R(-)$ is representable by $O(n)$. Commuting with the direct limit gives the following theorem. 

\begin{thm}{}{} The real and complex $K:\bold{CH}\to\bold{CRing}$ functors defined on the full subcategory $\bold{CH}$ of compact Hausdorff spaces are representable: $$KO(-)\cong[-,BO\times\Z]\;\;\;\;\text{ and }\;\;\;\;KU(-)\cong[-,BU\times\Z]$$ Similarly, the real and complex reduced $\widetilde{K}:\bold{CH}\to\bold{CRing}$ functors are representable: $$\widetilde{KO}(-)\cong[-,BO]\;\;\;\;\text{ and }\;\;\;\;\widetilde{KU}(-)\cong[-,BU]$$
\end{thm}

\subsection{K Theory as a Cohomology Theory}
\begin{thm}{Homotopy Invariance}{} If $X$ and $Y$ are paracompact space such that $f:X\to Y$ is a homotopy equivalence, then there is an isomorphism $$K(f):K(Y)\overset{\cong}{\longrightarrow}K(X)$$ given by the induced map. 
\end{thm}

\begin{thm}{Long Exact Sequence}{} Let $X$ be a compact Hausdorff space. Let $A\subseteq X$ be a closed subspace. Then there is a long exact sequence in reduced $K$-theory: \\~\\
\adjustbox{scale=0.85,center}{\begin{tikzcd}
	\cdots & {\widetilde{KU}(\Sigma(X/A))} & {\widetilde{KU}(\Sigma X)} & {\widetilde{KU}(\Sigma A)} & {\widetilde{KU}(X/A)} & {\widetilde{KU}(X)} & {\widetilde{KU}(A)} & \cdots
	\arrow[from=1-1, to=1-2]
	\arrow[from=1-2, to=1-3]
	\arrow[from=1-3, to=1-4]
	\arrow[from=1-4, to=1-5]
	\arrow[from=1-5, to=1-6]
	\arrow[from=1-6, to=1-7]
	\arrow[from=1-7, to=1-8]
\end{tikzcd}}\\~\\
\end{thm}

\subsection{External Product}

\pagebreak
\section{Bott Periodicity}
\subsection{Fundamental Product Theorem}
\subsection{The Bott Periodicity Theorem}
\begin{thm}{The Bott Periodicity Theorem}{} Let $X$ be a space. Let $L$ be a line bundle over $X$. 
\end{thm}











\end{document}
