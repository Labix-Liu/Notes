\documentclass[a4paper]{article}

\usepackage{mathtools}
\usepackage{amsfonts}
\usepackage{amsmath}
\usepackage{amsthm}
\usepackage[a4paper, total={6in, 8in}]{geometry}
\usepackage[english]{babel}
\usepackage[utf8]{inputenc}
\usepackage{fancyhdr}
\usepackage[english]{babel}
\usepackage[utf8]{inputenc}
\usepackage{graphicx}
\usepackage{physics}
\usepackage[colorinlistoftodos]{todonotes}

\DeclarePairedDelimiter\ceil{\lceil}{\rceil}
\DeclarePairedDelimiter\floor{\lfloor}{\rfloor}

\DeclareMathOperator{\adj}{adj}
\DeclareMathOperator{\im}{im}
\DeclareMathOperator{\nullity}{nullity}
\DeclareMathOperator{\sign}{sign}
\DeclareMathOperator\dom{dom}
\DeclareMathOperator\lcm{lcm}
\DeclareMathOperator{\ran}{ran}
\DeclareMathOperator{\ext}{Ext}
\DeclareMathOperator{\dist}{dist}
\DeclareMathOperator{\diam}{diam}
\DeclareMathOperator{\aut}{Aut}
\DeclareMathOperator{\inn}{Inn}
\DeclareMathOperator{\syl}{Syl}
\DeclareMathOperator{\homo}{Hom}

\newcommand{\C}{\mathbb{C}}
\newcommand{\CP}{\mathbb{CP}}
\newcommand{\GG}{\mathbb{G}}
\newcommand{\F}{\mathbb{F}}
\newcommand{\N}{\mathbb{N}}
\newcommand{\Q}{\mathbb{Q}}
\newcommand{\R}{\mathbb{R}}
\newcommand{\RP}{\mathbb{RP}}
\newcommand{\T}{\mathbb{T}}
\newcommand{\Z}{\mathbb{Z}}
\renewcommand{\H}{\mathbb{H}}

\theoremstyle{definition}
\newtheorem{defn}{Definition}[subsection]
\newtheorem{axm}[defn]{Axiom}
\newtheorem{thm}[defn]{Theorem}
\newtheorem{prp}[defn]{Proposition}
\newtheorem{lmm}[defn]{Lemma}
\newtheorem{crl}[defn]{Corollary}

\raggedright

\pagestyle{fancy}
\fancyhf{}
\rhead{Labix}
\lhead{Algebraic Curves}
\rfoot{\thepage}

\title{Algebraic Curves}

\author{Labix}

\date{\today}
\begin{document}
\maketitle
\begin{abstract}
\end{abstract}
\pagebreak
\tableofcontents
\pagebreak

\section{Algebraic Curves in Classical Algebraic Geometry}
\subsection{Basic Properties of Curves}
\begin{defn}{Curves}{} Let $k$ be a field. Let $X$ be a variety over $k$. We say that $X$ is a curve if $\dim(X)=1$. 
\end{defn}

\begin{eg}{}{}\\
Consider the projective curve $C=\V(x^2+y^2-z^2)\subset\Prj_\C^2$. Let $p=[p_0:p_1:p_2]$ be a point on the curve. Then $\mO_{C,p}$ is a DVR. 
\begin{proof}\\
If $p_2\neq 0$, then $p\in U_2$. Under the affine chart $(U_2,\varphi_2)$, we find that $C_2=\varphi_2(C\cap U_2)=\V(x^2+y^2-1)$. The corresponding coordinate ring is given by $\frac{\C[x,y]}{(x^2+y^2-1)}$. The formula for the local ring in the affine case gives $$\mO_{C,p}\cong\left(\frac{\C[x,y]}{(x^2+y^2-1)}\right)_{m_{(p_0/p_2,p_1/p_2)}}$$ Recall that the unique maximal ideal of the local ring is given as the $\mO_{X,p}$-module $m_p=\{f\in\C[C_2]\;|\;f(p_0/p_2,p_1/p_2)=0\}$, which under the nullstellensatz is the maximal ideal corresponding to the point $(p_0/p_2,p_1/p_2)$ and is given by $m_p=(x-r,y-s)$ where $r=p_0/p_1$ and $s=p_0/p_2$. By Nakayama's lemma, since $x-r,y-s$ generate $m_p$ we know that $x-r+m_p^2,y-s+m_p^2$ span the vector space $m_p/m_p^2$ over $\mO_{X,p}/m_p$. I claim that they are linearly dependent. This mean that I want to find $f+m_p^2$ and $g+m_p^2$ in $\mO_{X,p}/m_p$ that are non-trivial, and that $(x-r)f+(y-s)g+m_p^2=m_p^2$. This means that we want to find $f,g\in\mO_{X,p}\setminus m_p$ such that $(x-r)f+(y-s)g\in m_p^2$. Choose $f=x+r$ and $g=y+s$ to get $$(x-r)(x+r)+(y-s)(y+s)=x^2-r^2+y^2-s^2=1-1=0$$ since $(r,s)$ lie on the curve. Moreover, $x+r,y+s\mO_{X,p}\setminus m_p$ since evaluating at $(r,s)$ at the functions are non-zero. This verifies that $\mO_{X,p}$ is a regular local ring of dimension $1$, hence is a DVR. 
\end{proof}
\end{eg}

\begin{lmm}{}{} Let $k$ be an algebraically closed field. Let $C$ be an irreducible curve over $k$. Let $p\in C$ be a non-singular point. Then $\mO_{C,p}$ is a DVR. Moreover, the valuation is given by the degree of the regular function. 
\begin{proof}
Since $p$ is non-singular, by definition $\mO_{C,p}$ is a regular local ring. Moreover, we know that $1=\dim(C)=\dim(\mO_{C,p})$ so that $\mO_{C,p}$ has Krull dimension $1$. By the equivalent characterization of DVR we conclude. 
\end{proof}
\end{lmm}

We denote the valuation map by $v_p:\text{Frac}(\mO_{C,p})\to\Z$. 

\begin{prp}{}{} Let $C$ be an affine irreducible curve over $\C$. Then $C$ is smooth if and only if $C$ is a normal variety. 
\end{prp}

Recall that by taking the integral closure of the coordinate ring $k[C]$ of an irreducible affine curve $C\subseteq\A^n$, we obtain a corresponding variety $\widetilde{C}$ called the normalization of $C$. \\

Moreover, within each birational class of irreducible curves, one can discuss the isomorphism classes of curves. 

\begin{crl}{}{} Let $k$ be an algebraically closed field. Let $C\subseteq\A_k^n$ be an irreducible affine curve over $k$. Then the normalization $\widetilde{C}$ is smooth. 
\end{crl}

\begin{prp}{}{}\\
Let $k$ be an algebraically closed field. Then there is an equivalence of categories $$\substack{\text{Smooth projective curves}\\\text{over }k\text{ with}\\\text{dominant morphisms}}\cong\substack{\text{Function fields over }k\\\text{with morphisms of }k\text{-algebra}}$$ given by the contravariant functor $C\mapsto K(C)$. 
\end{prp}

\subsection{Morphisms between Curves}
\begin{prp}{}{}\\
Let $k$ be an algebraically closed field. Let $C$ be a smooth curve over $k$. Let $X\subseteq\Prj^n$ be a projective variety. Let $\phi:U\subseteq C\to X$ be a rational map. Then there exists a regular map $$\overline{\phi}:C\to X$$ such that $\overline{\phi}|_U=\phi|_U$ for some dense subset $U\subseteq C$. 
\begin{proof}\\
Notice that it suffices to extend $\phi$ to a map $\overline{\phi}:C\to\Prj^n$ because in this case we have $$\overline{\phi}(C)=\overline{\phi}(\overline{U})\subseteq\overline{\overline{\phi}(U)}=\overline{\phi(U)}\subseteq\overline{X}=X$$ anyway. Since $C$ is a curve and $U\subseteq C$ is open, $C\setminus U=\{p_1,\dots,p_n\}$ consists of finitely many points. We induct on $p_1,\dots,p_n$ by starting out with the map $\phi$. \\

Assume that we are given the map $\overline{\phi}$ that is well defined on $U\cup\{p_1,\dots,p_k\}$. Since $C$ is smooth, $\mO_{C,p_{k+1}}$ is a DVR. Let $\pi\in\mO_{C,p_{k+1}}$ be a uniformizer. Suppose that $\overline{\phi}$ is given in coordinates as $[f_0:\cdots:f_n]$ for $f_0,\dots,f_n\in K(C)$. Let $a=\min\{\text{val}_{p_{k+1}}(f_i)\;|\;1\leq i\leq n\}$. Then since our coordinates are projective, we have that $$\overline{\phi}(x)=[f_0(x):\cdots:f_n(x)]=[\pi^{-a}(x)f_0(x):\cdots\pi^{-a}(x)f_n(x)]$$ for all $x\in U$. Moreover, for each $i$, we have $\text{val}_{p_{k+1}}(\pi^{-a}f_i)\geq 0$ and so our new map is regular on $p_{k+1}$. By induction we are done. 
\end{proof}
\end{prp}

\begin{eg}{}{}\\
Let $k$ be a field. Consider the rational map $\phi:\Prj_k^1\to\Prj_k^2$ defined by $$\phi([s:t])=\left[\frac{s+t}{s}:\frac{st+t^2}{(s-t)^2}:\frac{s^2-t^2}{t^2}\right]$$ This map can be extended to the morphism of varieties $$\overline{\phi}([s:t])=[t^2(s-t)^2(s+t):t^3s(s+t):(s-t)^2s(s^2-t^2)]$$
\begin{proof}\\
Notice that the degrees of each component is constant. Moreover, they simultaneously vanish if and only if $s=t=0$. Finally, by dividing each coordinate with $\frac{s(s-t)^2t^2}{s+t}$ we obtain the original rational map, and hence this morphism of varieties agree with the rational map. 
\end{proof}
\end{eg}

Note that the extended map is obtained by clearing denominators, because by clearing denominators we are multiplying each coordinate with the same map, and because the coordinates are projective this does not change the projective coordinates. 

\begin{prp}{}{}\\
Let $k$ be an algebraically closed field. Let $X,Y$ be smooth irreducible projective varieties over $k$. Let $\phi:X\to Y$ be a regular map. Then exactly one of the following holds. 
\begin{itemize}
\item $\phi$ is constant. 
\item $\phi(X)=Y$ and $\phi$ is a finite morphism. 
\end{itemize}
\begin{proof}
If $\phi$ is non-constant, then $\phi(X)$ is closed and connected and so $\phi(X)=Y$. In particular, $\phi$ is a dominant rational map and induces a $k$-algebra homomorphism $k(Y)\to k(X)$. Since $k(X)$ and $k(Y)$ both have transcendence degree $1$, $k(X)$ is an algebraic extension of $k(Y)$. \\

Let $W\subseteq Y$ be an open affine subset. Let $\overline{k[W]}$ be the integral closure of $k[W]$ in $k(X)$. Then $\overline{k[W]}$ is a finitely generated $k[W]$-module and hence corresponds to an irreducible affine curve $V$. I claim that $k(X)=k(V)=\text{Frac}(\overline{k[W]})$. We already know that $k[V]\subseteq k(X)$. Now let $f/g\in k(X)$. Since $k(X)$ is algebraic over $k(Y)$, there exists $h\in k(Y)[t]$ such that $h(f/g)=0$. Clearing denominators, there exists $b_0,\dots,b_n\in k[W]$ such that $$b_n(f/g)^n+b_{n-1}(f/g)^{n-1}+\dots+b_1(f/g)+b_0=0$$ Then we have $$(b_nf/g)^n+b_{n-1}(b_nf/g)^{n-1}+b_{n-2}b_{n-1}(b_{n-1}f/g)^{n-2}+\dots+b_1b_{n-1}^{n-2}(b_{n-1}f/g)+b_0b_{n-1}=0$$ Hence $b_nf/g$ is integral over $k[W]$. Hence $b_nf/g\in\overline{k[W]}$ and so $f/g\in\text{Frac}(\overline{k[W]})$. \\

The map $k[W]\to k[V]$ is induced by $\phi^\ast:k(Y)\to k(X)$. Hence the associated map $V\to W$ is also induced by $\phi$. Thus $V$ is an affine open subset of $X$. Hence $V\subseteq\phi^{-1}(W)$. Now I claim that $V=\phi^{-1}(W)$. Consider the following commutative diagram: \\
\adjustbox{scale=1.0,center}{\begin{tikzcd}
	{k[\phi^{-1}(W)]} && {k[V]} \\
	& {k[W]}
	\arrow["{\text{incl}.^\ast}", from=1-1, to=1-3]
	\arrow["{\phi^\ast}", from=2-2, to=1-1]
	\arrow["{\phi^\ast}"', from=2-2, to=1-3]
\end{tikzcd}}\\
By definition, $k[V]$ is a subring of $k(X)$. Hence we can take integral closure of all three rings in $k(X)$ to obtain the diagram: \\
\adjustbox{scale=1.0,center}{\begin{tikzcd}
	{\overline{k[\phi^{-1}(W)]}} && {k[V]} \\
	& {k[V]}
	\arrow["{\text{incl}.^\ast}", from=1-1, to=1-3]
	\arrow["{\phi^\ast}", from=2-2, to=1-1]
	\arrow["{\text{id}_{k[V]}}"', from=2-2, to=1-3]
\end{tikzcd}}\\
where the lower right arrow is the identity because $k[V]=\overline{k[W]}$ and the map $V\to W$ is induced by the inclusion map $k[W]\hookrightarrow\overline{k[W]}$. If $k[\phi^{-1}(W)]$ is integrally closed, then the above diagram induces an inverse map to the inclusion $V\hookrightarrow\phi^{-1}(W)$ so that $V=\phi^{-1}(W)$. ????\\

Since $k[V]=\overline{k[W]}$ is the integral closure of $k[W]$ in $k(X)$, $\overline{k[W]}$ is a finitely generated $k[W]$-module. Hence $k[V]=k[\phi^{-1}(W)]$ is a finitely generated $k[W]$-module. Hence $\phi$ is a finite morphism. 


\end{proof}
\end{prp}

\begin{prp}{}{}\\
Let $k$ be an algebraically closed field. Let $X,Y$ be smooth irreducible projective curves over $k$. Let $\phi:X\to Y$ be a rational map. If $\phi$ is birational, then $\phi$ is an isomorphism of varieties. 
\begin{proof}
By 1.2.1, we can extend $\phi$ to a regular map $\overline{\phi}:X\to Y$. Similarly we can extend $\phi^{-1}$ to a regular map $\overline{\phi^{-1}}:Y\to X$. Then $\overline{\phi^{-1}}\circ\overline{\phi}:X\to X$ is the identity map since $\phi^{-1}\circ\phi$ agrees with $\text{id}_X$ on a dense open subset of $X$. 
\end{proof}
\end{prp}

\begin{prp}{}{} Let $k$ be an algebraically closed field. Let $C$ be an irreducible curve over $k$. Then $C$ is birational to a unique (up to isomorphism) projective smooth irreducible curve. 
\begin{proof}
We know that $C$ is birational to its normalization, which is a smooth curve. 
\end{proof}
\end{prp}

\begin{prp}{}{} Let $k$ be an algebraically closed field. Let $C$ be an irreducible curve over $k$. Suppose that $C$ is birational to $\Prj^1$ (rational) and $C$ is not isomorphic to $\Prj^1$. Then the following are true. 
\begin{itemize}
\item $C$ is isomorphic to an open subset of $\A_k^1$. 
\item $C$ is an affine curve. 
\item $k[C]$ is a UFD, 
\end{itemize}
\end{prp}

\subsection{Differential Forms on Curves}
\begin{prp}{}{} Let $C$ be a smooth irreducible curve over $\C$. Then we have $$\Omega_{\C(C)/C}^1=\left(\Omega_C^1\right)_{(0)}$$ is a one dimensional $\C(C)$-vector space. 
\end{prp}

\begin{defn}{Valuation of Differential $1$-Forms}{} Let $C$ be a smooth irreducible curve over $\C$. Let $p\in C$. Let $\omega\in\Omega_{\C(C)/\C}^1$ be a differential $1$-form of $C$. Define the valuation of $\omega$ at $p$ as follows. Choose a uniformizer $\pi\in\mO_{C,p}$. Write $\omega=fd\pi$ for $f\in\C(C)$. Then define the valuation as $$\text{val}_p(\omega)=\text{val}_p(f)$$
\end{defn}

\pagebreak
\section{Classical Divisors on Curves}
\subsection{The Pullback Map of Divisors}
\begin{defn}{Pullback Map of Divisors}{}\\
Let $k$ be an algebraically closed field. Let $C$ be a smooth irreducible projective curve. Let $X$ be a smooth irreducible projective variety. Let $\phi:C\to X$ be a regular map. Define the induced pullback map $\phi^\ast:\text{Div}(X)\to\text{Div}(C)$ on generators by $$\phi^\ast(H)=\sum_{p\in C}\text{val}_p(\phi^\ast(g))\cdot p$$ where $g$ is a generator of $\I(H)\mO_{Y,\phi(p)}$. 
\end{defn}

When $X$ is also a curve, we have $$\phi^\ast(q)=\sum_{p\in\phi^{-1}(q)}e_\phi(p)\cdot p=\sum_{p\in\phi^{-1}(q)}\text{val}_p(\phi^\ast(\pi))\cdot p$$ where $\pi$ is the uniformizer of $\mO_{Y,q}$. 

\begin{prp}{}{} Let $k$ be an algebraically closed field. Let $X,Y$ be smooth irreducible projective curves over $k$. Let $\phi:X\to Y$ be a non-constant regular map. Then we have $$\deg(\phi^\ast(D))=\deg(\phi)\deg(D)$$ for any $D\in\text{Div}(Y)$. 
\end{prp}

\begin{prp}{}{} Let $k$ be an algebraically closed field. Let $X,Y$ be smooth irreducible projective curves over $k$. Let $\phi:X\to Y$ be a non-constant regular map. Then $\phi(\text{Prin}(Y))\subseteq\text{Prin}(X)$. 
\end{prp}

\begin{prp}{}{} Let $k$ be an algebraically closed field. Let $C$ be an irreducible smooth curve over $k$. Then we have $$\text{Prin}(C)\leq\ker(\deg)$$
\end{prp}

\begin{defn}{Induced Map of Divisor Class Groups}{} Let $k$ be an algebraically closed field. Let $X,Y$ be smooth irreducible projective curves over $k$. Let $\phi:X\to Y$ be a non-constant regular map. Define the induced map of divisor class groups $\phi^\ast:\text{Cl}(Y)\to\text{Cl}(X)$ by $$\phi^\ast([D])=[\phi^\ast(D)]$$
\end{defn}

\subsection{The Linear System of Divisors}
Let $k$ be an algebraically closed field. Let $C$ be a smooth irreducible projective curve over $k$. Let $D\in\text{Div}(C)$ be a divisor. Recall that $D=\sum_{i=1}^rk_ip_i$ is effective if $k_i\geq0$ for all $i$. 

\begin{defn}{The Linear System of Divisors}{}\\
Let $k$ be an algebraically closed field. Let $X$ be a smooth irreducible projective curve over $k$. Let $D\in\text{Div}(X)$ be a divisor. Define the linear system of $D$ to be $$\mL(D)=\{0\}\cup\{f\in K(X)\;|\;D+\text{div}(f)\text{ is effective }\}\subseteq K(X)$$
\end{defn}

In other words, $\mL(D)$ consists rational functions in which the only poles allowed are governed by $D$. 

\begin{lmm}{}{} Let $k$ be an algebraically closed field. Let $X$ be a smooth irreducible projective curve over $k$. Let $D\in\text{Div}(X)$ be a divisor. Then $\mL(D)$ is a vector space over $k$. 
\end{lmm}

\begin{prp}{}{}\\
Let $k$ be an algebraically closed field. Let $X$ be a smooth irreducible projective curve over $k$. Let $D,D'\in\text{Div}(X)$ be divisors. If $D\sim D'$ are linearly equivalent, then we have $$\dim_k(\mL(D))=\dim_k(\mL(D'))$$
\begin{proof}\\
Since $D$ and $D'$ are linearly equivalent, there exists $g\in k(X)$ such that $D-D'=\text{div}(g)$. Define a map $\mL(D)\to\mL(D')$ by $f\mapsto fg$. This map is well defined since $D+\text{div}(f)$ is effective if and only if $D'+\text{div}(f)+\text{div}(g)$ is effective if and only if $D'+\text{div}(fg)$ is effective. Clearly it is $k$-linear. Moreover, its inverse is given by $h\mapsto h/g$. Hence $\mL(D)$ and $\mL(D')$ has the same dimension. 
\end{proof}
\end{prp}

\begin{prp}{}{}\\
Let $k$ be an algebraically closed field. Let $X$ be a smooth irreducible projective curve over $k$. Let $D\in\text{Div}(X)$ be a divisor. Then the following are true. 
\begin{itemize}
\item If $\deg(D)<0$, then we have $$\dim_k(\mL(D))=0$$
\item If $\deg(D)=0$, then we have $$\dim_k(\mL(D))=\begin{cases}
0 & \text{ if }D\not\sim0\\
1 & \text{ if }D\sim 0
\end{cases}$$
\end{itemize}
\begin{proof}\\
Suppose that $\deg(D)<0$. For any $f\in k(X)$, we have $\deg(D+\text{div}(f))=\deg(D)+\deg(\text{div}(f))=\deg(D)<0$. Hence $D+\text{div}(f)$ can never be effective. \\

Now suppose that $\deg(D)=0$. For any $f\in k(X)$, we have $\deg(D+\text{div}(f))=\deg(D)+\deg(\text{div}(f))=0$ and so $f\in\mL(D)$ if and only if $D+\text{div}(f)$ is effective if and only if $D+\text{div}(f)=0$ if and only if $\text{div}(1/f)=D$. If $D\not\sim0$ then $\text{div}(1/f)=D$ cannot happen by definition. Hence $\mL(D)=\{0\}$ in this case. If $D\sim 0$, then there exists $g\in k(X)$ such that $D=\text{div}(g)$. Then $\text{div}(1/f)=\text{div}(g)$ if and only if $\text{div}(fg)=0$ if and only if $fg$ is regular and does not vanish on $X$. But since $X$ is projective, the only regular and non-vanishing rational function are the constants. Hence $fg=c$ implies $f=c/g$ for some $c\in k$. Hence $\mL(D)=k\langle 1/g\rangle$ and so the dimension in this case is $1$. 
\end{proof}
\end{prp}

\begin{lmm}{}{}\\
Let $k$ be an algebraically closed field. Let $X$ be a smooth irreducible projective curve over $k$. Let $D\in\text{Div}(X)$ be a divisor. Let $p\in X$. Then we have $$0\leq\dim_k(\mL(D))-\dim_k(\mL(D-p))\leq 1$$
\begin{proof}\\
We have $\mL(D-p)\subseteq\mL(D)$ since for $f\in\mL(D-p)$, we have that $D-p+\text{div}(f)$ is effective and so $D+\text{div}(f)$ must be effective. Let $n_p$ be the coefficient of $p$ in $D$. Let $\pi$ be a uniformizer of $\mO_{X,p}$. Since $D+\text{div}(f)$ is effective, we must have $\text{val}_p(f\pi^{n_p})\geq 0$. Consider the map $\phi:\mL(D)\to k$ defined by $\phi(f)=(f\pi^{n_p})(p)$. Now we have that $f\in\ker(\phi)$ if and only if $\text{val}_p(f\pi^{n_p})>0$ if and only if $\text{val}_p(f)+n_p-1\geq 0$ if and only if $f\in\mL(D-p)$. Hence $\ker(\phi)=\mL(D-p)$. By rank nullity theorem, we have $\dim_k(\mL(D))=\dim_k(\mL(D-p))+\dim(\im(\phi))$. Since $\dim(\im(\phi))$ is either $0$ or $1$, we must have either $\dim_k(\mL(D))=\dim_k(\mL(D-p))$ or $\dim_k(\mL(D))=\dim_k(\mL(D-p))+1$. 
\end{proof}
\end{lmm}

\begin{prp}{}{}\\
Let $k$ be an algebraically closed field. Let $X$ be a smooth irreducible projective curve over $k$. Let $D\in\text{Div}(X)$ be a divisor. If $\deg(D)\geq 0$, then we have $$\dim_k(\mL(D))\leq\deg(D)+1$$
\begin{proof}\\
The results follow by induction and by applying the above lemma. 
\end{proof}
\end{prp}

\subsection{The Canonical Divisor for Curves}
\begin{defn}{Divisors of Differential Forms}{}\\
Let $C$ be a smooth irreducible projective curve over $\C$. Let $p\in C$. Let $\omega\in\Omega_{\C(C)/\C}^1$ be a differential $1$-form of $C$. Define the divisor of $\omega$ by $$\text{div}(\omega)=\sum_{p\in C}\text{val}_p(\omega)\cdot p\in\text{Div}(C)$$
\end{defn}

\begin{eg}{}{}\\
Let $C=\V^H(x^2+y^2-z^2)\subseteq\Prj_\C^2$. Let $\omega=\frac{x+z}{z}d\left(\frac{x}{z}\right)\in\Omega_{\C(C)/\C}^1$ be a one-form. Then we have $$\text{div}(\omega)=[1:0:1]+3[-1:0:1]-3[i:1:0]-3[-i:1:0]$$
\begin{proof}\\
We compute the coefficients of each point on each affine chart. On the chart $U_2$, $\omega$ is given by $x+1d(x)$. For any point $p=(p_0,p_1)\in C\cap U_2=\V(x^2+y^2-1)$, I claim that $x-p_0$ is a uniformizer of $\mO_{C\cap U_2,p}$ if and only if $p_1\neq 0$. Indeed, we know that $m_p\mO_{C\cap U_2,p}$ is generated by $x-p_0$ and $y-p_1$. Hence $[x-p_0]$ and $[y-p_1]$ spans $m_p/m_p^2$. If $p_1\neq 0$, then $y+p_1\in\mO_{C\cap U_2,p}$ and so $$y-p_1=\frac{y^2-p_1^2}{y+p_1}=\frac{1-x^2-1+p_0^2}{y+p_1}=-\frac{x+p_0}{y+p_1}(x-p_0)$$ Hence $[x-p_0]$ generates $m_p/m_p^2$ and so $x-p_0$ generates $m_p$ by a corollary of Nakayama's lemma. Conversely, if $p_1=0$, then we have $p_0=\pm1$ and $x\mp1\in\mO_{C\cap U_2,p}$. Then we have $$x\pm1=\frac{x^2-1}{x\mp1}=\frac{1}{x\mp1}y^2$$ so that $[x\pm 1]$ is the $0$ vector in $m_p/m_p^2$. Hence $x\pm 1$ cannot generate $m_p\mO_{C\cap U_2,p}$. So there are now two cases: \\

Case 1: $p_1\neq 0$. \\
Then $x-p_0$ is a uniformizer for $\mO_{C\cap U_2,p}$ and $d(x-p_0)=d(x)$. Then we have $\text{val}_p(\omega)=\text{val}_p(x+1)$ and this is non-zero if and only if $p_0=-1$, but since $p_1\neq 0$, no points in this case satisfy the criteria. \\

Case 2: $p_1=0$. \\
Then $y$ is a uniformizer. From $x^2+y^2-1=0$ in $\C[C]$, we infer that $d(x^2+y^2-1)=0$ in $\Omega_C^1$ and hence $\Omega_{\C(C)/\C}^1$ since the module of Kahler differential commutes with quotients and localizations. The condition then becomes $2xdx+2ydy=0$, or that $dx=-\frac{y}{x}dy$. Hence $\omega$ in this open chart can be written as $x+1dx=-\frac{(x+1)y}{x}dy$. Hence $$\text{val}_p(\omega)=\text{val}_p\left(-\frac{(x+1)y}{x}\right)$$ If $p_1=1$, then $x+1$ is a unit in $\mO_{C\cap U_2,p}$, hence $\text{val}_p\left(-\frac{x+1}{x}y\right)=1$. Hence the coefficient of $[1:0:1]$ is $1$. If $p_1=-1$, then $x-1$ is a unit in $\mO_{C\cap U_2,p}$, so we have $$\text{val}_p\left(-\frac{x+1}{x}y\right)=\text{val}_p\left(-\frac{x^2-1}{x(x-1)}y\right)=\text{val}_p\left(-\frac{-y^2}{x(x-1)}y\right)=\text{val}_p\left(\frac{1}{x(x-1)}y^3\right)=3$$ Hence the coefficient of $[-1:0:1]$ is $3$. \\

Notice that $C\setminus U_2=\V^H(x^2+y^2)\cap\V^H(z)=\{[i:1:0],[-i:1:0]\}$. It suffices to consider the coefficients of this two points, and they both lie in $U_1$. Now $C\cap U_1=\V(x^2+1-z^2)$. The $1$-form $\omega$ does not change under the chart $U_1$. Similarly as the above chart, the relation $x^2+1-z^2=0$ gives the relation $2xdx=2zdz$ in the module of Kaehler differentials. Now we have $$\frac{x+z}{z}d\left(\frac{x}{z}\right)=\frac{x+z}{z}\frac{zd(x)-xd(z)}{z^2}$$ In the local rings of both of our points, the function $x$ is invertible, so we have $dx=\frac{z}{x}dz$ and so $$\frac{x+z}{z}\frac{zd(x)-xd(z)}{z^2}=\frac{x+z}{z}\frac{z(z/xd(z))-xd(z)}{z^2}=\frac{xz^2-x^3+z^3-zx^2}{xz^3}d(z)=\frac{(z-x)(x+z)^2}{xz^3}d(z)$$ Moreover, $z$ is a uniformizer of both local rings because at least one of $x\mp i$ is invertible so that $x\pm i=\frac{1}{x\mp i}z^2$ and so the generator $x\pm i$ of the maximal ideal $m_p\mO_{C\cap U_1,p}$ is redundant. Hence we have that $$\text{val}_p(\omega)=\text{val}_p\left(\frac{(z-x)(x+z)^2}{xz^3}\right)=\text{val}_p((z-x)(x+z)^2)-\text{val}_p((x+z)^2)$$ The first term is non-zero if and only if $(z-x)(x+z)^2\in m_p\mO_{C\cap U_1,p}$ if and only if $p_0\pm p_2$, both of our points do not satisfy this. The second term is non-zero if and only if $xz^3\in m_p\mO_{C\cap U_1,p}$ if and only if $p_0=0$ or $p_2=0$. Hence we conclude that the coefficient of both $[i:1:0]$ and $[-i:1:0]$ is $-3$. \\

Putting all the non-zero coefficients together with the corresponding points, we deduce that $$\text{div}(\omega)=[1:0:1]+3[-1:0:1]-3[i:1:0]-3[-i:1:0]$$
\end{proof}
\end{eg}

\begin{prp}{}{}\\
Let $C$ be a smooth irreducible projective curve over $\C$. Let $p\in C$. Let $\omega,\tau\in\Omega_{\C(C)/\C}^1$ be non-zero. Then $\text{div}(\omega)$ and $\text{div}(\tau)$ are linearly equivalent. 
\begin{proof}\\
We know that $\dim(\Omega_{\C(C)/\C^1})=1$. Hence any non-zero element in $\Omega_{\C(C)/C}^1$ forms a basis. So there exists $f\in\C(C)$ such that $\tau=f\omega$. For any point $p\in X$, let $\pi$ be a uniformizer for $\mO_{C,p}$. Then we can write $\omega=hd(\pi)$ for some $h\in\C(C)$ since $d(\pi)$ is a basis for $\Omega_{\C(C)/C}^1$. Then we have $$\text{val}_p(\tau)=\text{val}_p(f\omega)=\text{val}_p(fh)=\text{val}_p(f)+\text{val}_p(h)=\text{val}_p(f)+\text{val}_p(\omega)$$ Hence we conclude that $\text{div}(\tau)=\text{div}(f)+\text{div}(\omega)$. Hence $\text{div}(\tau)$ and $\text{div}(\omega)$ are linearly equivalent. 
\end{proof}
\end{prp}

\begin{defn}{The Canonical Divisor}{}\\
Let $C$ be a smooth irreducible projective curve over $\C$. Let $p\in C$. Define the canonical divisor of $C$ to be $$K_C=[\omega]\in\text{Pic}(C)$$ in the divisor class group for any non-zero $\omega\in\Omega_{\C(C)/\C}^1$. 
\end{defn}

\begin{lmm}{}{}\\
Let $C$ be a smooth irreducible projective curve over $\C$. Then $$\dim_\C(\mL(K_C))=\dim_\C(\Omega_C^1)=p_g(C)$$
\begin{proof}\\
Let $\omega$ be a non-zero element of $\Omega_C^1$. Then $\text{div}(\omega)$ is effective. Define a map $\mL(K_C)\to\Omega_C^1$ by $f\mapsto f\omega$. This map is well defined because $\text{div}(f\omega)=\text{div}(f)+\text{div}(\omega)$ is effective. Conversely, for $\tau\in\Omega_C^1$, we have that $\omega$ is a basis for $\Omega_{\C(C)/C}^1$ and so there exists $g\in\C(C)$ such that $\tau=g\omega$. Then $\text{div}(g)+\text{div}(\omega)=\text{div}(g\omega)=\text{div}(\tau)$ is effective and so $g\in\mL(K_C)$. This gives a map $\Omega_C^1\to\mL(K_C)$ that is the inverse of the first map. They are both $\C$-linear and it is a vector space isomorphism, and hence the both have the same dimension. 
\end{proof}
\end{lmm}

\subsection{The Riemann-Roch Theorem}
\begin{thm}{The Riemann-Roch Theorem}{}\\
Let $C$ be a smooth irreducible projective curve over $\C$. Let $D\in\text{Div}(C)$ be a divisor on $C$. Then $$\dim_\C(\mL(D))-\dim_\C(\mL(K_C-D))=\deg(D)+1-p_g(C)$$
\end{thm}

\begin{prp}{}{}\\
Let $C$ be a smooth irreducible projective curve over $\C$. Let $D\in\text{Div}(C)$ be a divisor on $C$. Then $$\deg(D)+1-p_g(C)\leq\dim_\C(\mL(D))\leq\deg(D)+1$$
\begin{proof}\\
The upper bound is given in 2.2.5. The lower bound is given since $\dim_\C(\mL(K_C-D))\geq 0$ using the Riemann-Roch theorem. 
\end{proof}
\end{prp}

\begin{prp}{}{}\\
Let $C$ be a smooth irreducible projective curve over $\C$. Then we have $$\deg(K_C)=2p_g(C)-2$$
\begin{proof}\\
Taking $D=K_C$, the Riemann-Roch theorem gives $$\deg(K_C)=2p_g(C)-\dim_\C(\mL(0))-1$$ Also we have that $$\mL(0)=\{f\in\C(C)\;|\;\text{div}(f)\text{ is effective }\}=\mO_C(C)$$ Hence $\dim_\C(\mL(0))=\dim(\mO_C(C))=1$. 
\end{proof}
\end{prp}

\begin{lmm}{}{}\\
Let $C$ be a smooth irreducible projective curve over $\C$. If $D\in\text{Div}(C)$ is a divisor such that $\deg(D)>2g-2$, then the following are true. 
\begin{itemize}
\item $\dim_\C(\mL(K_C-D))=0$
\item $\dim_\C(\mL(D))=\deg(D)+1-g$
\end{itemize}
\begin{proof}\\
If $\deg(D)>2g-2$, then we have that $2g-2<\deg(D)=\deg(K_C)-\deg(K_C-D)=2g-2-\deg(K_C-D)$. Hence $\deg(K_C-D)<0$. Then by 2.2.4 we have that $\dim_\C(\mL(K_C-D))=0$. The second item follows from the Riemann-Roch theorem. 
\end{proof}
\end{lmm}

\subsection{The Riemann-Hurwitz Formula}
\begin{defn}{Ramification Index}{}\\
Let $k$ be an algebraically closed field. Let $X,Y$ be smooth irreducible projective curves over $k$. Let $\phi:X\to Y$ be a non-constant regular map. Let $p\in X$. Define the ramification index of $\phi$ at $p$ to be $$e_\phi(p)=v_p(\phi^\ast(\pi))$$ where $\pi$ is a uniformizer of $\mO_{Y,\phi(p)}$. 
\end{defn}

\begin{eg}{}{}\\
Let $k$ be an algebraically closed field. Let $X=\V^H(x^2+y^2-z^2)\subseteq\Prj_k^2$. Let $\phi:X\to\Prj_k^1$ be the map defined by $[x:y:z]\mapsto[x:z]$. Then we have $$e_\phi([1:0:1])=2$$
\begin{proof}\\
Taking local charts $U_2$ on the domain and $U_1$ on the codomain, the restriction of $\phi$ to the affine piece is the map $X\cap U_2=\V(x^2+y^2-1)\to\A_k^1$ defined by $(x,y)\mapsto x$. The induced map local rings is the $k$-algebra homomorphism $\phi^\ast:k[x]\to\frac{k[x,y]}{(x^2+y^2-1)}$ given by $x\mapsto x$. Under the affine chart, the point $[1:0:1]$ has affine coordinates $(1,0)$ in $U_2$. A uniformizer of $\mO_{\A_k^1,1}$ is $x-1$ and we have $\phi^\ast(x-1)=x-1$. Thus we have $$e_\phi([1:0:1])=e_{\phi|_{U_2}}(1,0)=\text{val}_{(1,0)}(x-1)$$ Now we want to work out a uniformizer for $\mO_{X\cap U_2,(1,0)}$. The maximal ideal $m_{(1,0)}\mO_{X\cap U_2,(1,0)}$ is generated by $x-1$ and $y$. Now $x+1\notin m_p$ and so $\frac{1}{x+1}\in\mO_{X\cap U_2,(1,0)}$. Then we have $$x-1=\frac{x^2-1}{x+1}=\frac{-y^2}{x+1}$$ Hence $x+1$ is a redundant generator of $m_{(1,0)}\mO_{X\cap U_2,(1,0)}$. Hence $m_{(1,0)}\mO_{X\cap U_2,(1,0)}=(y)\mO_{X\cap U_2,(1,0)}$ and $y$ is a uniformizer of the DVR. From the same calculations, we deduce that $$e_\phi([1:0:1])=\text{val}_{(1,0)}(x-1)=\text{val}_{(1,0)}\left(\frac{-1}{x+1}y^2\right)=2$$
\end{proof}
\end{eg}

\begin{lmm}{}{}\\
Let $k$ be an algebraically closed field. Let $X,Y$ be smooth irreducible projective curves over $k$. Let $\phi:X\to Y$ be a non-constant regular map. Let $p\in X$. Then $$e_\phi(p)=\dim_k\left(\frac{\mO_{X,p}}{(\phi^\ast(\pi))}\right)$$ where $\pi$ is a uniformizer of $\mO_{Y,\phi(p)}$. 
\begin{proof}\\
If $e_\phi(p)=0$, then $\phi^\ast(\pi)$ is a unit and so $\mO_{X,p}/(\phi^\ast(\pi))=\{0\}$. Hence $\dim_k(\mO_{X,p}/(\phi^\ast(\pi)))=0$. Now suppose that $e_\phi(p)=n\neq 0$. Let $t$ be a uniformizer of $\mO_{X,p}$. Then we have $$\frac{\mO_{X,p}}{(\phi^\ast(\pi))}=\frac{\mO_{X,p}}{(t^n)}$$ is a $\mO_{X,p}/(t)=\mO_{X,p}/m_p\cong k$ vector space. Since $\mO_{X,p}$ is a regular local ring of dimension $1$, we have that $$\dim_{R/m}\left(\frac{\mO_{X,p}}{(t^n)}\right)=\binom{n-1+1}{n-1}=n$$ from Commutative Algebra 2. 
\end{proof}
\end{lmm}

Let $\phi:X\to Y$ be a non-constant regular map between smooth irreducible and projective curves. Since $\phi$ is finite, the notion of degree makes sense. Recall that the degree is defined to be $$\deg(\phi)=\dim_{K(Y)}K(X)$$

\begin{prp}{}{}\\
Let $k$ be an algebraically closed field. Let $X,Y$ be smooth irreducible projective curves over $k$. Let $\phi:X\to Y$ be a non-constant regular map. Let $q\in Y$. Then we have $$\sum_{p\in\phi^{-1}(q)}e_\phi(p)=\deg(\phi)$$
\begin{proof}\\
Without loss of generality take $X$ and $Y$ to be affine. Since $\phi$ is non-constant, it is dominant and so that induced map $\phi^\ast:k[Y]\to k[X]$ is injective. Let $S=k[Y]\setminus m_q$. Localize both $k[Y]$ and $k[X]$ as $k[Y]$-modules to obtain a map $\mO_{Y,q}\to S^{-1}k[X]$. Since every non-constant regular map between curves is finite, we know that $k[X]$ is finitely generated as a $k[Y]$-module and hence $S^{-1}k[X]$ is finitely generated as a $\mO_{Y,q}$-module. \\

I claim that $S^{-1}k[X]$ is a torsion free $\mO_{Y,q}$-module. Indeed, if $f/g\in S^{-1}k[X]$ and $h/k\in\mO_{Y,q}$ are non-zero and that $fh/gk=0$, then $fh/gk=0$ in $k(X)$ is a contradiction since $k(X)$ is a field. Now since $\mO_{Y,q}$ is a DVR, it is also a PID. Hence by the structure theorem for finitely generated modules over a PID, we have that $S^{-1}k[X]\cong\mO_{Y,q}^r$ for some $r\in\N$. Taking fraction fields show that $k(X)$ is an $r$ dimensional vector space over $k(Y)$. Then by definition we have $r=\deg(\phi)$. \\

Let $\pi$ be a uniformizer for $\mO_{Y,q}$. Then $\frac{S^{-1}k[X]}{\phi^\ast(\pi)}$ is an $\frac{\mO_{Y,q}}{(\pi)}\cong k$-module. Then $S^{-1}k[X]\cong\mO_{Y,q}^{\deg(\phi)}$ implies that $$\frac{S^{-1}k[X]}{\phi^\ast(\pi)}\cong\left(\frac{\mO_{Y,q}}{(\pi)}\right)^{\deg(\phi)}$$ Also, by the correspondence of ideals of localization, we know that the maximal ideals of $S^{-1}k[X]$ as a ring is given by $m_p$ for $p\in\phi^{-1}(q)$. Since $\phi$ is a finite morphism, there are finitely many such maximal ideals. I claim that $$(\phi^\ast(\pi))k(X)=\bigcap_{p\in\phi^{-1}(q)}(\phi^\ast(\pi)\mO_{X,p}\cap S^{-1}k[X])\subseteq k(X)$$ Clearly the left is contained in the right. Now let $f$ be an element in the right. Then $f/(\phi^\ast(\pi))\in\mO_{X,p}$. Since any integral domain is the intersection of its localization at maximal ideals, we conclude that $f/(\phi^\ast(\pi))\in S^{-1}k[X]$. Hence $f\in(\phi^\ast(\pi))$. \\

Now the ideals $(\phi^\ast(\pi)\mO_{X,p}\cap S^{-1}k[X])$ for varying $p$ are coprime. By the CRT, we have $$\frac{S^{-1}k[X]}{(\phi^\ast(\pi))}\cong\prod_{p\in\phi^{-1}(q)}\frac{S^{-1}k[X]}{\phi^\ast(\pi)\mO_{X,p}\cap S^{-1}k[X]}\cong\prod_{p\in\phi^{-1}(q)}\frac{\mO_{X,p}}{\phi^\ast(\pi)\mO_{X,p}}$$ where the second isomorphism comes from the fact that localization at $S^{-1}k[X]\setminus m_p$ commutes with quotients and $S^{-1}k[X]\setminus m_p$ consists of units in $S^{-1}k[X]$. Then we have $$\deg(\phi)=\dim\left(\frac{S^{-1}k[X]}{(\phi^\ast(\pi))}\right)=\sum_{p\in\phi^{-1}(q)}\dim\left(\frac{\mO_{X,p}}{\phi^\ast(\pi)\mO_{X,p}}\right)=\sum_{p\in\phi^{-1}(q)}e_\phi(p)$$
\end{proof}
\end{prp}

\begin{defn}{Ramification Divisor}{}\\
Let $X,Y$ be smooth irreducible projective curves over $\C$. Let $\phi:X\to Y$ be a non-constant regular map. Define the ramification divisor to be $$R_\phi=\sum_{p\in X}(e_\phi(p)-1)\in\text{Div}(X)$$
\end{defn}

\begin{prp}{}{}\\
Let $X,Y$ be smooth irreducible projective curves over $\C$. Let $\phi:X\to Y$ be a non-constant regular map. Then we have $$K_X=\phi^\ast(K_Y)+R_\phi$$
\end{prp}

\begin{thm}{Riemann-Hurwitz Formula}{}\\
Let $X,Y$ be smooth irreducible projective curves over $\C$. Let $\phi:X\to Y$ be a non-constant regular map. Then we have $$2p_g(X)-2=\deg(\phi)(2p_g(Y)-2)+\deg(R_\phi)$$
\end{thm}

\begin{prp}{}{}\\
Let $C\subseteq\Prj_\C^2$ be a smooth curve of degree $d$. Then we have $$p_g(C)=\frac{(d-1)(d-2)}{2}$$
\end{prp}

\pagebreak
\section{Maps from Curves to Projective Space}
\subsection{The Associated Map of Divisors}
\begin{defn}{Associated Map of Divisors}{} Let $C$ be a smooth projective irreducible curve over $\C$. Let $D\in\text{Div}(C)$ be a divisor. Define the associated rational map $F_D:C\to\Prj(\mL(D)^\ast)$ by $$p\mapsto\left(\substack{\phi_p:\mL(D)\to\C\\f\mapsto(f\cdot\pi^{n_p})(p)}\right)$$ where $\pi$ is a uniformizer of $\mO_{C,p}$ and $n_p$ is the coefficient of $p$ in $D$. 
\end{defn}

\begin{lmm}{}{}\\
Let $C$ be a smooth projective irreducible curve over $\C$. Let $D\in\text{Div}(C)$ be a divisor. Then the associated map $F_D$ is a a rational map. 
\begin{proof}\\
Let $f\in\mL(D)$ be non zero. Let $n_p$ be the coefficient of $p\in C$ in $D$. By definition of $\mL(D)$, we know that $f\pi^{n_p}$ is regular at $p$. We have $(f\pi^{n_p})(p)\neq 0$ if and only if $f(p)\neq 0$ and $n_p=0$. The condition $f(p)\neq 0$ is an open subset of $C$ since $f$ is regular. Similarly, the condition that $n_p=0$ gives an open subset of $C$ since $D$ is a finite sum of points in $C$. Hence $S(f)=\{p\in C\;|\;(f\pi^{n_p})(p)\neq 0\}$ is an open subset of $C$. Ranging over $f$, we have $$\bigcup_{f\in\mL(D)\setminus\{0\}}S(f)=\{p\in C\;|\;\exists f\in\mL(D)\text{ such that }(f\pi^{n_p})(p)\neq 0\}$$ is an open set. \\

Now $F_D$ is a rational map as long as $\{p\in C\;|\;\phi_p\text{ is not the zero map }\}$ is an open set. For any point $p$, $\phi_p$ is not the zero map if and only if there exists $f\in\mL(D)$ such that $\phi_p(f)=(f\pi^{n_p})(p)\neq 0$. We just showed that $\{p\in C\;|\;\exists f\in\mL(D)\text{ such that }(f\pi^{n_p})(p)\neq 0\}$ is an open set hence $F_D$ is indeed a rational map. 
\end{proof}
\end{lmm}

\subsection{The Associated Map as a Regular Map}
We discuss the conditions in which $F_D$ is a regular map, and when $F_D$ is an embedding. 

\begin{defn}{Basepoint Free Divisor}{} Let $C$ be a smooth projective irreducible curve over $\C$. Let $D\in\text{Div}(C)$ be a divisor. We say that $D$ is basepoint free for all $p\in C$, there exists $f\in\mL(D)$ such that $$\text{val}_p(f\pi^{n_p})=0$$ where $\pi$ is a uniformizer of $\mO_{C,p}$ and $n_p$ is the coefficient of $p$ in $D$. 
\end{defn}

\begin{lmm}{}{}\\
Let $C$ be a smooth projective irreducible curve over $\C$. Let $D\in\text{Div}(C)$ be a divisor. If $D$ is basepoint free, then the associated map $F_D:C\to\Prj(\mL(D)^\ast)$ is a regular map. 
\begin{proof}\\
As seen from the above lemma, $F_D$ is a well defined rational map on the open set $\{p\in C\;|\;\exists f\in\mL(D)\text{ such that }\text{val}_p(f\pi^{n_p})=0\}$. The condition that $D$ is base point free guarantees that this set is equal to $C$ and so $F_D$ is regular on all of $C$. 
\end{proof}
\end{lmm}

\begin{prp}{}{}\\
Let $C$ be a smooth projective irreducible curve over $\C$. Let $D\in\text{Div}(C)$ be a divisor. Then $D$ is basepoint free if and only if $$\dim_\C(\mL(D-p))=\dim_\C(\mL(D))-1$$ for all $p\in C$. 
\begin{proof}\\
Let $n_p$ be the coefficient of a point $p\in C$ in $D$. Suppose that $D$ is basepoint free. Then there exists $f\in\mL(D)$ such that $\text{val}_p(f)+n_p=0$ and so $\text{val}_p(f)+n_p-1<0$. Hence $D-p+\text{div}(f)$ is not effective and so $f\notin\mL(D-p)$. Hence $\mL(D-p)\subset\mL(D)$ is a strict subset. Then by 2.2.5 we are done. \\

Conversely, suppose that the condition on dimensions are satisfied. Since $\mL(D-p)\subset\mL(D)$ is a strict subset, there exists $f\in\mL(D)$ such that $D-p+\text{div}(f)$ is not effective but $D+\text{div}(f)$ is effective. This means that $\text{val}_p(f\pi^{n_p-1})<0$ and $\text{val}_p(f\pi^{n_p})\geq0$. Hence $\text{val}_p(f\pi^{n_p})=0$. Since this is true for all $p$, we conclude that $D$ is base point free. 
\end{proof}
\end{prp}

\begin{crl}{}{}\\
Let $C$ be a smooth projective irreducible curve over $\C$. Let $D\in\text{Div}(C)$ be a divisor. If $\deg(D)\geq 2g$ then $D$ is base point free. 
\begin{proof}
If $\deg(D)\geq 2g$, then for any $p\in C$, we have $\deg(D-p)\geq 2g-1$. Hence by a corollary of the Riemann-Roch theorem, we have $$\dim_\C(\mL(D-p))=\deg(D-p)+1-g=\deg(D)+1-g-1=\dim_\C(\mL(D))-1$$ By the above proposition, $D$ is base point free. 
\end{proof}
\end{crl}

\subsection{The Associated Map as an Emebdding}
\begin{defn}{Very Ample Divisor}{} Let $C$ be a smooth projective irreducible curve over $\C$. Let $D\in\text{Div}(C)$. We say that $D$ is very ample if $D$ is base point free and the associated map $F_D:C\to\Prj(\mL(D)^\ast)$ is an embedding. 
\end{defn}

\begin{prp}{}{} Let $C$ be a smooth projective irreducible curve over $\C$. Let $D\in\text{Div}(C)$. Then $D$ is very ample if and only if for all $p,q\in C$, we have $$\dim_\C(\mL(D-p-q))=\dim_\C(\mL(D))-2$$
\end{prp}

\begin{crl}{}{} Let $C$ be a smooth projective irreducible curve over $\C$. Let $D\in\text{Div}(C)$. If $\deg(D)\geq 2g+1$ then $D$ is very ample. 
\end{crl}

\subsection{The Moduli Space of Effective Divisors Equivalent to a Divisor}
\begin{prp}{}{}\\
Let $C$ be a smooth projective irreducible curve over $\C$. Let $D\in\text{Div}(C)$ be a divisor. Then there is a bijection $$\Prj(\mL(D))\;\;\overset{1:1}{\longleftrightarrow}\;\;\{E\in[D]\;|\;E\text{ is effective }\}$$ given by $[f]\mapsto D+\text{div}(f)$. 
\end{prp}

Notice that if $D$ itself is effective, then the inverse of the above bijection sends $D$ to the equivalence class of all constant functions in $\Prj(\mL(D))$. 

\begin{prp}{}{}\\
Let $C$ be a smooth projective irreducible curve over $\C$. Let $D\in\text{Div}(C)$ be a base point free divisor. Then there is a one-to-one correspondence $$\{H\subseteq\Prj(\mL(D)^\ast)\;|\;H\text{ is a hyperplane }\}\;\;\overset{1:1}{\longleftrightarrow}\;\;\{E\in\text{Div}(C)\;|\;E\text{ is effective and }E\sim D\}$$ The map is given by $H\mapsto(F_D)^\ast(H)$ where $F_D^\ast$ is the map $\text{Div}(\Prj(\mL(D)^\ast))\to\text{Div}(C)$. 
\begin{proof}\\
We know from Algebraic Geometry 1 that there is a bijection between between points in $\Prj(\mL(D))$ and hyperplanes in $\Prj(\mL(D)^\ast)$. Thus combining with thew above prp gives a bijection between our two sets. We want to check that this composition of bijection is indeed the map $H\mapsto(F_D)^\ast(H)$. \\

Choose a basis $e_0,\dots,e_n$ for $\mL(D)$ and hence a dual basis $\alpha_0,\dots,\alpha_n$ for $\mL(D)^\ast$. Let $H$ be a hyperplane in $\Prj(\mL(D)^\ast)$. Then $H=\V^H(A_0x_0+\dots+A_nx_n)$ for $x_0,\dots,x_n$ the coordinates on $\Prj(\mL(D)^\ast)$ and $A_0,\dots,A_n\in\C$. Then the duality between points and hyperplanes gives the point $[A_0:\cdots:A_n]\in\Prj_\C^n$. Under the bijection $\Prj_\C^n\cong\Prj(\mL(D))$, this corresponds to the point $\left[\sum_{k=0}^nA_ke_k\right]\in\Prj(\mD)$. From the above prp, this point is sent to the divisor $D+\text{div}\left(\sum_{k=0}^nA_ke_k\right)$. The coefficient of any point $p\in C$ in the divisor is given by $n_p+\text{val}_p\left(\sum_{k=0}^nA_ke_k\right)$. Choosing a uniformizer $\pi\in\mO_{C,p}$, the coefficient becomes $\left(\left(\sum_{k=0}^nA_ke_k\right)\pi^{n_p}\right)$. \\

We want to show that the coefficient of $p$ above is the same as the coefficient for $p$ in $(F_D)^\ast(H)$. Firstly, notice that a generator for the ideal $\I^H(H)\mO_{\Prj(\mL(D)^\ast),\phi(p)}$ is given by $\I^H(\V^H(A_0x_0+\dots+A_nx_n))=(A_0x_0+\dots+A_nx_n)$. The coefficient of $p$ in this divisor is by definition, given by $\text{val}_p\left((F_D)^\ast\left(\sum_{k=0}^nA_kx_k\right)\right)$ where $F_D^\ast$ here is the map $\mO_{\Prj(\mL(D)^\ast),\phi(p)}\subseteq k(x_0,\dots,x_n)\to\mO_{C,p}$. The map sends $\sum_{k=0}^nA_kx_k$ to $\left(\sum_{k=0}^nA_kx_k\right)\circ F_D=\sum_{k=0}^nA_k(x_k\circ F_D)$. Now since $x_k$ are the coordinate functions, $x_k\circ F_D$ picks out the effect of $F_D$ on the $k$th basis vector $e_k$ of $\mL(D)$. For each point $p$, $F_D(p)$ sends $f\in\mL(D)$ to $(f\pi^{n_p})(p)$ hence $x_k\circ F_D=e_k\pi^{n_p}$. Thus the coefficient of $p$ of the divisor $(F_D)^\ast(H)$ coincides with that of the above paragraph. 
\end{proof}
\end{prp}


\subsection{Classification of Regular Maps into Projective Space}
\begin{thm}{}{}\\
Let $C$ be a smooth irreducible projective curve over $\C$. Let $F:C\to\Prj_\C^n$ be a non-constant regular map. Let $H\subseteq\Prj_\C^n$ be a hyperplane such that $D=\phi^\ast(H)$ is a divisor. Then $F$ is equal to the composition of $F_D$ together with a sequence of projections $\Prj_\C^k\setminus\V(p)\to\Prj_\C^{k-1}$ for $p\in\Prj_\C^k$ and a linear embedding $\Prj_\C^k\to\Prj_\C^n$. 
\end{thm}

\begin{thm}{}{}\\
Let $C$ be a smooth irreducible projective curve over $\C$. Then $C$ admits an embedding $C\to\Prj_\C^3$. 
\end{thm}

\pagebreak
\section{Curves with Low Genus}
\subsection{Genus 0 Curves}
We show that $\Prj_\C^1$ is the unique (up to isomorphism) smooth irreducible projective curve with genus $0$. We also exhibit characterizing properties of $\Prj_\C^1$ in terms of divisors. 

\begin{prp}{}{}\\
Let $C$ be a smooth irreducible projective curve over $\C$. Then the following are equivalent. 
\begin{itemize}
\item $C$ is isomorphic to $\Prj_\C^1$. 
\item The geometric genus $p_g(C)=0$ is zero. 
\item For all $p,q\in C$, $p\sim q$ are linearly equivalent. 
\item There exists distinct $p,q\in C$, such that $p\sim q$ are linearly equivalent. 
\item The degree map $\deg:\text{Pic}(C)\to\Z$ is an isomorphism. 
\item For all $D\in\text{Div}(C)$ with $\deg(D)>0$, we have $\dim_\C(\mL(D))=\deg(D)+1$. 
\item There exists $D\in\text{Div}(C)$ with $\deg(D)>0$ such that $\dim_\C(\mL(D))=\deg(D)+1$. 
\end{itemize}
\begin{proof}\\
$(1)\implies(2)$: Consider $d(x/y)\in\Omega_{\C(\Prj_1)/\C}^1$. On the affine chart $U_y$, the differential form is given by $d(x)$. For any $p\in U_y$, $x-p$ is a uniformizer for $\mO_{U_y,p}$ since $m_p$ is generated by $x-p$ alone. Then $d(x-p)=d(x)$ and so $\text{val}_{[p:1]}(d(x/y))=0$ for all $p\in\C$. Now for the remaining point $[0:1]$, we consider the open chart $U_x$. The differential form is now given by $d(1/y)$. Moreover, $m_0\mO_{U_x,0}$ is generated by $y$ and so it is also the uniformizer. Then we have $d(1/y)=-1/y^2d(y)$ be the chain rule and so $\text{val}_{[0:1]}(d(x/y))=-2$. Hence $\text{div}(d(x/y))=-2$. Then we have $p_g(\Prj_\C^1)=\dim(K_{\Prj_\C^1})=\dim_\C(\mL(\text{div}(d(x/y))))=0$. \\

$(2)\implies(3)$: Let $p,q\in C$ be two points. By the Riemann Roch theorem, we have $$\dim_\C(\mL(p-q))-\dim_\C(\mL(K_C-p+q))=1$$ Since $\deg(p-q)=0\geq -2=2p_g(C)-2$, we have that $\dim_\C(\mL(K_C-p+q))=0$ and so $\dim_\C(\mL(p-q))=1$. Then by 2.2.4 we conclude that $p-q$ is a principal divisor. \\

$(3)\implies(4)$: Clear. \\

$(4)\implies(1)$: Suppose that $p-q$ is a principal divisor for some $p,q\in C$ distinct. Then there exists $f\in\C(C)$ such that $\text{div}(f)=p-q$. The rational function $f:C\to\A_\C^1$ gives rise to a rational map $\phi:C\to\Prj_\C^1$ by identifying $\A_\C^1$ with the open chart $U_y$. It is non constant since $f(p)=0$. Moreover, it is regular at $C\setminus\{p\}$. Then we can extend this rational map into a non-constant regular map $\overline{\phi}:C\to\Prj_\C^1$. In particular, $\overline{\phi}$ is surjective. Since $\phi(C\setminus\{q\})\subseteq\Prj_\C^1\setminus\{[1:0]\}$ and $\overline{\phi}$ is surjective, we conclude that $\overline{\phi}(q)=[1:0]$. Now we know that $\phi^\ast([1:0])=e_\phi(q)q$ where $\phi^\ast$ is the map $\text{Div}(\Prj_\C^1)\to\text{Div}(C)$. ??????? Hence $e_\phi(q)=1$. Then we have $\deg(\overline{\phi})=e_\phi(q)=1$ by 2.1.2. Hence $k(C)$ is one dimensional over $k(\Prj_\C^1)$ and so $C$ and $\Prj_\C^1$ are birational. Hence $C$ and $\Prj_\C^1$ are isomorphic. \\

$(3)\iff(5)$: If $p\sim q$ for any two points $p,q\in C$, then any degree $0$ divisor is principal and so the surjective homomorphism $\deg:\text{Pic}(C)\to\Z$ is an isomorphism by the first isomorphism theorem. Conversely, if $\deg:\text{Pic}(C)\to\Z$ is an isomorphism, then $[p-q]\in\ker(\deg)$ for any $p,q\in C$ implies that $p-q$ is principal. \\

$(2)\implies(6)$: Let $D\in\text{Div}(C)$ be a divisor such that $\deg(D)\geq 0$. Then we have $\deg(D)\geq 0\geq -2=2p_g(C)-2$ and so $\dim_\C(\mL(D))=\deg(D)+1$ by a corollary of the Riemann Roch theorem. \\

$(6)\implies(7)$: Clear. \\

$(7)\implies(4)$: Let $D$ be a divisor as in the assumption. Let $p_1,\dots,p_n\in C$ be points. Then we have 
\begin{align*}
\deg(D)+1&=\dim_\C(\mL(D))\\
&\leq\dim_\C(\mL(D-p_1))+1\\
&\leq\cdots\\
&\leq\dim_\C\left(\mL\left(D-\sum_{k=1}^{n-1}p_k\right)\right)+(n-1)\\
&\leq\dim_\C\left(\mL\left(D-\sum_{k=1}^np_k\right)\right)+n
\end{align*} by lemma 2.2.5. When $n=\deg(D)+1$, we have that $\dim_\C\left(\mL\left(D-\sum_{k=1}^np_k\right)\right)=0$ by 2.2.4. So all the inequalities are equal. In particular, we have $$\dim_\C\left(\mL\left(D-\sum_{k=1}^{\deg(D)-1}p_k\right)\right)+(\deg(D)-1)=\deg(D)+1$$ and so $\dim_\C\left(\mL\left(D-\sum_{k=1}^{\deg(D)-1}p_k\right)\right)=2$. Let $f_1,f_2\in\mL\left(D-\sum_{k=1}^{\deg(D)-1}p_k\right)$ be linearly independent. Since $D-\sum_{k=1}^{\deg(D)-1}p_k$ has degree $1$ and $D-\sum_{k=1}^{\deg(D)-1}p_k+\text{div}(f_1)$ and $D-\sum_{k=1}^{\deg(D)-1}p_k+\text{div}(f_2)$ are effective, there exists $p,q\in C$ such that $D-\sum_{k=1}^{\deg(D)-1}p_k=p$ and $D-\sum_{k=1}^{\deg(D)-1}p_k=q$. Also, we have $p-q=\text{div}(f_1/f_2)$ and so $p\sim q$. Finally, $p$ and $q$ are distinct points because otherwise $\text{div}(f_1/f_2)=0$ implies that $f_1$ and $f_2$ are linear multiples of each other, contradicting linear independence. 
\end{proof}
\end{prp}

\subsection{Genus 1 Curves}
\begin{prp}{}{}\\
Let $C$ be a smooth irreducible projective curve over $\C$. Then the following are true. 
\begin{itemize}
\item If $p_g(C)=1$, then $C$ admits an embedding $C\to\Prj_\C^2$ as a degree $3$ curve. 
\item If $C\subseteq\Prj_\C^2$ is a degree $3$ curve, then $p_g(C)=1$. 
\end{itemize}
\end{prp}

\begin{lmm}{}{}\\
Let $C$ be a smooth irreducible projective curve over $\C$. Then $K_C$ is a principal divisor. 
\end{lmm}

\begin{defn}{Elliptic Curve}{}\\
Let $k$ be an algebraically closed field. Let $C$ be a smooth irreducible curve over $k$. We say that $C$ is an elliptic curve if $C$ admits an embedding $C\to\Prj_k^1$. 
\end{defn}

\begin{prp}{}{}\\
Let $C$ be an elliptic curve over $\C$. Then there is a bijection of sets $$C\cong\{D\in\text{Pic}(C)\;|\;\deg(D)=0\}$$
\end{prp}

\subsection{Genus 2 Curve}
\begin{prp}{}{}\\
Let $C$ be a smooth irreducible projective curve over $\C$ of genus $2$. Then $C$ admits an embedding $C\to\Prj_\C^3$ as a degree $5$ curve. 
\end{prp}




\pagebreak
\section{Basic Properties of Algebraic Curves}
\subsection{Some Classical Results}
\begin{prp}{}{} Let $C$ be an affine irreducible curve over $\C$. Then $C$ is smooth if and only if $C$ is a normal variety. 
\end{prp}

Recall that by taking the integral closure of the coordinate ring $k[C]$ of an irreducible affine curve $C\subseteq\A^n$, we obtain a corresponding variety $\widetilde{C}$ called the normalization of $C$. \\

Moreover, within each birational class of irreducible curves, one can discuss the isomorphism classes of curves. 

\begin{crl}{}{} Let $k$ be an algebraically closed field. Let $C\subseteq\A_k^n$ be an irreducible affine curve over $k$. Then the normalization $\widetilde{C}$ is smooth. 
\end{crl}

\begin{prp}{}{}\\
Let $k$ be an algebraically closed field. Then there is an equivalence of categories $$\substack{\text{Smooth projective curves}\\\text{over }k\text{ with}\\\text{dominant morphisms}}\cong\substack{\text{Function fields over }k\\\text{with morphisms of }k\text{-algebra}}$$ given by the contravariant functor $C\mapsto K(C)$. 
\end{prp}

\begin{prp}{}{} Let $k$ be a field. Let $C$ be a smooth curve over $k$. Then for any projective variety $X\subseteq\Prj^n$ and rational map $\phi:C\to X$, there exists a regular map $$\overline{\phi}:C\to X$$ such that $\overline{\phi}|_U=\phi|_U$ for some dense subset $U\subseteq C$. 
\end{prp}

\begin{prp}{}{} Let $k$ be an algebraically closed field. Let $X,Y$ be smooth irreducible projective curves over $k$. Let $\phi:X\to Y$ be a rational map. If $\phi$ is birational, then $\phi$ is an isomorphism of varieties. 
\end{prp}

\begin{thm}{}{} Let $k$ be an algebraically closed field. Let $C$ be an irreducible curve over $k$. Then $C$ is birational to a unique (up to isomorphism) projective smooth irreducible curve. 
\begin{proof}
We know that $C$ is birational to its normalization, which is a smooth curve. 
\end{proof}
\end{thm}

\subsection{Basic Properties of Algebraic Curves}
\begin{defn}{Algebraic Curves}{} Let $k$ be an algebraically closed field. Let $X$ be variety over $k$. We say that $X$ is a curve if $\dim(X)=1$. 
\end{defn}

\begin{prp}{}{}\\
Let $k$ be an algebraically closed field. Let $X$ be a curve over $k$. Then the following are equivalent. 
\begin{itemize}
\item $X$ is a projective variety. 
\item $X$ is a complete variety. 
\item There exists a finitely generated field extension $K$ of transcendence degree $\text{tr deg}_k(K)=1$ such that $X\cong C_K$ where $C_k$ is the associated scheme of the variety associated to the field extension $K$. 
\end{itemize}
\end{prp}

\begin{prp}{}{} Let $k$ be an algebraically closed field. Let $X,Y$ be curves over $k$. Let $\phi:X\to Y$ be a morphism. If $X$ is regular and complete, then exactly one of the following are true. 
\begin{itemize}
\item $\phi$ is a constant map. 
\item $\phi$ is surjective, finite and $Y$ is complete. 
\end{itemize}
\end{prp}

\begin{prp}{}{}\\
Let $k$ be a field. Let $X$ be a complete regular curve over $k$. Then we have $$p_g(C)=\dim_k(H^1(X,\mO_X))$$
\end{prp}

\pagebreak
\section{Divisors on Curves}
\subsection{The Pullback Map of Divisors}

\subsection{The Linear System of Divisors}

\subsection{The Riemann-Roch Theorem}
\begin{defn}{Canonical Divisor}{}
\end{defn}

\begin{thm}{Riemann-Roch Theorem}{} Let $X$ be an algebraic curve. Let $D$ be a divisor on $X$ and let $K$ be the canonical divisor of $X$. Let $\mL(D)$ be the associated sheaf of the divisor $D$. Then $$\dim_k(H^0(X,\mL(D)))+\dim_k(H^0(X,\mL(K-D)))=\deg(D)+1-p_g(X)$$
\end{thm}

\subsection{Classification of Curves in $\Prj^3$}




















\end{document}