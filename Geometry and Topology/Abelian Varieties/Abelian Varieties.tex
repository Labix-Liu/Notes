\documentclass[a4paper]{article}

\input{C:/Users/liula/Desktop/Latex/Headers V1.2.tex}

\pagestyle{fancy}
\fancyhf{}
\rhead{Labix}
\lhead{Abelian Varieties}
\rfoot{\thepage}

\title{Abelian Varieties}

\author{Labix}

\date{\today}
\begin{document}
\maketitle
\begin{abstract}
\begin{itemize}
\end{itemize}
\end{abstract}
\pagebreak
\tableofcontents

\pagebreak
\section{Properties of Abelian Varieties}
\subsection{Group Schemes and Group Varieties}
\begin{defn}{Group Schemes}{} A group scheme is a group object in the category $\bold{Sch}$ of schemes. A group scheme over a scheme $S$ is a group object in the category $\bold{Sch}_S$ of schemes over $S$. 
\end{defn}

\begin{defn}{Group Varieties}{} A group variety over a field $k$ is a group object in the category $\bold{Var}_k$ of varieties over $k$. 
\end{defn}

\begin{defn}{Algebraic Groups}{} An algebraic group over a field $k$ is a group variety over $k$ that is also smooth. 
\end{defn}

\begin{prp}{}{} Let $k$ be a field with characteristic $0$. Then every group scheme over $k$ is smooth. 
\end{prp}

\subsection{Basic Definitions}
Let us start by recalling the definition of an abelian variety in Algebraic Geometry 3. 

\begin{defn}{Abelian Varieties}{} An abelian variety over a field $k$ is a group variety that is complete and connected. 
\end{defn}

\begin{thm}{Rigidity Theorem}{}
\end{thm}

\begin{crl}{}{} The group law on any abelian variety is commutative, hence every abelian variety has a the structure of an abelian group. 
\end{crl}

\subsection{Rational Maps into Abelian Varieties}
\begin{thm}{}{} Let $A$ be an irreducible abelian variety over $k$. Then for any non-singular irreducible variety $V$ and rational map $\varphi:V\to A$, $\varphi$ extends to a morphism $V\to A$. 
\end{thm}

\subsection{Abelian Varieties are Projective}
\begin{thm}{Abelian Varieties are Projective}{} Every abelian variety over an algebraically closed field $k$ is projective. 
\end{thm}

\begin{thm}{}{} Every abelian variety over $\C$ is a compact complex submanifold of $\Prj^n(\C)$. 
\end{thm}

\end{document}
