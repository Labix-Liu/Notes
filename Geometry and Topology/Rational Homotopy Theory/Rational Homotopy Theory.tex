\documentclass[a4paper]{article}

\input{C:/Users/liula/Desktop/Latex/Headers V1.2.tex}

\pagestyle{fancy}
\fancyhf{}
\rhead{Labix}
\lhead{Rational Homotopy Theory}
\rfoot{\thepage}

\title{Rational Homotopy Theory}

\author{Labix}

\date{\today}
\begin{document}
\maketitle
\begin{abstract}
\end{abstract}
References
\begin{itemize}
\item Rational Homotopy Theory and Differential Forms
\end{itemize}
\pagebreak
\tableofcontents

\pagebreak
\section{Homotopy with Coefficients}
\subsection{Rational Spaces}
The following construction is a special case of localization of spaces, taken with the localizing set $T=\emptyset$ so that $\Z_T=\Q$. 

\begin{defn}{Rational Spaces}{} Let $X$ be a space. We say that $X$ is a rational space if the following are true. 
\begin{itemize}
\item $X$ is homotopy equivalent to a CW complex
\item $\pi_1(X)=0$
\item $\pi_n(X)$ is a $\Q$-vector space for each $n\in\N$
\end{itemize}
\end{defn}

\begin{lmm}{}{} The following are true regarding the Eilenberg-maclane spaces of $\Q$. 
\begin{itemize}
\item $\widetilde{H}_k(K(\Q,n);\Z)$ is a $\Q$-vector space for each $k\in\N$. 
\item $H^k(K(\Q,2n);\Q)$ is a $\Q$-polynomial algebra on one generator, and the generator has degree $2n$. 
\item 
\end{itemize}
\end{lmm}

\begin{crl}{}{} The induced map $K(\Z,n)\to K(\Q,n)$ given by the map $\Z\to\Q$ induces an isomorphism of rational cohomology and rational homology. 
\end{crl}

\begin{thm}{}{} Let $X$ be a space. Then $X$ is a rational space if and only if the following conditions are satisfied: 
\begin{itemize}
\item $X$ is homotopy equivalent to a CW complex
\item $\pi_1(X)=0$
\item $\widetilde{H}_n(X,\Z)$ is a $\Q$-vector space
\end{itemize}
\end{thm}

\subsection{Rational Homotopy Type}
\begin{defn}{Rational Homotopy Equivalence}{} Let $f:X\to Y$ be a map of spaces. We say that $f$ is a rational homotopy equivalence if the induced map gives isomorphisms $$f_\ast:\pi_n(X)\otimes\Q\overset{\cong}{\longrightarrow}\pi_n(Y)\otimes\Q$$ of $\Q$-vector spaces. 
\end{defn}

Intuitively, tensoring the homotopy groups with $\Q$ forgets about the torsion subgroups, thus leading to an even more crude algebraic invariant for (simply connected) spaces. In other words, we have the following relation: $$\text{Homeomorphisms}\subset\substack{\text{Homotopy}\\\text{Equivalences}}\subset\substack{\text{Weak Homotopy}\\\text{Equivalences}}\subset\substack{\text{Rational Homotopy}\\\text{Equivalences}}$$ The hope is that while we are losing some information, it makes the rational homotopy groups more computable. 

\begin{thm}{}{} Let $X,Y$ be simply connected CW complexes and let $f:X\to Y$ be a map. If $Y$ is a rational space, then the following conditions are equivalent. 
\begin{itemize}
\item The induced map $f_\ast:\pi_n(X)\otimes\Q\to\pi_n(Y)\otimes\Q\cong\pi_n(Y)$ is a rational homotopy equivalence
\item The induced map $f_\ast:\widetilde{H}_n(X;\Q)\to\widetilde{H}_n(Y;\Q)\cong\widetilde{H}_n(Y;\Z)$ is an isomorphism for all $n\in\N$
\item $f$ is universal for maps of $X$ into $\Q$-space. This means that if $g:X\to Z$ is another map into a $\Q$-space $Z$, then there exists a map $h:Y\to Z$ unique up to homotopy such that the following diagram commutes: \\~\\
\adjustbox{scale=1,center}{\begin{tikzcd}
	X & Z \\
	Y
	\arrow["g", from=1-1, to=1-2]
	\arrow["f"', from=1-1, to=2-1]
	\arrow["{\exists h}"', dashed, from=2-1, to=1-2]
\end{tikzcd}}\\~\\
\end{itemize}
\end{thm}

\begin{lmm}{}{} Let $X$ be a simply connected CW complex. Then $\pi_n(X)\otimes\Q=0$ for all $n\in\N$ if and only if $\widetilde{H}_n(X;\Q)=0$ for all $n\in\N$. 
\end{lmm}

\begin{defn}{Rationalization and Rational Homotopy Type}{} Let $X$ be a CW complex. The rationalization of $X$ is a rational space $X_{(0)}$ together with a rational homotopy equivalence $f:X\to X_{(0)}$. In this case we say that $X_{(0)}$ is the rational homotopy type of $X$. 
\end{defn}

Prereq: postinokov towers

\begin{thm}{}{} Let $X$ be a CW complex. The rationalization of $X$ exists and is unique up to homotopy equivalence. \\~\\

Explicitly, if $X_{(0)}$ and $Y$ are both rationalizations of $X$ with maps $f:X\to X_{(0)}$ and $g:X\to Y$, then there exists a homotopy equivalence $h:X_{(0)}\to Y$ such that the following diagram commutes: \\~\\
\adjustbox{scale=1,center}{\begin{tikzcd}
	X & Y \\
	{X_{(0)}}
	\arrow["g", from=1-1, to=1-2]
	\arrow["f"', from=1-1, to=2-1]
	\arrow["{\exists h}"', dashed, from=2-1, to=1-2]
\end{tikzcd}}\\~\\
\end{thm}


\end{document}
