\documentclass[a4paper]{article}

\input{C:/Users/liula/Desktop/Latex/Headers V1.2.tex}

\pagestyle{fancy}
\fancyhf{}
\rhead{Labix}
\lhead{Complex Analytic Geometry}
\rfoot{\thepage}

\title{Complex Analytic Geometry}

\author{Labix}

\date{\today}
\begin{document}
\maketitle
\begin{abstract}
References: Algebraic and Analytic Geometry Neeman, \\
Several Complex Variables with Connections to Algebraic Geometry and Lie Groups
\end{abstract}
\pagebreak
\tableofcontents

\pagebreak
\section{Analytic Manifolds}
\subsection{Real Analytic Manifolds}
\begin{defn}{Real Analytic Manifolds}{} A real analytic manifold is a topological manifold with analytic transition maps. 
\end{defn}

In the real case, every analytic manifold is a smooth manifold because every analytic function is necessarily infinitely differentiable. While not every analytic function is smooth, we can still equip a smooth structure on analytic manifolds. 

\begin{thm}{(Grauert-Whitney)}{} Every smooth manifolds admits a compatible real analytic structure. 
\end{thm}

\begin{defn}{Analytic Functions}{} Let $M$ be a real analytic manifold. Let $U\subseteq M$ be a subset. Let $f:U\to\R$ be a function. We say that $f$ is analytic if for all $x\in U$, there exists a chart $(V,\varphi=x^1,\dots,x^n))$ of $M$ such that $$f\circ\varphi^{-1}:\R^n\to\R$$ is analytic in the usual sense. 
\end{defn}

\begin{defn}{Sheaf of Analytic Functions}{} Let $M$ be an analytic manifold. Define the sheaf of analytic functions $$\mA_M:\bold{Open}\to\bold{CRing}$$ as follows. 
\begin{itemize}
\item For each open set $U\subseteq M$, $\mA_M(U)$ consists of the ring of analytic functions on $U$
\item For each inclusion $V\subseteq U$, there is a unique morphism $$\text{res}_V^U:\mA_M(U)\to\mA_M(V)$$ given by $f\mapsto f|_V$
\end{itemize}
\end{defn}

\subsection{Complex Analytic Manifolds}
\begin{defn}{Complex Analytic Manifolds}{} A complex analytic manifold is a topological manifold with complex analytic transition maps. 
\end{defn}

Since holomorphic functions are the same as analytic functions in the complex case, this is just the usual definition of complex manifolds. 

\begin{thm}{Oka's (Coherence) Theorem}{} Let $M$ be a complex manifold. Then the sheaf $\mO_M$ of holomorphic functions on $M$ is a coherent sheaf. 
\end{thm}

\pagebreak
\section{Analytic Varieties}
\subsection{Real Analytic Varieties}

\subsection{Complex Analytic Varieties}
\begin{defn}{Complex Analytic Varieties}{} A complex analytic variety $Z$ is a subset of $\C^n$ that is the common vanishing locus $$Z=\V(f_1,\dots,f_k)=\left\{z\in\C^n\;|\;f_1(z)=\dots=f_k(z)=0\right\}$$ of analytic functions $f_1,\dots,f_k\in\mA_{\C^n}(U)$ defined on an open set $U\supseteq Z$. 
\end{defn}

\begin{defn}{The Structure Sheaf}{} Let$ U\subseteq\C^n$ be an open subset. Let $f_1,\dots,f_k\in\mA_{\C^n}(U)$ be analytic functions on $U$. Let $Z=\V(f_1,\dots,f_k)$ be the vanishing locus of $f_1,\dots,f_k$. Define the structure sheaf $$\mO_Z:\bold{Open}(Z)\to\bold{CRing}$$ as follows. 
\begin{itemize}
\item For each open set $U\subseteq Z$, define $$\mO_Z(U)=\frac{\mO_{\C^n}(U)}{(f_1,\dots,f_k)}$$
\item For each inclusion $U\subseteq V\subseteq Z$ of open sets, define $$\mO_Z(V)\to\mO_Z(U)$$ to be the induced map of the restriction $\mO_{\C^n}(V)\to\mO_{\C^n}(U)$. 
\end{itemize}
\end{defn}

\subsection{Complex Analytic Varieties of Manifolds}
\begin{thm}{Cartan's (Coherence) Theorem}{} Let $M$ be a complex manifold and let $A$ be an analytic subset of $M$. Then $\mI_A$ is a coherent sheaf of $\mO_M$-modules. 
\end{thm}

\subsection{Regular and Singular Points}
\begin{defn}{Regular and Singular Points}{} Let $M$ be a complex manifold and let $A$ be an analytic subset of $M$. We say that $x\in A$ is a regular point if there exists some open neighbourhood $U$ of $x$ such that $A\cap U$ is a complex submanifold of $M$. Otherwise, $x$ is said to be singular. Denote $$A_\text{reg}=\{x\in A\;|\;x\text{ is a regular point }\}\;\;\;\;\text{ and }\;\;\;\; A_\text{sing}=\{x\in A\;|\;x\text{ is a singular point }\}$$
\end{defn}

\begin{thm}{}{} Let $M$ be a complex manifold and let $A$ be an analytic subset of $M$. Then $A_\text{sing}$ is an analytic subset of $A$. 
\end{thm}

\pagebreak
\section{Analytic Spaces}
\subsection{Complex Analytic Spaces}
\begin{defn}{Complex Analytic Spaces}{} Let $(X,\mO_X)$ be a locally ringed space. We say that it is a complex analytic space if for all $x\in X$, there exists an open neighbourhood $U$ of $x$ such that there are isomorphisms of locally ringed spaces $$(U,\mO_X|_U)\cong(Z=\V(f_1,\dots,f_k),\mO_Z)$$ where $(Z,\mO_Z)$ is an analytic variety. 
\end{defn}

\begin{thm}{}{} Let $X$ be a complex space. Then $X_\text{reg}$ is dense and open in $X$. Moreover, $X_\text{reg}$ consists of a disjoin union of complex manifolds. 
\end{thm}

\begin{thm}{}{} Let $X$ be an irreducible complex space. Then every non-constant holomorphic function $f:X\to\C$ is an open map. 
\end{thm}

\begin{crl}{}{} Let $X$ be an irreducible compact complex space. Then every holomorphic function $f:X\to\C$ on $X$ is constant. 
\end{crl}






\end{document}