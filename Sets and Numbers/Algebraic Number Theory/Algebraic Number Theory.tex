\documentclass[a4paper]{article}

\input{C:/Users/liula/Desktop/Latex/Headers V1.2.tex}

\pagestyle{fancy}
\fancyhf{}
\rhead{Labix}
\lhead{Algebraic Number Theory}
\rfoot{\thepage}

\title{Algebraic Number Theory}

\author{Labix}

\date{\today}
\begin{document}
\maketitle
\begin{abstract}
\end{abstract}
\tableofcontents
\pagebreak

\section{Basic Field Theory Over $\Q$}
\subsection{Algebraic Number Fields}
\begin{defn}{Algebraic Number Field}{} An algebraic number field is a finite field extension $\Q<K$. 
\end{defn}

Let $K,L$ be two fields. Recall that any field homomorphism $K\to L$ is injective, and hence it makes sense to say that they are embeddings. 

\begin{defn}{Embeddings of Number Fields}{} Let $K$ be a number field. Let $\sigma:K\to\C$ be a field embedding. 
\begin{itemize}
\item We say that $\sigma$ is a real embedding if $\sigma(K)\subseteq\R$. 
\item Otherwise we say that $\sigma$ is a complex embedding. 
\end{itemize}
\end{defn}

\begin{lmm}{}{} Let $K$ be a number field. Let $\sigma:K\to\C$ be an embedding. Then $\sigma(a)=a$ for all $a\in\Q$. 
\end{lmm}

\begin{defn}{Signature of a Number Field}{} Let $K$ be a number field. Define the signature of $K$ to be $(r,s)\in\N\times\N$ where $r$ is the total number of real embeddings, and $s$ is one half of the total number of complex embeddings. 
\end{defn}

\begin{defn}{Quadratic Fields}{} Let $K$ be a number field. We say that $K$ is a quadratic field if $$[K:\Q]=2$$
\end{defn}

Recall that a number of square free if $d$ is not divisible by $m^2$ for any $m\in\N$. 

\begin{lmm}{}{} Let $K$ be a quadratic field. Then $K=\Q(\sqrt{d})$ for some unique square free number $d\neq 1$. 
\end{lmm}

\subsection{The Trace and Norm Map of Number Fields}
The trace and norm map simplifies (is it true?) when we consider number fields. 

Recall in Fields and Galois Theory that $\text{Emb}_\Q(K,\C)$ is defined to the the set of all field homomorphisms from $K$ to $\C$ fixing $\Q$. 

\begin{prp}{}{} Let $K$ be a number field. Then for any $\alpha\in K$, we have that $$\text{Tr}_{K/\Q}(\alpha)=\sum_{\sigma\in\text{Emb}_\Q(K,\C)}\sigma(\alpha)$$ Similarly we have that $$N_{K/\Q}(\alpha)=\prod_{\sigma\in\text{Emb}_\Q(K,\C)}\sigma(\alpha)$$
\end{prp}

\pagebreak
\section{Algebraic Integers}
\begin{defn}{Ring of Integers}{} Let $K$ be an algebraic number field. Define the ring of integers of $K$ to be $$\mO_K=\{\alpha\in K\;|\;\alpha\text{ is integral over }\Z\}$$ 
\end{defn}

\begin{lmm}{}{} Let $K$ be a number field. Then $$K=\text{Frac}(\mO_K)$$
\end{lmm}

\begin{defn}{Algebraic Integers}{} An algebraic integer is an element of $\mO_\C$. 
\end{defn}

\begin{prp}{}{} Let $\alpha\in\C$ be an algebraic number. Then $\alpha$ is an algebraic integer if and only if $\min(\Q,\alpha)=\min(\Z,\alpha)$. 
\end{prp}

\end{document}