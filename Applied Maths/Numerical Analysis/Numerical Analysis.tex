\documentclass[a4paper]{article}

\input{C:/Users/liula/Desktop/Latex/Headers V1.2.tex}

\pagestyle{fancy}
\fancyhf{}
\rhead{Labix}
\lhead{Numerical Analysis}
\rfoot{\thepage}


\title{Numerical Analysis}

\author{Labix}

\date{\today}
\begin{document}
\maketitle
\begin{abstract}
\end{abstract}
\tableofcontents

\pagebreak
\section{Numerical Techniques for Linear Algebra}
\subsection{Solving For Solutions}
For an $n\times m$ matrix $M$, we use the notation $a_{i,j}$ to mean the element at the $(i,j)$th position. We refer to the $i$th row as $R_i$ and the $j$th column as $C_j$. 

\begin{defn}{The Gauss-Jordan Elimination Method}{} Let $M\in M_{n\times m}(\R)$. The Gauss-Jordan elimination method computes the row echelon form of $M$ using the following algorithm. For $1\leq i\leq m$, repeat the following steps: 
\begin{enumerate}
\item Step 1: Find the pivot of the $i$th column. 
\item Step 2: Swap the the $i$th row with the row containing the pivot. 
\item Step 3: For $j\neq i$, replace $R_j$ with $R_j-\frac{a_{i,i}}{a_{j,i}}R_i$. 
\end{enumerate}
\end{defn}

\begin{defn}{The Gaussian Elimination Method}{} Let $M\in M_{n\times n}(\R)$. 
\end{defn}

\begin{defn}{LU Decomposition}{}
\end{defn}

\begin{defn}{QR Decomposition}{}
\end{defn}

\subsection{Singular Value Decomposition}
\begin{defn}{Singular Value Decomposition}{}
\end{defn}



\end{document}