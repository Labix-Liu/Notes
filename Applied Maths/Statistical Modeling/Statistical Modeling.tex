\documentclass[a4paper]{article}

\input{C:/Users/liula/Desktop/Latex/Headers V1.2.tex}

\pagestyle{fancy}
\fancyhf{}
\rhead{Labix}
\lhead{Statistical Modeling}
\rfoot{\thepage}


\title{Statistical Modeling}

\author{Labix}

\date{\today}
\begin{document}
\maketitle
\tableofcontents
\pagebreak

\section{General Methods in Statistical Modeling}
\subsection{Estimators}
\begin{defn}{Least Square Estimation}{}
\end{defn}

\begin{defn}{Maximum Likelihood Estimation}{}
\end{defn}

\subsection{Hypothesis Testing}
\begin{defn}{Null Hypothesis}{}
\end{defn}

\begin{defn}{Alternative Hypothesis}{}
\end{defn}

\pagebreak
\section{Regression Analysis}
We use regression analysis in a certain pattern recognition problem when the dependent variable is real-valued. 

\subsection{Linear Regression}
\begin{defn}{Linear Regression Model}{} A linear regression model assumes the relationship between the dependent random variable $Y$ and the independent random variables $X_1,\dots,X_n$ are linear. In other words, there exists $\beta_0,\dots,\beta_n\in\R$ together with an unobserved random variable $\varepsilon$ such that $$Y=\beta_0+\sum_{k=1}^n\beta_kX_k+\varepsilon$$
\end{defn}

\subsection{Logistic Regression}

\pagebreak
\section{Statistical Classification}
We use statistical classification methods in a certain pattern recognition problem when the dependent variable is discrete. 

\subsection{Classification following Logistic Regression}






\end{document}